\documentclass{report}

\begin{document}
\chapter{Midtvejs process Analyse}
\section{L\ae{}ringsm\aa{}l}
\begin{itemize}
\item L\ae{}r at l\o{}se Rubiks Cuben.
\item L\ae{}re at programmere i Java.
\item L\ae{}re  og forst\aa{} det bagved liggende matematik i Rubiks Terningen. 
\item L\ae{}e gruppe- og grafteori.
\end{itemize}

\section{Motivation}
I starten af projektet var motivationen rigtig god og nu er den generelle motivation stigende. 

\section{Videndeling}
Vi snakker sammen hver dag og diskuterer vores opdagelser. Plus SVN.
 Dette fungerer fint. Ingen \ae{}ndringer. 

\section{Projekt organisering}
En stor del af den organisatoriske strukturering af projektet tager form af en v\ae{}g kalender. Dette havde vi ikke med  i starten men er tilf\o{}jet senere. 

\section{Vejledersamarbejde}
VI har et udm\ae{}rket forhold til vores vejledere og vi f\aa{}r udvekslet vores viden og f\o{}ler at vores vejledere virkeligt st\o{}tter os. Vejleder samarbejdet med den ene vejleder er sat i bero da vejleder selv mener at denne ikke kan bidrage med mere f\o{}r til slutningen af projektet. 

\section{Konflikth\aa{}ndtering}
Problemer l\o{}ses ad-hoc. Denne metode har fungeret fint indtil videre.

\end{document}