\chapter*{Preface}
\addcontentsline{toc}{chapter}{\numberline{}Preface}
\emptyTop{}
%\textit{Note: Remember to thank Herbert Kociemba for his e-mail. Thank Anders and Heather and Leif for help with litterature. Prerequisites for reading this report: A Rubik's Cube, knowledge of graph and group theory. Citations are with square brackets. Trademarks on stuff is omitted. The longterm scale for long numbers will be used.}

This report is written by seven software engineer students attending Aalborg University, group A215, and is associated with a P2 project in 2010.
The main theme of this semester is ``Network and Algorithms" and our project deals with the \rubik{}.
%The purpose of this project is to obtain a better understanding of how a report is written as a group and what it is like to 
The course of the project was commenced on \myDate{1}{2}{2010}, and the paper was handed in on \myDate{27}{5}{2010}.

Readers are required to have a basic understanding of programming and mathematics in order to properly understand the content of the report. 

The report is divided into four parts; Introduction, Theory, Implementation and Discussion. Every chapter except for the preface and problem definition starts with a head and ends with a tail. They act as an introduction and a summary. Heads and tails are written in italic and are separated from the rest of the report by lines. 
Additionally we have produced an application, which is described further in the report and appended on a CD-rom.

All citations are written in square brackets such as [xx]. The number is a reference to the source, which can be found in the bibliography on page \pageref{chap:bib}. 
\rubik{} is a registered trademark, but the trademark symbol is omitted during the report.

All figures are written in long scale.%NOTE: figures betyder tal

We, the authors, would like to use this opportunity to thank our guidance councilor and assistant guidance councilor for their assistance in finding credible sources, and guidance, and generally aiding in the process of writing this report. 

We would also like to thank Leif Kj\ae{}r J\o{}rgensen and Diego Ruano Benito for assisting us in obtaining usable sources concerning group theory.
Furthermore we would like to thank Herbert Kociemba for his response to the e-mail we sent him.
