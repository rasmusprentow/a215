This is the last step of the solving process and only the corner \cpiece{}s in the last layer need to be orientated correctly. This is more generally described in subsection \ref{sub:step5}

The algorithm takes one corner \cpiece{} at the time and checks its orientation. If the corners are not orientated correctly \vr{algorithm 11} is applied, this algorithm rotates a corner without ruining the rest of the \cube{}. When the corner is correctly orientated the \m{U} face is \twist{}ed so a new corner is placed in this position and the same procedure is applied. After four \m{U} turns and appliance of \vr{algorithm 11} the cube is finally solved. 

\begin{comment}
In the last layer the corners in the last layer were poisitend correctly but not oreiented. In this step will the coreners be oreinted correctly and as result it will lead to that the \rubiks{} will be solved.

This step is very simple because it there is only four corners to control and either the corner is oriented correctly or is isn't.  

The program vil first control that the front-right-up corner is oriented correctly if not the will use the an algorithm twice and after the program vil control the corner again if the corner is not oriented correctly his time the use teh algorithme and will continue with this until the corner is oriented correctly. 
Then the corner is orented correctly the program will make at up move ("U") and will control the new corner and the program will do this with every corner in the last layer until they are oriented corectly.  

\end{comment}
