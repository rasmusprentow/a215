\chapter{Fitting the Rubik's Cube into the computer}
A \rubik{} is a rather complicated 3-dimensional structure and fitting this structure into a computer system is quite complicated. 
The \rubik{} is build up by 26 \cpiece{}s held together by each other. 
This type of structure is not obvious to the computer. If the cube is depicted in the computer in a two-dimensional space the original \rubik{} structure gets even more out of hand. 

A simple way though to handle the \rubik{} in the program is a two dimensional depiction and  just simply move the \facelet{}s around.
But this approach will be far from reality and will make the implementations of solving algorithms more complicated.

An object-oriented approach to the problem will give a more useful structure. 
Dividing the cube into its sub structures. 
The cube consist of the 6 faces each with 9 shared \cpiece{}. 
A face also consist of 9 cubicle which acts as placeholders for the \cpiece{}s. 
There are 3 types of \cpiece{} and cubicles; corner, edge and centers. 
The center \cpiece{}s will never move and therefore there are no reason to define a \cubicle{} and a \cpiece{} for those. Instead the face is simple granted a \facelet{}.

This structure allows for the same \cubicle{} to be in several faces. 
2 faces for edges and three for corners. 
In this structure moving a face will swap the \cpiece{}s around on the \cubicle{}(See some picture) This makes the adjacent face to the moving face, get some of its \cpiece{}s get swapped as well. 
For example when twisting the up face of a \rubik{} will inflict the move the \cpiece{}s of the R, L, B and F faces. 

The problem of this structure though is it requires variables to determined how the \facelet{} should be orientated. 
How this is done is describes in chapter ref\{Somewhere\}.
