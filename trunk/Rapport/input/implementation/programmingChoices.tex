\chapter{Choices Prior to Implementation}
\label{cha:choicesPriorToImplementation}
%\myTop{This chapter, along with the rest of this part, will be dedicated to the last question of our problem statement -- how we will test the different solving algorithms described in chapter \ref{cha:solvingAlgorithms}.}
\myTop{This chapter presents our approach to test the two algorithms from chapter \ref{cha:solvingAlgorithms}. Furthermore the choices prior to the implementation will be described.}
We want to test the two algorithms from chapter \ref{cha:solvingAlgorithms}.
This can be done in two ways: Making the tests by hand or implementing the algorithms in an application, which can perform the tests.
To test Kociemba's optimal solver and beginner's algorithm we will create an application that has a digitalized \rubik{}, which the user can permute and scramble.
In this application the algorithms will be implemented for testing.


Before we start implementing our application we have to make some choices, namely: Programming language, version control, and which User Interface (UI) we want.
These choices and the dilemmas, which we encountered are presented here.

\begin{itemize}
	\item The first issue we will discuss is, which language we want to write our application in.
The most important issue regarding the programming language is that all the members of the group have a knowledge of the language prior to the implementation phase.
This leads us to choose Java as our programming language since we all have been following a course in this language.

	\item Secondly we have to choose a way to share our source code, since we want to work in smaller subgroups and then collect the work when we complete a task.
For this we choose to use subversion (SVN) because we have all been using this protocol while working on the report and therefore know how to use it.

	\item We have to consider how to represent the \rubik{} in a UI. The choice is between Graphical User Interface (GUI) and Textual User Interface (TUI). We choose a GUI because it will be easier to see how the \rubik{} permutes in our application.
	  
	\item Finally we have to decide how much time we want to spend on the GUI since we do not consider it as an important part of our project according to our problem limitations (see section \ref{sec:problemLimitations}).
We choose to make a simple GUI with the \rubik{} to the left, buttons to right, and a console in the bottom.
The technical documentation of the creation of the GUI will be omitted.
\end{itemize}
\myTail{This chapter presents the choices which we make before we start our actual implementation. What approach we will take to test Kociemba's optimal solver and the beginner's algorithm is described.}
