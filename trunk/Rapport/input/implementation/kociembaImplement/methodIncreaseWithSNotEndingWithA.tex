\section{The Move Incrementing Methods}
\label{sec:increaseWithSNotEndingWithA}
The move incrementing methods are vital parts of Kociemba's optimal solver. 
They ensure that every possible move sequence is tested.
The methods increment a move sequence. This means that it will find the next move in an enumset of all possible moves. See section \ref{sec:incMoveSequence}.


When the method, named \vr{increaseWithSNotEndingWithA}, is called it takes two parameters, a move sequence and an integer.
The integer decides where in the move sequence the move should be incremented. 


\begin{lstlisting}[style=sourceCode, caption=\myCaption{Key point in the incrementing method of kociemba's optimal solver}, label=src:kociemba2]
.
.
.
if (i == length - 1) {
	try {
		do {
			moveSequence[i] = (MoveButtons)S.toArray()[moveSequence[i].ordinal() + 1];
		} while(A.contains(moveSequence[i]));
		try {
			if(isSameFace(moveSequence[i], moveSequence[i-1])) {
				increaseWithSNotEndingWithA(moveSequence, i);
			} else {
				Cube.permute(cube, moveSequence[i]);
			}
		} catch (ArrayIndexOutOfBoundsException e2) {
			Cube.permute(cube, moveSequence[i]);
		}
		return;
	} catch (ArrayIndexOutOfBoundsException e) {
		moveSequence[i] = F;
		i--;
		try {
			Cube.permute(cube, moveSequence[i].invert());
			moveSequence[i].invert();
		} catch (ArrayIndexOutOfBoundsException e4) {
			throw new UnableToIncreaseMoveSequenceException();
		}
		increaseWithSNotEndingWithA(moveSequence, i);
		return;
	}
} else { ...
\end{lstlisting}

The last move performed on the cube can never be an \m{A} move since \m{H} is a closed group, which means that applying \m{A} moves to a cube in \{H} will never make the cube leave \{H} (see section \ref{sec:subgroup}). The part of the code that increments the last move is shown in code snippet \ref{src:kociemba2}.
At first a \textbf{do-while} loop is executed. 
This loop adds a move from the enumset \m{S} (see code snippet \ref{src:enumset}, enumset \m{S}), which is not an \m{A} (see code snippet \ref{src:enumset}, enumset \m{A}) move.

After the loop has run a new method \vr{isSameFace} is called.
This method tests if the new move is on the same face as the move before.
If the moves are on the same face the latter of the two moves will be incremented untill they are no longer on the same face.
This test is only done on the last two moves, so they are not on the same face (see section \ref{sec:groupDefinition}).

When all the moves in the enumset \m{notA} (see code snippet \ref{src:enumset}, enumset \m{notA}) have been used it will start over from the beginning of the enumset \m{notA}.

\begin{lstlisting}[style=sourceCode, caption=\myCaption{The definition of the enumsets S, A, and notA.}, label=src:enumset]
private EnumSet<MoveButtons> S = EnumSet.of(U, UP ,U2, D, DP, D2, F, FP, F2, B, BP, B2, L, LP, L2, R, RP, R2);
private EnumSet<MoveButtons> A = EnumSet.of(U, UP ,U2, D, DP, D2, F2, B2, L2, R2);
private EnumSet<MoveButtons> notA = EnumSet.of(F, FP, B, BP, L, LP, R, RP);
\end{lstlisting}

All the code in code snippet \ref{src:kociemba2} is executed if it is at the last move in the move sequence.
If it is not at the last move the moves in the move sequences will be incremented with the moves from the enumset \m{S}.
This ensures that all possible move sequences not ending with \m{A} will be tested.

There is a similar method named \vr{increaseWithA} that uses a very similar code.
This is called from the \vr{solveFromH} method.
The difference is that it only increments with moves from \m{A}.
