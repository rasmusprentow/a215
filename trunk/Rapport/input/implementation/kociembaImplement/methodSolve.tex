\section{Method Solve}

- Initialiserer 5 variabler af forskellige typer
- hopper ind i en while l\o{}kke
- hvergang while l\o{}kke bliver k\o{}rt bliver et array initialiseret lavet af moves.
- twist
- to moves af samme type
-  humle = pilsen knappes
- hopper ind i l\o{}kke der k\o{}rer uendeligt
- hvor den permuter med moveseq b
- k\o{}rer lignende funktion , som hedder solveFromH
- g\o{}r den med en parameter l - d
- resultat den f\aa{}r s\ae{}tter den = c
- tester om den nuv\ae{}rende l\ae{}ngde d + l\ae{}ngden c < l
- hvis ja l = de to foreg\aa{}ende l\ae{}ngder
- initialiserer resultat arrayet
- tilf\o{}jer moveseq b og moveseq c til moveseq resultatet
- permuter cuben tilbage igen, inverse af b

- increasewithSnotendwithA --> reference
- t\ae{}ller d op
- returner resultatet i sidste ende.
-  og der er nogle try catch som s\o{}rger for at det hele ikke l\o{}ber i endless loop

The method Solve is the method that connects all the submethods into one big method that apply the Kociemba's algorithm onto the \rubik{}. It start with initializing five variables; a result array which will contain the final move sequence, a d integer which will determine the search depth, a l integer which also is used to determine the search depth, a b array which will contain the move sequence until the \rubik{} enters H and a c array which will be the move sequence after the \rubik{} has entered H. Thereafter it enters a while loop which initialize an array of moves everytime the loop runs. Thereafter the method enters a while loop which runs in a endless loop where it permutes with the move sequence b. The permute method BLALABLABLA. After that it will try solve the \rubik{} from H with the method of the same name, solveFromH, with the parameter l - d.  This will return a result that will be putted in the c array. Thereafter it test if the length d + c's length is lower than the length of l. If that is true the case l will be set as the product of d + c and the result array will be initialized with l. In The result array the move sequence b and the movesequence c is added. after it has tested the length the \rubik{}  permutes the b move sequence.



try {
				while (true) {
					//System.out.println("Permuting!");
					Cube.permute(cube, b);
					try {
						c = solveFromH(l - d);
						//c = solveFromH(5);
						if (d + c.length < l) {
							l = d + c.length;
							result = new MoveButtons[l];
							output.addTextln("The solutions of the length " + l + ". The solution is:");
							System.out.println("The solutions of the length " + l + ". The solution is:");
							int j = 0;
							for ( ; j < d; j++) {
								result[j] = b[j];
								output.addText(b[j] + " ");
								System.out.print(b[j] + " ");
							}
							for (int k = 0 ; k < c.length; k++,j++) {
								result[j] = c[k];
								output.addText(c[k] + " ");
								System.out.print(c[k] + " ");
							}
							output.addTextln("");
							System.out.println();
						}

					} catch (InvalidCube e) {
						//System.out.println("Not in H!");
					}
/*
					System.out.print("S: ");
					for (int i = 0 ; i < b.length; i++) {
						System.out.print(b[i] + " ");
					}
					System.out.println();*/
					
					Cube.permute(cube, MoveButtons.inverseOf(b));
					MoveButtons.inverseOf(b);					
					
					increaseWithSNotEndingWithA(b, b.length-1);
				}
			} catch (UnableToIncreaseMoveSequenceException e) {

			}
			d++;
		}
		return result;
	}
