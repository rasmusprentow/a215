\section{Method Solve}

The method \textit{solve} connects all the submethods into one big method that apply Kociemba's algorithm onto the \rubik{}. It start with initializing five variables; a \textit{result} array which will contain the final move sequence, an integer \textit{d} which will determine the search depth, an integer \textit{l} which also is used to determine the search depth, an array \textit{b} which will contain the move sequence until the \rubik{} enters H and an array \textit{c} which will be the move sequence after the \rubik{} has entered H. Thereafter it enters a while loop, which initialize an array of moves and permutes the \rubik{} with these everytime the loop runs. After that the method enters a while loop which runs in an endless loop. The permute method applies the moves to the cube. After that it will try solve the \rubik{} from H with the method of the same name, \textit{solveFromH}, with the parameter \textit{l - d}.  This will return a result that will be putted in the \textit{c} array. Thereafter it test if the length \textit{d} + \textit{c}'s length is lower than the length of \textit{l}. If that is true \textit{l} will be set as the product of \textit{d} + \textit{c} size and the result array will be initialized with \textit{l}. In The \textit{result} array the move sequence \textit{b} and the movesequence \textit{c} is added. after it has tested the length the \rubik{}  permutes the \textit{b} move sequence back one depth and perform the \textit{increaseWithSNotEndingWithA} method which will be described in section BLABLA( REF HER!!!!!!!!). After the  \textit{increaseWithSNotEndingWithA} method the program counts the \textit{d} integer up by one. When the program is unable to find a solution length shorter than the last solution it will return the \textit{result} array.


\begin{verbatim}
			
			try {
				while (true) {
					try {
						c = solveFromH(l - d);
\begin{verbatim}
						if (d + c.length < l) {
							l = d + c.length;
							result = new MoveButtons[l];
							output.addTextln("The solutions of the length " + l + ". The solution is:");
							int j = 0;
							for ( ; j < d; j++) {
								result[j] = b[j];
								output.addText(b[j] + " ");
							}
							for (int k = 0 ; k < c.length; k++,j++) {
								result[j] = c[k];
								output.addText(c[k] + " ");
							}
							output.addTextln("");
							output.addTextln("Time spend: " + ((curTime - startTime)/1000) + " seconds");
						}
\end{verbatim}
Thereafter it test if the length \textit{d} + \textit{c}'s length is lower than the length of \textit{l}. If that is true \textit{l} will be set as the product of \textit{d} + \textit{c} size and the result array will be initialized with \textit{l}. In The \textit{result} array the move sequence \textit{b} and the movesequence \textit{c} is added.

\begin{verbatim}
					} catch (InvalidCube e) {}
					try {
						Cube.permute(cube, b[d-1].invert());
						b[d-1].invert();
					} catch (ArrayIndexOutOfBoundsException e) {}
					increaseWithSNotEndingWithA(b, d-1);
					
\end{verbatim}
If the cube has not entered H the program will catch and exception called \textit{InvalidCube}. If that is the case the program will permute the cube back by the lenght of d.
			}
			} catch (UnableToIncreaseMoveSequenceException e) {}
\end{verbatim}
\begin{verbatim}
			d++;
		}
		return result;
	}
\end{verbatim}

 After the  \textit{increaseWithSNotEndingWithA} method the program counts the \textit{d} integer up by one. When the program is unable to find a solution length shorter than the last solution it will return the \textit{result} array.