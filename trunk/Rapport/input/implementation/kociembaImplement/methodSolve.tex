\section{The Solving Method}
\label{sec:kSolve}
The method named \vr{solve} (see code snippet \ref{src:kociemba} for a key point) acts like a main routine for the solving mechanism of Kociemba's optimal solver. By calling this method Kociemba's optimal solver will be applied to the \rubik{}. 

The solver's approach to solving the \rubik{} is trying every move sequence not tried so far. If the cube is inside \m{H} it will start using only \m{A} moves. Every time a shorter move sequence is found it is saved. 

\begin{lstlisting}[style=sourceCode, caption=\myCaption{Key point in the solve method of Kociemba's optimal solver}, label=src:kociemba]
.
.
.
try {
	c = solveFromH(l - d);
	if (d + c.length < l) {
		curTime = System.currentTimeMillis();
		l = d + c.length;
		result = new MoveButtons[l];
		output.addTextln("The solutions of the length " + l + ". The solution is:");
		int j = 0;
		for ( ; j < d; j++) {
			result[j] = b[j];
			output.addText(b[j] + " ");
		}
		for (int k = 0 ; k < c.length; k++,j++) {
			result[j] = c[k];
			output.addText(c[k] + " ");
		}
		output.addTextln("");
		output.addTextln("Time spend: " + ((curTime - startTime)/1000) + " seconds");
	}
} catch (InvalidCubeException e) {}
...
\end{lstlisting}

The \vr{solve} method starts with initializing five variables; 
\begin{itemize}
\item The array \vr{result} which will contain the final move sequence.
\item The integer \vr{d} which determines the search depth. 
\item The integer \vr{l} which determines the length of the current solution. 
\item The array \vr{b} which will contain the move sequence until the \rubik{} enters \m{H}. This is initialized with the length zero; an empty move sequence. 
\item The array \vr{c} which will be the move sequence after the \rubik{} has entered \m{H}.
\end{itemize}

After the initialization the \vr{solve} method calls the \vr{solveFromH}, which solves the cube if the cube is in \m{H}. 
When the cube has been solved from \m{H} it test if the length of the new solution is better than the old solution this will be saved as the new solution. 
This can be seen in code snippet \ref{src:kociemba}.
If the cube is not in \m{H}  the exception \vr{InvalidCubeException} will be thrown.
When this exception is caught it forces the \vr{solve} method to skip to the next part of the algorithm. 
%If the new move sequence (\vr{b+c}) is shorter than the currently shortest move sequence (\vr{result}) the new one will be saved in \vr{result} and \vr{l} is set to be the length of the new move sequnce.
\begin{comment}
 method to test if the cube is in \m{H}. 
 The method \vr{solveFromH} is called with a parameter, which determines how deep it will search.

If the \rubik{} is in \m{H} then \vr{solveFromH} will start solving it with \m{A} moves. If the \rubik{} is not inside \m{H} the \vr{solveFromH} will throw an exception called \vr{InvalidCubeException}. 

When the InvalidCubeException is caught by it cause 

\end{comment}

After the code snippet \ref{src:kociemba} the method will permute the \rubik{} back with the inverse of the last move. In first run this is redundant since the move sequence, \vr{b}, is empty. 
When it has been permuted back the method \vr{increaseWithSNotEndingWithA} will be called. 
See section \ref{sec:increaseWithSNotEndingWithA}. 
The new increased move sequence will now be executed and the part displayed in code snippet \ref{src:kociemba} will be executed again. 

When the method \vr{increaseWithSNotEndingWithA} has been through every move sequence of the length \vr{d} it will throw an exception of the type  \vr{UnableToIncreaseMoveSequenceException}. 
The \vr{solve} method will catch this exception and increment \vr{d}. When \vr{d} is incremented to \vr{l} the method will return the array \vr{result} which contains the shortest move sequence to the given \rubik{}.  
\begin{comment}
Thereafter it tests if \vr{d + c.length} > \vr{l}. 
If that is true, \vr{l} will be set equal to the sum of \vr{d + c.length} and the \vr{result} array will be initialized with the size \vr{l}. 
In the \vr{result} array the move sequence \vr{b} and the move sequence \vr{c} is added and the console will print the current \vr{result} and the time it took in seconds.

After the method has tested if \vr{d + c.length < l} the method will permute the \rubik{} back with the inverse of the last move. When it has been permuted the method \vr{increaseWithSNotEndingWithA} will be called. See section \ref{sec:increaseWithSNotEndingWithA}.

When the method \vr{increaseWithSNotEndingWithA} throws an exception of the type \vr{UnableToIncreaseMoveSequenceException} the method increments \vr{d}. 
When \vr{d} is incremented to \vr{l} the method will return the \vr{result} array.
\end{comment}