\chapter{Introduction}
\emptyTop{}
Since 1977, when the \rubik{} was initially released for sale, the \rubik{} has frustrated, inspired and entertained many people. This 3x3x3 cube has so many possible settings that the solution can not just be guessed out of sheer luck. Because of this a community around solving the \rubik{} has emerged. The community is divided into two groups both concerning efficient solving -- one efficient time-wise and the other efficient twist-wise i.e. solving in the least amount of time and solving in the least amount of twists \cite{speedsolving.forum}. 

The group concerning time-wise efficiency, often referred to as speedcubing is the largest group of the community and the majority of the competitions held by the WCA\footnote{WCA, World Cube Association, is the official organization for Rubik's Cube related competitions.} \cite{wca} revolve around speedcubing.

The first official competition was held in 1982 in Hungary and is regarded as the first World championship. Since 2002 there have been held annual world championships and plenty other events concerning speedcubing. 

The group of the community concerning twist-wise efficiency is much smaller than the speedsolving group. The majority of the research in the twist-wise efficient area is published as scientific articles explaining the algorithms. Even though competitions with the goal of the least amount of twists to solve the cube are held, many of the twist-wise efficient algorithms are not useful for human solving. These algorithms rely on computer power to look through a large amount of possibilities, which is not a viable option for a human competitor.


The ultimate goal for the twist-wise efficiency community is to find \textit{god's algorithm}, which is the algorithm that solves the cube in the absolute least amount of twists from any given position. A part of finding god's algorithm is to find the amount of moves needed to perform it. 
The upper bound of the \rubik{} is the proven number of twists needed to solve the cube from any position. The lower bound is the least number of twists required to solve the cube in the currently known worst case scenario. 