\chapter{Problem Definition}
\myTop{The following problem statement and problem limitation is based on the short introduction from the previous chapter.
Since the problem of the \rubik{} is not a society oriented no problem analysis is given.}
\section{Problem Statement}
What are the current upper and lower bound of the \rubik{} and how have they been proven? \newline\newline
Which solving algorithms exist and which is the most efficient with respect to the number of twists? \newline\newline
How can we create an application which can solve the \rubik{}?

\section{Problem Limitations}
Because the amount of different algorithms for \rubik{} solving, not every algorithm will be covered in this project.
For the reason of comparison we will choose to look at algorithms which takes different approaches to solving the \rubik{}, e.g. an algorithm which is easy to remember vs. an algorithm which is efficient with respect to the number of twists.

The \rubik{} solving algorithms will be primarily for technical use, meaning that usability will not be in focus.

Our application must be able to solve a \rubik{} using only operations which are possible on an actual \rubik{}.
It must also be able to generate a scrambled \rubik{}.
The application has to be able to present the \rubik{} in a graphical way, but the details of this will not be covered and it is not considered as an important part of the project.

The theory in this report will only present the parts which is being used to solve our problem statement, and it will not cover every aspect of the given subject.