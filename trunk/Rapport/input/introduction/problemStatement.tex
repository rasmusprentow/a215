\chapter{Problem Definition}
%\myTop{The following problem statement and problem limitation is based on the short introduction from the previous chapter.Since the problem of the \rubik{} is not a society oriented no problem analysis is given.}
\emptyTop{}
\section{Problem Statement}
The following problem statement has been defined:

\vspace{2mm}
\begin{centering}
\hspace{2mm}
\framebox[\textwidth - 6mm]{
\parbox{\textwidth - 12mm}{
\vspace{2mm}
\textit{What are the current upper and lower bounds of the \rubik{} and how have they been proven? \newline\newline 
Which solving algorithms exist and how efficient are they? \newline\newline
How can we create an application which can solve the \rubik{}?
\vspace{2mm}
}

}}
\end{centering}
\section{Problem Limitations}
\label{sec:problemLimitations}
Since the amount of different algorithms for \rubik{} solving is too large for this project, not every algorithm will be covered.
For the reason of comparison we will choose to look at two algorithms which take different approaches to solving the \rubik{}. We will choose an algorithm which is easy to remember and an algorithm which is efficient with respect to the number of twists. When studying the efficiency of the algorithms the focus will be on twist-wise efficiency.

The \rubik{} application will primarily be for technical use, meaning that usability will not be in focus.

Our application must be able to solve a \rubik{} using only operations which are possible on an actual \rubik{}.
It must also be able to generate a scrambled \rubik{}.
The application has to be able to present the \rubik{} in a graphical way, but this will not be a primary concern and is considered irrelevant to our problem statement. 

% the details of this will not be covered and it is not considered as an important part of the project.
The theory in this report will only present the parts which is being used to solve our problem statement, and it will not cover every aspect of the given subjects.