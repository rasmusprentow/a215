\subsection{Magic Cube}
A Magic Cube\cite{Heinz09} is created from squares put on top of each other so they make up a cube form. 
This makes it clear that there is a connection between Magic Squares and Magic Cubes.
An example of this can be seen on figure \ref{fig:presentMagicCube}.

\begin{figure}[h]
	\centering
		\includegraphics[scale=0.8]{\picturepath{}presentMagicCube}
	\caption{\myCaption{This is a magic cube split up into 3 magic squares.}}
	\label{fig:presentMagicCube}
\end{figure}

Both of them have a magic constant, which can be the sum of each row, colon and diagonal if it is a Magic Cube or normal Magic Square.
This i where the similarity ends.

We have shown how to calculate the magic constant in a Magic Square.
In a Magic Cube there is not a big differens in the formula to calculate the magic constant.
\begin{equation}
	M(n)=\frac{n \cdot (n^3+1)}{2}
\end{equation}
As shown in the formula the only diffenrence is the power of $n$ that is changed from 2 to 3.
This will be show in appendix \ref{sec:proofOfMagicConstant} why this is correct.

To create a Magic Cube, there is some parts that needs to be explanied.
All these basics is shown on figure \ref{fig:cubeparts}.

\begin{figure}[h]
	\centering
		\includegraphics[scale=0.4]{\picturepath{}cubeparts}
	\caption{\myCaption{This is a Magic Cube where the colors show all of the parts.}}
	\label{fig:cubeparts}
\end{figure}

Because of all these different parts there is a lot of different ways to define Magic Cubes.
The simplets of them all is a simpel Magic Cube, the only requerements to make such is the following:
\begin{itemize}
	\item All 9- rows, columns and pillars must equal the magic constant.
	\item All 4 triagonals must also equal the magic constant.
\end{itemize}

