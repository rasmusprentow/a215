\subsection{Magic Square}
\label{sec:magicSquare}
A \msquare{} is a square which is divided into a number of sub squares. The number of sub squares on any side is referred to as the ``order'' of that \msquare{}. In each sub square there is a positive integer. In order for the \msquare{}, to be ``magical'', the sum of any row, column or diagonal must be the same, this sum is referred to as the magic constant.

The \msquare{}\cite{aiden06} originates from ancient China. It was said that the people near the river Lo made offerings. Every time they made an offering a tortoise emerged from the river. On the back of the tortoise there was said to be a \msquare{}.

The \msquare{} from this tale was a 3 order \msquare{}. This is not the only order in which a \msquare{} can be created; it is possible to make an ``$n$'' order \msquare{}. Although it has been proven that it is not possible to make a second order \msquare{}.

In order to solve the \msquare{}, it is needed to know the magic constant -- the constant which every row, line and diagonal adds up to for the given order $n$. This constant can be computed with the formula in \ref{align:magicConstant}.

\begin{align}
\label{align:magicConstant}
	M(n) = \frac{n \cdot (n^2+1)}{2}
\end{align}

The proof of this formula is quite straight forward. As the table \ref{tab:magicSquareOrder3} illustrates, a \msquare{} of the order 3 contains the numbers from 1 to 9. Generally a \msquare{} of the order $n$ contains the numbers from 1 to $n^2$.

The sum of the numbers of a row in a \msquare{} is equal to the magic constant. If the magic constant is multiplied by the order $n$ it would be equal to the sum of all the integers, since each number only occurs once in a \msquare{}.

\renewcommand{\arraystretch}{1.3}
\begin{table}[h]
	\centering
		\begin{tabular}{|c|c|c |@{\vrules}| c|}
			\hline
			6&1&8&15 \\
			\hline
			7&5&3&15 \\
			\hline
			2&9&4&15 \\
			\noalign{\hrules}
			15&15&15&45 \\
			\hline
		\end{tabular}
	\caption{\myCaption{A Magic Square of the order 3, by adding the three numbers in any row, column or diagonal, the magic constant is seen to be 15}}
	\label{tab:magicSquareOrder3}
\end{table}

The equation \ref{proof:magicConstant1} can be rewritten into the equation \ref{proof:magicConstant2}(See proof of the right hand side transcription in appendix X).

\begin{align}
\label{proof:magicConstant1}
	n \cdot M \left( n \right) = \sum ^{n^2}_{i = 1} i = 1 + \cdots + n^2
\end{align}
\begin{align}
\label{proof:magicConstant2}
	n \cdot M \left( n \right) = \frac{n^2 \cdot \left( n^2 + 1 \right)}{2} \\
\label{proof:magicConstant3}
	M \left( n \right) = \frac{n \cdot \left( n^2 + 1 \right)}{2} 
\end{align}

The equation \ref{proof:magicConstant3} shows the function which gives the magic constant for a \msquare{} of the order $n$.

Variations of the \msquare{} exists. For example the numbers which can be inserted into the sub squares, could exceed $n^2$. This would change the magic constant. A \msquare with the integers 1 to $n^2$ within its sub squares is called a ``Normal \msquare''.