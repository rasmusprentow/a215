%\documentclass{report}\begin{document}
\chapter{Community}
\myTop{
In this chapter the community concerning the original \rubik{} and other similar puzzles will be presented and described briefly. This chapter will include a description of the online and offline community and the competitions in which members of the community partake. The community is an important and interesting topic because it is the primary place where solving strategies and algorithms have been produced, presented and discussed. 
}

The \rubik{} community consists of both a real life community handling competitions and events and an online community where cubers can find the real life competitions, improve their skills and talk to each other. The majority of the community is focussed on the speed cubing aspect. In this chapter both the online and offline community will be discussed. Note that the two sides of the community often interfere with each other. 

\section{The Online Community}
In the world of the \rubik{} there is a large online community. The community consist of everything from forums and guides to competitions and bragging. Cubers, as they often refer to them selves as, have a place that is the online community to express and compare their abilities, experiences and skills with each other. 

\subsection{Forums}
Forums give the cubers a place for sharing their knowledge and experience\cite{speedsolving.com}\cite{speedcubing.dk}\cite{wca}. Forums or specific \rubik{} sites allow visitors to find information concerning the cube. The main focus of these forums is on how to solve the Rubik's cube or similar puzzles in the least amount of time. Forums are often split into several divisions. Some concerning the hardware used and maintenance techniques for making the Rubik's cube spin with greater ease giving the cuber a slightly better solve time.

A major part is reserved for the theory and the algorithms used to solve the cube -- most of which strive for an efficient solve both with respect to time and number of \twist{}s.
Another part of the community is focused on competitions of various kinds. The offline competitions are held in cooperation with the World Cube Association(WCA) \cite{wca}, which are further discussed in section \ref{sec:wca}. Beside the WCA-competitions some forums hold weekly online competitions where the forum members can upload and compare their solve times for the \rubik{} or similar puzzles. 

Other than the forums the online community offers a wide variety of sites containing guides, solutions and algorithms for solving the cube. The majority of the \rubik{} sites contain the beginner's guide\cite{jasminLee08} whose target group is the beginners who may recently have gotten their first cube and want to learn how to solve it. 

\section{Competitions}
\label{sec:wca}
Speed cubing competitions are held on a regular basis\cite{wca/competitions}. These competitions have different disciplines for various puzzles related to the original 3x3x3 \rubik{}. All the official competitions are held in cooperation with the World Cube Association (WCA). The WCA governs the official regulations on speed cubing and holds annual world and regional championships. The first World championship in speed cubing was held in 1982 in Budapest, Hungary. WCA governs the official rankings and records for solving the Rubik's cube. In total WCA keeps regulation, ranking and records for 19 different types of competitions. All 19 competitions include puzzles which are related to or based on the original \rubik{}. 
\myTail{
In this chapter the community in regards to the \rubik{} has been described. It has been stated that the community consists of both online and offline aspects. In general the strategies and theories are produced and discussed online and they are for one thing put to use in competitions where the right solving method is crucial in order to finish in a competitive time.
}
%\end{document}