\chapter{Problem Analysis}
Since 1977, when the \rubik{} was initially released for sale, the \rubik{} has frustrated, inspired and entertained many people. This 3x3x3 cube has so many possible settings that the solution can not just be guessed out of sheer luck and since 1982 people has been competing each other in solving the \rubik{} fastest or by the least number of twist. Because of these competitions, it has been interesting for the competitors to find algorithms for solving the \rubik{} in the least number of twists. The development of these algorithms is an ongoing process which has given the latest theory in 2008, that states that an algorithm which can solve the \rubik{} in 22 twists, no matter which setting the \rubik{} starts in, is possible to create. No such algorithm has been created so far \cite{rokicki09}.

It could be interesting to study the implementation of the solving algorithms in a computer program. The efficiency of these algorithms with respect to the time of calculation and the number of twists is an interesting focus point.

\section{Problem Statement}
How has it been proven that it is possible to solve the \rubik{} in 22 steps when no such algorithm has been made?
\begin{itemize}
	\item Which algorithms are there now and how efficient are they with respect to the number of twists?
\end{itemize}
How can we create an application which can solve the \rubik{}?
\begin{itemize}
	\item How efficient can we make this application with respect to the number of twists?
\end{itemize}

\section{Problem Limitations}
Because the amount of different algorithms for \rubik{} solving, not every algorithm will be covered in this project.

The \rubik{} solving algorithm will be primarily for technical use, meaning that usability will not be in focus.