\chapter{E-mail Correspondence With Herbert Kociemba}
\begin{verbatim}
From:	Herbert Kociemba [kociemba@t-online.de]
Sent:	16. marts 2010 16:44
To:	Alex Bondo Andersen
Subject:	Re: Rubik's Cube study at Aalborg University

Hello,

it is quite unusual to give away private informations to some "strangers", but ok, 
here is some information.
I studied mathematics and physics (for "Lehramt an Gymnasien") at the Technische 
Universit�t Darmstadt http://www.tu-darmstadt.de/ from
1974-1979 and am teacher for mathematics and physics since then at a Gymnasium. I 
still live in Darmstadt. I am interested in Rubik's Cube since the beginning in 
1980 - the same time were personal computers came up - and was immediately 
interested to solve the cube algorithmically. 
But it was not before 1990 that the PC- power was big enough to develop the ideas 
and implementation for the two-phase-algorithm. I used an Atari St with 1 MB of 
main memory for the first implementation and already got average solutions lengths 
of about 21 moves....

If you have some other specific question, let me know.

Best regards

Herbert Kociemba
>
> Good day Herbert Kociemba,
>
> My name is Alex Bondo Andersen, I am attending Aalborg University in 
> Denmark. My university group and I are working on a paper about the 
> Rubik's Cube and are interested in using your webpage:
> http://kociemba.org/cube.htm as reference for a solving algorithm.
>
> If you are okay with us using your webpage as reference we would like 
> to know a little about you in order to verify you as a credible 
> source. So if you have the time it would be appreciated if you wrote 
> which schools you have attended, where you live, which jobs you have 
> had and why you are interested in the Rubik's Cube.
>
> In advance I would like to thank you for your time.
>
> Best Regards
>
> Group A215, Alex Bondo Andersen
>
\end{verbatim}