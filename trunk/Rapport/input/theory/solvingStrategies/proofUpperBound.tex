\section{Proof of the Upper Bound}

%INTRODUCTION

\subsection{Set Solver}
The algorithm used to prove the current lowest upper bound is known as a set solver and uses Kociemba's algorithm. The set solver is a viable method for proving the upper bound because it does not solve every single cube but a whole set of cubes at the time as the name suggests. This means that it solves approximately 19.5 billion cubes at a time.
The set solver does this by finding all the move sequences of a relabeled cube of the distance d that transforms the cube into $H$. %All the move sequences to transform the unlabeled cube to $e$ is in a database so it is fast to find the shortest move sequence, once the move sequence to $H$ has been found.

\begin{algorithm}[!h]                     
\caption{Set Solver \cite{rokicki09}}          
\label{alg:setSolver}        
\begin{algorithmic}[1]
\STATE {$f=null$}
\STATE {$d=0$}
\WHILE {true} 
		\IF {$d \leq m$}
			\FOR {$b \in S^d$}
				\IF {$r(ab) = r(e)$}
					\STATE {$f = f \cup ab$}
				\ENDIF
			\ENDFOR
		\ENDIF
		\IF {$f = H$}
			\STATE {return $d$}
		\ENDIF
	\STATE {$d = d + 1$}
	\STATE {$f = f \cup fA$}
	\IF {$f = H$}
		\STATE {return $d$}
	\ENDIF
\ENDWHILE
\end{algorithmic}
\end{algorithm}

First in the set sovler two variable is initialized. The first one $f$ which is a set that can hold all the positions of $H$ is set to $null$. The second variable is the distance $d$, which is the distance from a scrambled position $a$ to a position in $H$. This distance $d$ is set to $0$.

When these 


%First time the while loop is run the first part, $f = f \cup fA$, does nothing. The reason why nothing happens is because f is initialized to $null$. When applying moves from $A$ to $null$ nothing happens. The next if-statement that test if $f = H$ will come out false, so $d$ will not be returned.