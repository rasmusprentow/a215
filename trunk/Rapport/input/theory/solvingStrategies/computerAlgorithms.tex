\section{Computer Algorithms}
\label{sec:computerAlgorithms}
In this section two algorithms, Kociemba's Two Phase and Rokicki's Set Solver, will be presented. These algorithms requires a large amount of memory and speed in order to solve a given scrambled \rubik{} in a reasonable amount of time. Therefore we categorize these algorithms as computer algorithm because no human would be able to use them efficiently.

First of all there is a few additions to the terminology which will be used in this section. The following expressions are hereby added to the terminology:
\begin{itemize}
\item S : 18 standard twist
\item s : Specific position
\item |s| : Length from start till s
\item H : Positions obtained by A
\item A : twist u, u', u2 ,f2, r2, b2 ,l2, d, d'
\item G : All positions. 
\item r(s) : The relabeling of position s
\item S^{n} : Set best\aa{}ende af maks n antal moves i S.
\item S* : Alle kombinationer af twister
\item d(s) : Diameter, shortest |s|
\item R : set of r(G)
\item M : 48 color permutation turn and mirror

\item a (kociemba) : En position i G
\item b (kociemba) : Path from a to h S*
\item c (kociemba) : Path from h to e (abc = e) A*
\item d (kociembe) : distance for search
\item d2 (kociemba) : Table lookup of distance b from a position.

\end{itemize}
	\subsection{Kociemba's Two Phase Algorithm}
This subsection will explain how the Kociemba Two Phase Algorithm works and why it can solve any \rubik{} in 29 \twist{}s or less. Kociemba's Algorithm finds a solution to a scrambled \rubik{} using to phases.

\subsubsection{Relabeling}
The relabeling process  starts with choosing 