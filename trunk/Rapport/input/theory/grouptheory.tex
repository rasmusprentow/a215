%\documentclass{report}\begin{document}
\chapter{Group Theory}

\myTop{In this chapter we will explain group theory and how it can be used to solve the \rubik{}. When we have a good understand of the group theory we will be able to use it later for our own program.
}

\section{Permutations}
In this section we will use the same move notations as in \ref{move_notation}.
To calculate how many positions the cube can be placed in, we have to look at the general cube terminology \ref{general_notation}.
As stated in general cube terminology there is 8 corners \cpiece{} and the first corner you place can be placed in 8 positions, and after that is placed the next corner \cpiece{} can be placed in one of the 7 positions left, since we allready use 1, and so on. So that means that the corner pieces can be placed in $8*7*6*5*4*3*2*1=8!$ Now the 8 corner \cpiece{} is placed at the right position they might not have the right orientation since a corner \cpiece{} have three different colors and therefor 3 different orientations, so there is $3^8$ orientations of the 8 corner \cpiece{}. That means there is $8!*3^8$ ways the corner \cpiece{} can be placed. As stated in the terminology there is 12 edge \cpiece{}. These edge \cpiece{} (flertal?) can be positioned in 12 different positions and every edge \cpiece{} can be oriented in two different ways. So there is $2^12$ different orientations and $12!$ different positions of the edge \cpiece{}. This gives us $2^{12}*12!$ different edge permutations and a total of $3^8*2^{12}*12!*8!$



\myTail{
}
%\end{document}