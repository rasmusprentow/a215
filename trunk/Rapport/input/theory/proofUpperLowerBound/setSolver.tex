\section{Set Solver}
\label{sec:setSolver}

The set solver created by Thomas Rockicki, which was described in the previous section will now be further described.

The set solver has a special way of testing the \rubik{}s. It does not solve them to the unit position $e$, instead it finds a move sequence for a subgroup of the \rubik{} this way it can solve approximately 19.5 billion cubes at a time and not just one. The reason for this is that if you relabel an arbitrary cube, that given cube can be unlabeled to approximately 19.5 billion different cube positions. Recall that there are approximately 19.5 billion positions in the set \m{H} and all these positions are equal to $e$ when relabeled. The same logic applies to any other given position.

The algorithm used to prove the current lowest upper bound is known as a set solver and uses Kociemba's algorithm. The set solver is a viable method for proving the upper bound because it does not solve every single cube but a whole set of cubes at the time as the name suggests. This means that it solves approximately 19.5 billion cubes at a time.
The set solver does this by finding all the move sequences of a relabeled cube of the distance $d$ that transforms the cube into \m{H}. 
The algorithm set solver is described in pseudo code below in algorithm \ref{alg:setSolver}.
\begin{algorithm}[!h]                     
\caption{Set Solver \cite{rokicki09}}          
\label{alg:setSolver}        
\begin{algorithmic}[1]
\STATE {$f=\oslash$}
\STATE {$d=0$}
\WHILE {true} 
		\IF {$d \leq m$}
			\FOR {$b \in S^d$}
				\IF {$r(ab) = r(e)$}
					\STATE {$f = f \cup ab$}
				\ENDIF
			\ENDFOR
		\ENDIF
		\IF {$f = H$}
			\STATE {return $d$}
		\ENDIF
	\STATE {$d = d + 1$}
	\STATE {$f = f \cup fA$}
	\IF {$f = H$}
		\STATE {return $d$}
	\ENDIF
\ENDWHILE
\end{algorithmic}
\end{algorithm}

First in the set solver two variables are initialized. The first one \m{f} is a set that can hold all the positions of \m{H} is set to an empty set. The second variable is the distance $d$, which is the distance from a scrambled position $a$ to a position in \m{H}. This distance $d$ is initially set to $0$.

Next the \textbf{while} loop is run. It will run until $d$ is returned, which is when all positions in \m{H} has been found.

$m$ is the maximum number of \m{S} moves performed from the position $a$. When $d$ is equal to $m$ only \m{A} moves are performed.
If $d$ is lower than or equal to $m$ a for loop will be run. This loop performs all possible move sequences of the the length $d$, and adds the position $ab$ to \m{f} if it is a position in \m{H}. 
For the sake of efficiency move sequences that give the exact same position more than once are not used. If a move sequence contained F F', that part would  of the move sequence would be unnecessary. 

If \m{f} is equal to \m{H} all positions in \m{H} have been found, $d$ is returned and the algorithm has finished. 
If this is not the case, $d$ is incremented by one. 
The different \m{A} moves are performed on all the current \m{H} positions in \m{f} and the new \m{H} positions are saved in \m{f}.
If \m{f} contains all positions in \m{H}, $d$ is returned -- if not the while loop continues.



When the algorithm has finished all the different possible \m{H} positions should be saved, if the maximum distance $m$ is set sufficiently high. The theoretically highest number of twists needed to transform any scrambled cube into \m{H} is 12, but more positions are found in the for loop if $m$ is set higher. This is because the set solver both performs \m{A} moves on a cube in \m{H}, which gives more \m{H} positions. The set solver also performs moves that transforms a cube in \m{H} to a cube not in \m{H} and then back again by using \m{S} moves.