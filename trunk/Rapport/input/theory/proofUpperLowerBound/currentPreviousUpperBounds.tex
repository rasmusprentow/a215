\section{The Current and Previous Upper Bounds}
The progression of finding the current and previous upper bounds have 
%This is a slow moving field, but some events have occurred in the field the last couple of years.%% DAN DAN DAN DAN giv kilder pl0x
The reason why it is a slow process to prove the upper bound, is that there is a vast amount of different positions a \rubik{} can be in. Even with todays computer power there is simply to much data to process. This have inspired a small group of people to dedicate a lot of time to create and improve algorithms to solve arbitrary \rubik{}s.
%the effect that a small group of people dedicate a lot of time to create and improve algorithms to solve arbitrary \rubik{}s. 

\begin{comment}

The set solver created by Thomas Rockicki, which was described in the previous section will now be further described.

%The result

The set solver has a special way of testing the \rubik{}s. It does not solve them to the unit position $e$, instead it finds a move sequence for a subgroup of the \rubik{} this way it can solve approximately 19.5 billion cubes at a time and not just one. The reason for this is that if you relabel an arbitrary cube, that given cube can be unlabeled to approximately 19.5 billion different cube positions. Recall that there are approximately 19.5 billion positions in the set \m{H} and all these positions are equal to $e$ when relabeled. The same logic applies to any other given position.
\end{comment}

The progression greatly accelerated when that set solver proved the first upper bound of 25 moves. This was done on home computers from October 2007 to March 2008. They only needed to solve 6000 sets, but after this they got contacted by John Welborn from Sony Pictures Imageworks and he offered a lot of idle computers from a computer farm to help on the project. 

After this the process of lowering the bound sped up, not long after they proved the upper bound of 24 and 23. As the upper bound is lowered they need to solve more and more sets to ensure that it is the upper bound, and they needed to test almost 27000 and 180000 sets for 24 and 23. 

The current upper bound is on 22 moves and was proved in 2009 \cite{rokicki09}. To prove this they needed to compute 1,265,326 different sets. At the moment they have not proven that the upper bound can be 21, but the computer farm is currently working on it, and they expect that it is possible to lower the upper bound to 20. This means that any arbitrary \rubik{} could be solved in just 20 moves.
