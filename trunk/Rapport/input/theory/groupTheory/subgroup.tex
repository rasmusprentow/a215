\section{Subgroup}
\label{sec:subgroup}
%From section \ref{sec:permutations} it is given that there is $3^8\cdot2^{12}\cdot12!\cdot8!$ permutations, but it is not all of these permutations that are possible to do. To make it more easy there will only be looked at some moves of the \rubik{} but only the moves of the down and right faces. To better understand $G$, it can be split up in small pieces.
In this chapter it will be proved that a subset of a group also is a subgroup.
A nonempty subset $H$ of the group $(G,*)$ is called a subgroup of $G$ if $(H,*)$ is a group.

Let $(G,*)$ be a group. A nonempty subset $H$ of $G$ is a subgroup of $(G,*)$ if, for every $a, b \in H$, $a * b^{-1} \in H$.

Proof. First, suppose $H$ is a subgroup. If $b \in H$, then $b^{-1} \in H $since$ (H,*)$ is a group. So, if $a \in H$ as well, then $a * b^{-1} \in H$.

Conversely, suppose that, for every $a, b \in H$, $a * b^{-1} \in H$.

\begin {itemize}
\item First, notice that $*$ is associative since $(G,*)$ is a group.
\item Let $a \in H$. Then, $e = a * a^{-1}$, so $e \in H$.
\item Let $b \in H$. Then $b^{-1} = e * b^{-1} \in H$, so inverse exist in $H$.
\item Let $a, b \in H$. By the previous step, $b^{-1} \in H$, so $a* (b^{-1})^{-1} = a* b \in H$. Thus, $H$ is closed under $*$.
\end {itemize}

Therefore, $(H,*)$ is a group, which means that $H$ is a subgroup of $G$.
