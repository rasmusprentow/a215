\section{Definition of the Rubik's Cube group}
To understand how the \rubik{} can be described as a group, it is necessary to understand what a group is.
In group theory the group is written as $(Set, Operator)$ where the set are the elements the group consists of and the operator is the operation that is performed between the elements. When looking at the \rubik{} as a group the set will be a move or a move sequence that is applied to the \rubik{}. The operator in the \rubik{} group will be astrix ($*$) because the \rubik{}s movements are not commutative. The moves are not commutative because it matters in which order they are applied to the \rubik{}\cite[p. 157]{Rubik87}.

To prove that the \rubik{} is a group there are four points it must fulfill:

\begin {itemize}
\item When two elements are combined with an operation then the new element these create must also be a part of the group. For instance a move $M_1$ and a move $M_2$ then $M_1 * M_2$ will form a new move or move sequence which also is a part of the group.

\item $e$ is an empty move (which does not change the configuration of the \rubik{}), if the move $M * e$ is performed. This means that only the move $M$ is performed (the move $e$ could be a $360^o$ twist of a face), which means that $M*e=M$.

\item If $M$ is a move then there is a reverse move, which is called $M'$. Therefore $M*M' = e$, which means every element in the group has a reverse move.

\item The operation must be associative. A move sequence can be defined as the orientation and position it puts each \cpiece{} in. If $c$ is an oriented  cubie, $M(c)$ will be the orientation $c$ for the oriented cubicle $c$ ends in after the move is applied. For instance the move $R$ will move the $ur$ cubie to the $br$ cubicle, so therefore $R(ur)=br$. If a move sequence is applied, then the operation will look like this $B'(R(ur))=db$.\\
$*$ is associative because $(M_1 *M_2 )*M_3 = M_1 *(M_2 *M_3 )$ for any moves $M_1$, $M_2$ and $M_3$. $(M_1 *M_2 )*M_3$ and $M_1 *(M_2 *M_3 )$ does the same operation to every cubie. This is the same as saying $[(M_1 *M_2 )*M_3 ](c)=[M_1 *(M_2 *M_3 )](c)=M_3 (M_2 (M_1 (c)))$ for any cubie $c$. Therefore $*$ is associative.

\end {itemize}

%Mangler en afrunding og m�ske lidt udskrivning af den sidste linje om at * er associativ.
