\section{Definition of the Rubik's Cube Group}
\label{sec:groupDefinition}
To understand how the \rubik{} can be described as a group, it is necessary to understand what a group is.
In group theory the group is written as $(Set, Operator)$ where the set are the elements the group consists of and the operator is the operation that is performed between the elements. When looking at the \rubik{} as a group the set will be a move or a move sequence that is applied to the \rubik{}. The operator in the \rubik{} group will be astrix ($*$) because the \rubik{}s movements are not commutative. The moves are not commutative because it matters in which order they are applied to the \rubik{} \cite[p. 157]{Rubik87}.

To prove that the \rubik{} is a group there are four points it must fulfill:
\begin {enumerate}
\item When two elements are combined with an operation then the new element these create must also be a part of the group. For instance a move \m{M$_1$} and a move \m{M$_2$} then \m{M$_1$} * \m{M$_2$} will form a new move or move sequence which also is a part of the group.

\item $e$ is an empty move (which does not change the configuration of the \rubik{}), if the move \m{M}$* e$ is performed. This means that only the move \m{M} is performed (the move $e$ could be a $360^o$ twist of a face), which means that \m{M}$*e=$ \m{M}.

\item If \m{M} is a move then there is a reverse move, which is called \m{M'}. Therefore \m{M}$*$\m{M'} $= e$, which means every element in the group has a reverse move. This holds for the \rubik{} since every move has a reverse, e.g. \m{R} has \m{R'}, and a move sequence can be reversed by first inverting the order and the reverting each move in the sequence.

\item The operation must be associative meaning that the parentheses can be reordered without affecting the result. A move sequence can be defined as the orientation and position it puts each \cpiece{} in. If $c$ is a positioned \cubie{}, \m{M}($c$) will be the \cubicle{} which $c$ ends in after the move is applied. For instance the move \m{R} will move the $ur$ \cubie{} to the $br$ cubicle, so therefore \m{R}$(ur)=br$. If a move sequence is applied, then the operation will look like this \m{B'}$($\m{R}$(ur))=db$.\\
$*$ is associative because (\m{M}$_1*$\m{M}$_2)*$\m{M}$_3$ = \m{M}$_1 *$(\m{M}$_2 *$\m{M}$_3$) for any moves \m{M$_1$}, \m{M$_2$} and \m{M}$_3$. (\m{M}$_1 *$\m{M}$_2$)$*$\m{M}$_3$ and \m{M}$_1 *$(\m{M}$_2 *$\m{M}$_3$) does the same operation to every \cubie{}. This is the same as saying [(\m{M}$_1 *$\m{M}$_2 )*$\m{M}$_3$]$(c)=$ [\m{M}$_1 *$(\m{M}$_2 *$\m{M}$_3$)]$(c)=$ \m{M}$_3$(\m{M}$_2$(\m{M}$_1 (c)$)) for any \cubie{} $c$.
\end {enumerate}

As an extension to the first point some moves will form a new move rather than a move sequence, e.g. \m{R}$*$\m{R} $=$ \m{R2}. This goes for any move sequence only containing moves of the same \face{}.

In fact it is possible to define the \rubik{} only using the following set of moves:
(\m{U D F B L R}) since every other move and move sequence can be made out of these six moves.
This is called the Singmaster notation\cite[p. 7]{Joyner02}, but to make shorter move sequences two of such \twist{}s will be denoted with the letter of the move appended by \m{2}, three of such \twist{}s will be denoted with the letter of the move appended by a prime (\m{'}) as stated in section \ref{sec:moveNotation}.