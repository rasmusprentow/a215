\chapter{Terminology}

%((INDS∆T FORKLARING AF TERMINOLOGY))

% When adding a new entry, make sure to use the same syntax as the previous entries.
% Make sure to insert new entries at the right place, alphabetically
\section{General terminology}
\begin{itemize}
\item \myTermHigh{Cuber}The self reference for people who are devoted to the community of the \rubik{}. 
\end{itemize}

\section{General cube terminology}
\begin{itemize}
\label{general_notation}
\item \myTermHigh{Face}A face is an entire side of the cube. A \rubik{} has 6 faces.
\item \myTermHigh{\facet{c}}The small stickers on the cube. Each face has 9 \facet{}s.
\item \myTermHigh{Corner}Corner pieces have 3 \facet{}s and are placed at the corners. 
\item \myTermHigh{Edge}Edge pieces have 2 \facet{}s and are placed at the edges of each face. 
\item \myTermHigh{Center}Center pieces have 1 \facet{} and are placed at the center of each face and are immovable unless the cube is turned. 
\item \myTermHigh{Turn}A turn of the cube is equal to rotating the whole cube 90 degrees(=changing view angle).
\item \myTermHigh{Twist}A twist is a rotation of a face.%Twisting means the rotation of a side of the cube. Often denoted as R' for exampel. This yields twist the right side counter-clockwise.
\end{itemize}

\section{Movement notation terminology}
A cube consists of 6 faces and the notations of these are the following.
\begin{itemize}
\label{move_notation}
\item \myTermHigh{Front face} F -- This face faces the cuber.
\item \myTermHigh{Left face} L -- This face faces the left hand side of the cuber.
\item \myTermHigh{Right face} R -- This face faces the right hand side of the cuber.
\item \myTermHigh{Up face} U -- This face faces up.
\item \myTermHigh{Down face} D -- This face faces down.
\item \myTermHigh{Back face} B -- This face faces away from the cuber.
\end{itemize} 

A face can be twisted in two directions -- clockwise and counterclockwise. When twisting a face the direction is determined as if you were facing the face.
A twist in the clockwise direction has the same name as the face. i.e. a clockwise turn of the right face is notated "R" and pronounced "right". A counterclockwise twist of the right face is notated "R'" and pronounced "right prime". This gaoes for all the faces.

A turn of the cube can be done in six directions. Clockwise and counterclockwise around each of the three axes.


%MANGLER FORKLARING OG NOTATION AF "`TURNS"'


\section{Group theory related terminology}
\huge{Under construction}
\begin{itemize}
\item $S$ : 18 standard twist.
\item $s$ : Specific position.
\item $|s|$ : Length from $e$ till $s$.
\item $H$ : Positions obtained by $A$.
\item $A$ : twist U, U', U2 ,F2, R2, B2 ,L2, D, D' and D2.
\item $G$ : All positions. 
\item $r(s)$ : The relabeling of position $s$.
\item $S^{n}$ : Set consisting of max $n$ moves in $S$.
\item $S*$ : Every combination of twists in $S$.
\item $d(s)$ : Diameter, shortest |s|.
\item $R$ : set of $r(G)$.
\item $M$ : 48 color permutation turn and mirror.
\item $e$: The solved state of the \rubik{}(unit cube).

\item $a$ (kociemba) : En position i $G$
\item $b$ (kociemba) : Path from $a$ to $H$ $S*$
\item $c$ (kociemba) : Path from $H$ to $e$ ($abc = e$) $A*$
\item $d$ (kociembe) : distance for search %what?
\item $d2$ (kociemba) : Table lookup of distance $b$ from a position.

\end{itemize}




