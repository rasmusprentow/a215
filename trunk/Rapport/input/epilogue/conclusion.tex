\chapter{Conclusion}
In our project we started to get some knowledge about the \rubik{} and the algorithms used to solve the \rubik{}. We used different communities about the \rubik{} to get the basic knowledge about the \rubik{}. This lead us to our problemstatement:
%\input{problemballs}

To answer our problemstatement we investigated which algorithms there was used to solve the \rubik{}, and we picked two algorithms that where very different to eachother. One algorithm was "difficult" and impossible for humans to remember and use to solve the \rubik{} and the other was easy for humans to remember. The "difficult" algorithm is called Kociemba's optimal solver and the easy one for humans is called Beginner's algorithm. We compared these algoritms to see which one was the most efficient twistwise.

To get a better understanding of how the moves of the \rubik{} interacts with eachother, we had to get an understanding of group theory and graph theory. This information is the foundation of our Terminology.

To compare the two algorithms we implemented them into an application, to illustrate which one was the must efficient twistwise. 

Therefor we came to the conclusion that Kociemba's optimal solver was more efficient twist wise.