\chapter{Conclusion}
In our project we started to get some knowledge about the \rubik{} and the algorithms used to solve it. We used different communities about the \rubik{} to get the basic knowledge about the \rubik{}. This led us to our problem statement:
%\input{problemballs}

\vspace{2mm}
\begin{centering}
\hspace{2mm}
\framebox[\textwidth - 6mm]{
\parbox{\textwidth - 12mm}{
\vspace{2mm}
\textit{What are the current upper and lower bounds of the \rubik{} and how have they been proven? \newline\newline 
Which solving algorithms exist and how efficient are they? \newline\newline
How can we create an application which can solve the \rubik{}?
\vspace{2mm}
}
}}
\linebreak

To answer our problem statement we investigate which algorithms there is used to solve the \rubik{}, and picked two algorithms which are very different from each other. 
One algorithm is "difficult" and impossible for humans to remember and use to solve the \rubik{}. 
The other is easy for humans to remember. 
The "difficult" algorithm is called Kociemba's optimal solver and the easy one for humans is called Beginner's algorithm. 
The two algorithms has been compared to see which one is the most efficient twist-wise. 
In addition we investigate the current upper and lower bound and how they have been proven.

To get a better understanding of how the moves of the \rubik{} interacts with each other, we had to get an understanding of group theory and graph theory. This information is the foundation of our Terminology.

To compare the two algorithms we implemented them into an application, to illustrate which one was the must efficient twist-wise. 

Therefore we came to the conclusion that Kociemba's optimal solver is more efficient twist-wise and that the current lower bound is 20 and the current upper bound is 22 which was proven with the set solver made by Thomas Rokicki with inspiration from Kociemba's optimal solver.

%\chapter{Perspectivering}
%If we had more time we would have optimized our application so the Kociemba's optimal solver can give a solution of a scrambled cube within a forseeable future. In addition we could also look at more algorithms to implement in our application and thereby maybe get a better solution.