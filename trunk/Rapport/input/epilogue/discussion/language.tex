\section{Another Programming Language}
Java (see section \ref{cha:choicesPriorToImplementation}) is an interpreted language \cite{kilde}, and therefore there is some advantages and disadvantages when using it.
One of the disadvantages with programs written in an interpreted language is that it executes slower compared to direct machine code executions \cite{kilde2}.
Java is first compiled to virtual machine code (Java Bytecode) and then it is interpreted to machine code by a runtime application.

The advantage with using an interpreted programming language like Java is that it is very safe \cite{kilde3}.
It can be executed on all platforms as long as the platform has a runtime application installed. 
E.g. if the program was written in C++ and compiled on one platform it would not be able to execute on a different platform \cite{kilde4}.

If we focused our application to only work on one platform, e.g. Windows, we could write the application in C++.
It has a faster execution time due to the fact that it compiles directly to machine code and this will reduce the solving time of our implemented algorithms \cite{kilde5}. 
