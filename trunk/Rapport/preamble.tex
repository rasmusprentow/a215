\documentclass{book}

\usepackage[english]{babel}
\usepackage{natbib}
\usepackage{dk-bib}
\usepackage[a4paper, bookmarks=true, bookmarksopen=true, bookmarksnumbered=true, colorlinks=true, breaklinks=true, backref=true, pdfborder={0 0 0 0}]{hyperref}
\usepackage[latin1]{inputenc}
\usepackage{amsmath,amsfonts,amssymb}
\usepackage{fancyhdr}
\usepackage[dvips]{graphicx}
%\usepackage[Lenny]{fncychap}
%\usepackage[Sonny]{fncychap}
%\usepackage[Glenn]{fncychap}
%\usepackage[Conny]{fncychap}
%\usepackage[Rejne]{fncychap}
\usepackage{lastpage}
%\usepackage{xparse}

%% BEGIN BLAME RASMUS
\usepackage{xifthen}% provides \isempty test

%% END BLAME RAMSUS


%\graphicspath{/input/pics/}
\newcommand{\picturepath}[1]{input/pics/}
%% Tilf¯j nedenst\aa{}ende linje i din sandkasse.
%% \renewcommand{\picturepath}[1]{../Rapport/input/pics/}


%\pagestyle{fancy}

\setcitestyle{numbers}
\bibliographystyle{plainnat}

\setcounter{tocdepth}{1}

\linespread{1}

%\setlength{\marginparwidth}{10pt}
%\setlength{\textwidth}{400pt}
%\setlength{\textheight}{620pt}
%\setlength{\voffset}{0pt}
%\setlength{\hoffset}{0pt}
%\setlength{\topmargin}{0pt}
%\setlength{\headsep}{10pt}
%\setlength{\oddsidemargin}{50pt}
%\setlength{\evensidemargin}{10pt}



% c = first letter capital
% cap = all capital
% i = italic
% b = bold
% ci = first letter cap and all italic.
 \newcommand{\theWord}{some}
\newcommand{\caseControl}[4]{%
  \ifthenelse{\equal{#3}{c}}% First letter capital
    {\renewcommand{\theWord}{\MakeUppercase{#1}#2}}% if #1 true
    {}% if #1 false
     \ifthenelse{\equal{#3}{ci}}% first letter cap and italic
    {\renewcommand{\theWord}{\textit{\MakeUppercase{#1}#2}}}% if #1 true
    {}% if #1 false
      \ifthenelse{\equal{#3}{ic}}%% first letter cap and italic
    {\renewcommand{\theWord}{\textit{\MakeUppercase{#1}#2}}}% if #1 true
    {}% if #1 false
      \ifthenelse{\equal{#3}{i}}%% italic
    {\renewcommand{\theWord}{\textit{#1#2}}}% if #1 true
    {}% if #1 false
     \ifthenelse{\equal{#3}{cap}}% all cap
    {\renewcommand{\theWord}{\MakeUppercase{#1#2}}}% if #1 true
    {}% if #1 false
       \ifthenelse{\isempty{#3}}% %if nothing is stated 
    {\renewcommand{\theWord}{#1#2}}% if #1 true
    {}% if #1 false 
    \ifthenelse{\equal{\theWord}{some}}% % Double check actually
    {\renewcommand{\theWord}{#1#2}}% if #1 true
    {}% if #1 false 
    %%%%%% Standard definitions of words %%%%%%
        \ifthenelse{\equal{#4}{i}}% 
    {\textit{\theWord}}% if #1 true  % Print it italic
    {}% if #1 false 
	    \ifthenelse{\equal{#4}{b}}% Print it in bold
    {\textbf{\theWord}}% if #1 true
    {}% if #1 false 
        \ifthenelse{\equal{#4}{u}}% print it underlinde (not working yet)
    {{\theWord}}% if #1 true
    {}% if #1 false 
       \ifthenelse{\isempty{#4}}%  If there is nothing stated here just print the shit. 
    {\theWord}% if #1 true
    {}% if #1 false    
    \renewcommand{\theWord}{some}%
  }
%% WORD LIST
%% This is where we define words. 
%% The last parameter should be blank as standard, but you can add i,b,u. 
\newcommand{\john}[1]{\caseControl{j}{ohn}{#1}{}}
\newcommand{\michael}[1]{\caseControl{M}{ikael}{#1}{}}
\newcommand{\rubik}[1]{\caseControl{R}{ubik's cube}{#1}{}}
\newcommand{\facet}[1]{\caseControl{f}{acet}{#1}{}}
\newcommand{\facelet}[1]{\caseControl{f}{acelett}{#1}{}}
\newcommand{\cube}[1]{\caseControl{c}{ube}{#1}{}}
\newcommand{\cuber}[1]{\caseControl{c}{uber}{#1}{}}
\newcommand{\face}[1]{\caseControl{f}{ace}{#1}{}}
\newcommand{\cpiece}[1]{\caseControl{p}{iece}{#1}{}}
\newcommand{\twist}[1]{\caseControl{t}{wist}{#1}{}}
\newcommand{\turn}[1]{\caseControl{t}{urn}{#1}{}}
\newcommand{\rotate}[1]{\caseControl{r}{otate}{#1}{}}
\newcommand{\erno}[1]{\caseControl{E}{rn\"{o} Rubik}{#1}{}}
\newcommand{\mpuzzle}[1]{\caseControl{M}{agic Puzzle}{#1}{}}
\newcommand{\msquare}[1]{\caseControl{M}{agic Square}{#1}{}}
\newcommand{\mcube}[1]{\caseControl{M}{agic Cube}{#1}{}}
%COMMANDS:
%Used to determine the highlight of the first word in the terminology
\newcommand{\myTermHigh}[1]{\textbf{#1}: }

%tops 'n' tails
\newcommand{\myTop}[1]{\textit{#1} \  \\  \hrule \  \\}%
\newcommand{\myTail}[1]{  \ \\ \hrule \ \\ \textit{#1}}%
\newcommand{\startTop}{\textbf{Change this command til myTop with the top as argument} \\}%
\newcommand{\stopTop}{ \textbf{Change this command til myTop with the top as argument} \\}%
\newcommand{\startTail}{ \textbf{Change this command til myTail with the tail as argument} \\}%
\newcommand{\stopTail}{ \textbf{Change this command til myTail with the tail as argument} \\ }%


%Use this for caption text
\newcommand{\myCaption}[1]{\textit{\footnotesize #1} }

%Bruges til skillekolonner og r\ae{}kker . Definerer tykkelsen. 
\newcommand{\vrules}{{\vrule width 0.6pt}}
\newcommand{\hrules}{{\hrule height 1.2pt}}

%Commands for getting eg. ``st'' lifted in 1st
\newcommand{\superscript}[1]{\ensuremath{^{\textrm{#1}}}}
\newcommand{\subscript}[1]{\ensuremath{_{\textrm{#1}}}}
\newcommand{\ths}[0]{\superscript{th}}
\newcommand{\st}[0]{\superscript{st}}
\newcommand{\nd}[0]{\superscript{nd}}
\newcommand{\rd}[0]{\superscript{rd}}