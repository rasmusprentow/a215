\documentclass{book}

\usepackage[english]{babel}
\usepackage{natbib}
\usepackage{dk-bib}
\usepackage[a4paper, bookmarks=true, bookmarksopen=true, bookmarksnumbered=true, colorlinks=true, breaklinks=true, backref=true, pdfborder={0 0 0 0}]{hyperref}
\usepackage[latin1]{inputenc}
\usepackage{amsmath,amsfonts,amssymb}
\usepackage{fancyhdr}
\usepackage[dvips]{graphicx}
%\usepackage[Lenny]{fncychap}
%\usepackage[Sonny]{fncychap}
%\usepackage[Glenn]{fncychap}
%\usepackage[Conny]{fncychap}
%\usepackage[Rejne]{fncychap}
\usepackage{lastpage}
%\usepackage{xparse}

%% BEGIN BLAME RASMUS
\usepackage{xifthen}% provides \isempty test

%% END BLAME RAMSUS


%\graphicspath{/input/pics/}
\newcommand{\picturepath}[1]{input/pics/}
%% Tilf¯j nedenst\aa{}ende linje i din sandkasse.
%% \renewcommand{\picturepath}[1]{../Rapport/input/pics/}


%\pagestyle{fancy}

\setcitestyle{numbers}
\bibliographystyle{plainnat}

\setcounter{tocdepth}{1}

\linespread{1}

%\setlength{\marginparwidth}{10pt}
%\setlength{\textwidth}{400pt}
%\setlength{\textheight}{620pt}
%\setlength{\voffset}{0pt}
%\setlength{\hoffset}{0pt}
%\setlength{\topmargin}{0pt}
%\setlength{\headsep}{10pt}
%\setlength{\oddsidemargin}{50pt}
%\setlength{\evensidemargin}{10pt}




% c = first letter capital
% cap = all capital
% i = italic
% b = bold
% ci = first letter cap and all italic.
 \newcommand{\theWord}{some}
\newcommand{\caseControl}[4]{%
  \ifthenelse{\equal{#3}{c}}%
    {\renewcommand{\theWord}{\MakeUppercase{#1}#2}}% if #1 true
    {}% if #1 false
     \ifthenelse{\equal{#3}{ci}}%
    {\renewcommand{\theWord}{\textit{\MakeUppercase{#1}#2}}}% if #1 true
    {}% if #1 false
      \ifthenelse{\equal{#3}{ic}}%
    {\renewcommand{\theWord}{\textit{\MakeUppercase{#1}#2}}}% if #1 true
    {}% if #1 false
      \ifthenelse{\equal{#3}{i}}%
    {\renewcommand{\theWord}{\textit{#1#2}}}% if #1 true
    {}% if #1 false
     \ifthenelse{\equal{#3}{cap}}%
    {\renewcommand{\theWord}{\MakeUppercase{#1#2}}}% if #1 true
    {}% if #1 false
       \ifthenelse{\isempty{#3}}%
    {\renewcommand{\theWord}{#1#2}}% if #1 true
    {}% if #1 false 
    \ifthenelse{\equal{\theWord}{some}}%
    {\renewcommand{\theWord}{#1#2}}% if #1 true
    {}% if #1 false 
        \ifthenelse{\equal{#4}{i}}%
    {\textit{\theWord}}% if #1 true
    {}% if #1 false 
	    \ifthenelse{\equal{#4}{b}}%
    {\textbf{\theWord}}% if #1 true
    {}% if #1 false 
        \ifthenelse{\equal{#4}{u}}%
    {{\theWord}}% if #1 true
    {}% if #1 false 
       \ifthenelse{\isempty{#4}}%
    {\theWord}% if #1 true
    {}% if #1 false    
    \renewcommand{\theWord}{some}%
  }



\input{functions/myDate}
\input{functions/wordList}

%COMMANDS:
%Used to determine the highlight of the first word in the terminology
\newcommand{\myTermHigh}[1]{\textbf{#1}: }

%tops 'n' tails
\newcommand{\myTop}[1]{\textit{#1} \  \\  \hrule \  \\}%
\newcommand{\myTail}[1]{  \ \\ \hrule \ \\ \textit{#1}}%
\newcommand{\startTop}{\textbf{Change this command to myTop with the top as argument} \\}%
\newcommand{\stopTop}{\textbf{Change this command to myTop with the top as argument} \\}%
\newcommand{\startTail}{\textbf{Change this command to myTail with the tail as argument} \\}%
\newcommand{\stopTail}{\textbf{Change this command to myTail with the tail as argument} \\ }%


%Use this for caption text
\newcommand{\myCaption}[1]{\textit{\footnotesize #1} }

%Bruges til skillekolonner og r\ae{}kker . Definerer tykkelsen. 
\newcommand{\vrules}{{\vrule width 0.6pt}}
\newcommand{\hrules}{{\hrule height 1.2pt}}

%Commands for getting eg. ``st'' lifted in 1st
\newcommand{\superscript}[1]{\ensuremath{^{\textrm{#1}}}}
\newcommand{\subscript}[1]{\ensuremath{_{\textrm{#1}}}}
\newcommand{\ths}[0]{\superscript{th}}
\newcommand{\st}[0]{\superscript{st}}
\newcommand{\nd}[0]{\superscript{nd}}
\newcommand{\rd}[0]{\superscript{rd}}