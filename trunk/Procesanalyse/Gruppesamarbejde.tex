\section{Gruppesamarbejde}
Inden vi begyndte selve vores arbejdsproces diskuterede vi vores forventninger til hinanden. Vi havde generelt et hjt ambitionsniveau til vores rapport. 
Vi diskuterede hvad der kunne virke motiverende og demotiverende. Vi kom frem til at hvis der var nogen, der ikke arbejdede i et et tidsrum, hvor der var skulle arbejdes, kunne det virke demotiverende.
Derfor besluttede vi at vi i gruppen at alle skulle arbejde og holde pause p samme tid. Dette udmntede sig i at man kunne nvne at man havde brug for en pause, hvorefter alle holdt pause. 
P den mde blev tid til arbejde og tid til pause direkte adskildt.


Vi startede projektprocessen med at udarbejde en samarbejdsaftale for gruppen. Mange af punkterne er blevet overholdt, men andre af punkterne er blevet glemt. 
Grunden til at nogle punkter ikke er blevet overholdt, er at arbejdsprocessen har get godt uden nogle enkelte af de vrktjer samarbejdsaftalen tilbd.


Vi har ikke holdt organiserede mder i gruppen. Vi har kun holdt mder nr det virkede ndvendigt i gruppen. 
Nr et gruppemedlem flte, at der var noget der krvende en diskussion, blev gruppen bedt om at lukke deres skrme ned og deltage i diskussionen. 
Diskussionen har vret fri; der har alts ikke vret en mdeleder eller ordstyrer.
Frekvensen af disse mder har vret svingende, da deres ndvendighed ikke har vret s stor.


Iflge vores samarbejdsaftale var der mdepligt for gruppens medlemmmer alle hverdage fra kl. 08.15 til 15.00. 
Dette blev hovedsageligt overholdt, der var dog nogle gruppemedlemmer, der mdte for sent om morgenen. 
Dette frte til en straffeordning, hvor det medlem, der mdte for sent skulle give kage til gruppen. 


Al gruppearbejde har foreget i grupperummet. Dette har medfrt at problemer hurtigt kunne imdeges. 
Hovedsageligt deltog alle gruppemedlemmer i disse diskussioner.
Hvis et gruppemedlem har flt sig umotiveret er dette blevet diskuteret og rsagen til den manglende motivation blevet kortlagt, s arbejdet kunne fortstte.
Hvis hele gruppen har flt sig umotiveret blev arbejdet sat p stand-by, og der blev holdt en pause, hvor vi eksempelvis har spillet et computerspil sammen, eller s tog vi fri og fortsatte arbejdet den flgende dag.
Manglende motivation har dog ikke vret et serist problem, og arbejdsprocessen har hovedsageligt forlbet efter planen.


Vi har i gruppen fordelt de forskellige emner, der skulle skrives om efter interesse; Strrelsen af emnerne er blevet vurderet og et passende antal gruppemedlemer blev sat p det givne emne.

Arbejdet i gruppen har hovedsageligt foreget i mindre grupper p omkring to eller tre gruppemedlemmer. Dette har generelt fungeret godt. 
Hvis der var problemer i en af de mindre grupper var det altid muligt at sprge de andre gruppemedlemmer til rds, da alle var til stede i grupperummet.
Hvis et emne har vret strre end frst antaget, blev flere gruppemedlemmer flyttet til emnet, hvis det var muligt.

Hver dag blev tidspunktet for frokostpausen aftalt. I gruppen var der en madordning, hvor gruppemedlemmerne betalte et bestemt belb hver uge, som der blev handlet ind for til frokost. 

Gruppearbejdet har gennemgende get godt. Vi har ikke get i st i lngere tid, da vi altid har kunne sprge hinanden til rds. Vores arbejdsprocess har vret produktiv, hvilket er en indikator for at den mde, vi har arbejdet p, har fungeret godt.

Selvom arbejdet i mindre grupper har fungeret godt, kunne man eksperimentere med at tildele arbejde hjemme. Dette kunne foreg som et suplement til arbejde i gruppen, hvor nogle dage arbejdet foregik i grupperummet og andre dage bestod af arbejde hjemme.
Dette kunne muligvis effektivisere arbejde og undg eventuelle gnidninger imellem gruppemedlemmer.

Til P3 ville det vre en god ide at aftale sociale arrangementer p forhnd, i modstning til spontant, som det hovedsageligt er foreget i lbet af P2. 
Problemet med udelukkende spontane arrangementer er at, mange gruppemedlemmer ofte havde aftaler, og arrangementerne derfor ofte ikke blev til noget.