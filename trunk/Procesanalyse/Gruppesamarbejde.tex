\section{Gruppesamarbejde}
\subsection{Beskrivelse}
Inden vi begyndte selve vores arbejdsproces diskuterede vi vores forventninger til hinanden. Vi havde generelt et h\o{}jt ambitionsniveau til vores rapport. 
Vi diskuterede hvad der kunne virke motiverende og demotiverende. Vi kom frem til at hvis der var nogen, der ikke arbejdede i et et tidsrum, hvor der var skulle arbejdes, kunne det virke demotiverende.
Derfor besluttede vi at vi i gruppen at alle skulle arbejde og holde pause p\aa{} samme tid. Dette udm\o{}ntede sig i at man kunne n\ae{}vne at man havde brug for en pause, hvorefter alle holdt pause. 
P\aa{} den m\aa{}de blev tid til arbejde og tid til pause direkte adskildt.


Vi startede projektprocessen med at udarbejde en samarbejdsaftale for gruppen. Mange af punkterne er blevet overholdt, men andre af punkterne er blevet glemt. 
Grunden til at nogle punkter ikke er blevet overholdt, er at arbejdsprocessen har g\aa{}et godt uden nogle enkelte af de v\ae{}rkt\o{}jer samarbejdsaftalen tilb\o{}d.


Vi har ikke holdt organiserede m\o{}der i gruppen. Vi har kun holdt m\o{}der n\aa{}r det virkede n\o{}dvendigt i gruppen. 
N\aa{}r et gruppemedlem f\o{}lte, at der var noget der kr\ae{}vende en diskussion, blev gruppen bedt om at lukke deres sk\ae{}rme ned og deltage i diskussionen. 
Diskussionen har v\ae{}ret fri; der har alts\aa{} ikke v\ae{}ret en m\o{}deleder eller ordstyrer.
Frekvensen af disse m\o{}der har v\ae{}ret svingende, da deres n\o{}dvendighed ikke har v\ae{}ret s\aa{} stor.


If\o{}lge vores samarbejdsaftale var der m\o{}depligt for gruppens medlemmmer alle hverdage fra kl. 08.15 til 15.00. 
Dette blev hovedsageligt overholdt, der var dog nogle gruppemedlemmer, der m\o{}dte for sent om morgenen. 
Dette f\o{}rte til en straffeordning, hvor det medlem, der m\o{}dte for sent skulle give kage til gruppen. 


Al gruppearbejde har foreg\aa{}et i grupperummet. Dette har medf\o{}rt at problemer hurtigt kunne im\o{}deg\aa{}es. 
Hovedsageligt deltog alle gruppemedlemmer i disse diskussioner.
Hvis et gruppemedlem har f\o{}lt sig umotiveret er dette blevet diskuteret og \aa{}rsagen til den manglende motivation blevet kortlagt, s\aa{} arbejdet kunne forts\ae{}tte.
Hvis hele gruppen har f\o{}lt sig umotiveret blev arbejdet sat p\aa{} stand-by, og der blev holdt en pause, hvor vi eksempelvis har spillet et computerspil sammen, eller s\aa{} tog vi fri og fortsatte arbejdet den f\o{}lgende dag.
Manglende motivation har dog ikke v\ae{}ret et seri\o{}st problem, og arbejdsprocessen har hovedsageligt forl\o{}bet efter planen.


Vi har i gruppen fordelt de forskellige emner, der skulle skrives om efter interesse; St\o{}rrelsen af emnerne er blevet vurderet og et passende antal gruppemedlemer blev sat p\aa{} det givne emne.

Arbejdet i gruppen har hovedsageligt foreg\aa{}et i mindre grupper p\aa{} omkring to eller tre gruppemedlemmer. Dette har generelt fungeret godt. 
Hvis der var problemer i en af de mindre grupper var det altid muligt at sp\o{}rge de andre gruppemedlemmer til r\aa{}ds, da alle var til stede i grupperummet.
Hvis et emne har v\ae{}ret st\o{}rre end f\o{}rst antaget, blev flere gruppemedlemmer flyttet til emnet, hvis det var muligt.

Hver dag blev tidspunktet for frokostpausen aftalt. I gruppen var der en madordning, hvor gruppemedlemmerne betalte et bestemt bel\o{}b hver uge, som der blev handlet ind for til frokost. 

Gruppearbejdet har gennemg\aa{}ende g\aa{}et godt. Vi har ikke g\aa{}et i st\aa{} i l\ae{}ngere tid, da vi altid har kunne sp\o{}rge hinanden til r\aa{}ds. Vores arbejdsprocess har v\ae{}ret produktiv, hvilket er en indikator for at den m\aa{}de, vi har arbejdet p\aa{}, har fungeret godt.

Selvom arbejdet i mindre grupper har fungeret godt, kunne man eksperimentere med at tildele arbejde hjemme. Dette kunne foreg\aa{} som et suplement til arbejde i gruppen, hvor nogle dage arbejdet foregik i grupperummet og andre dage bestod af arbejde hjemme.
Dette kunne muligvis effektivisere arbejde og undg\aa{} eventuelle gnidninger imellem gruppemedlemmer.

Til P3 ville det v\ae{}re en god ide at aftale sociale arrangementer p\aa{} forh\aa{}nd, i mods\ae{}tning til spontant, som det hovedsageligt er foreg\aa{}et i l\o{}bet af P2. 
Problemet med udelukkende spontane arrangementer er at, mange gruppemedlemmer ofte havde aftaler, og arrangementerne derfor ofte ikke blev til noget.