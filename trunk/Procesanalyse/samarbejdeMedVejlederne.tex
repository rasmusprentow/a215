\section{Samarbejde med vejlederne}
\label{H8R}

Vores samarbejde med vores hovedvejleder har v\ae{}ret rigtig godt. 
N\aa{}r vi har sendt noget materiale til ham, har vi f\aa{}et god og hurtig respons. 
Vi har n\ae{}sten haft m\o{}der hver uge med vores hovedvejleder, og der har altid v\ae{}ret god og
konstruktiv kritik. 
Vi har f\o{}lt at vores hovedvejleder har fulgt os igennem dette projekt og vist stor interesse for emnet, samt kommet med masser af forslag til forbedringer.

Vores samarbejde med vores bivejleder har ogs\aa{} fungeret godt, og hun har ogs\aa{} givet os god feedback. 
I starten havde vi mange m\o{}der med hende, men den sidste halvanden m\aa{}ned havde vi kun et m\o{}de, da vi kun skrev teori og implementation i denne periode.

N\aa{}r vi skulle sende arbejdsblade, som vi ville have respons p\aa{}, sendte vi dem til vejlederne 48 timer f\o{}r m\o{}det. Dette gav dem tid til at l\ae{}se dem igennem.

Et eksempel p\aa{} et referat fra et af vores vejlederm\o{}der se sektion \ref{ref}

\subsection{Refleksion}
Vores samarbejde med vejlederne har generelt fungeret rigtig godt, og de har bidraget meget til projektets udformning. 
Nogle gange har vi v\ae{}ret frustrerede over nogle af de ting, de har sagt. 
Fx hvis det har v\ae{}ret st\o{}rre \ae{}ndringer i strukturen af rapporten. Andre gange har vi ikke forst\aa{}et pr\ae{}cist, hvad de
mente. 
Efter langt de fleste vejlederm\o{}der har vi dog siddet med f\o{}lelse af at have et godt overblik over rapporten og hvad der skulle ske efterf\o{}lgende. 
I slutfasen burde vi have holdt nogle flere m\o{}der med vores bivejleder, da hun havde en hel del \ae{}ndringer til det sidste m\o{}de. Vi burde have holdt det sidste m\o{} med vores bivejleder noget f\o{}r, og ikke tage for god vare, at hun sagde vores kontekstuelle del var fint halvanden m\aa{}ned inden afleverings datoen.


Afsnittet ``Samarbejde med vejledere'' i vores samarbejdsaftale, har vi ikke brugt, da vores samarbejde med vores vejledere har fungeret godt og de har altid kommet med konstruktiv kritik og vejledning. 
Vi fik ikke altid sendt vores arbejdsblade til vores vejledere i ordentlig tid, fordi vi var lidt pressede en gang imellem.

\subsection{Gode og d\aa{}rlige erfaringer}
Gode erfaringer, som vi vil bruge i P3:
\begin{itemize}
\item Vi pr\o{}vede altid at sende arbejdsbladene 48 timer inden m\o{}derne.
\end{itemize}
Ting, som vi vil g\o{}re anderledes i P3:
\begin{itemize}
\item Vi vil ikke tage det for god vare halvanden m\aa{}ned inden, n\aa{}r bivejleder siger at den kontekstuelle del er f\ae{}rdig, og vil f\o{}lge op p\aa{} det l\o{}bende ved at sende igen til bivejlederen.
\end{itemize}