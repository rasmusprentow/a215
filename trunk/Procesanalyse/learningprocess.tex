\section{L\ae{}ringsm\aa{}l}
Da vi startede i vores P2 gruppe blev vi hurtige enige om nogle grundregler, og herefter fandt vi nogle l\ae{}ringsm\aa{}l som vi mente var realistiske at n\aa{} i l\o{}bet af projektforl\o{}bet.

Vi blev hurtigt enige om nogle forskellige l\ae{}ringsm\aa{}l da vi begyndte med projektet, og de var som f\o{}lger:

\begin{itemize}
\item L\ae{}re at l\o{}se Rubik's terningen.
\item L\ae{}re at programmere i Java.
\item L\ae{}re  og forst\aa{} det bagvedliggende matematik i Rubik's terningen. 
\item L\ae{}re gruppe- og grafteori.
\end{itemize}

Vi har i gruppen arbejdet mod vores l\ae{}ringsm\aa{}l.
Vi har ikke brugt det direkte som drivkraft for projektet, men vi har brugt dem til at definere retningen p\aa{} vores projekt.
Flere af vores l\ae{}ringsm\aa{}l har v\ae{}ret essentielle punkter for indholdet af vores rapport, og derfor har vi kunnet arbejde med vores l\ae{}ringsm\aa{}l samtidig med at vi har skrevet p\aa{} vores projekt.

Vi har holdt et lille m\o{}de, hvor vi har snakket lidt om hvordan det er g\aa{}et med vores l\ae{}ringsm\aa{}l. Da det er forskelligt hvordan folk l\ae{}rer, og der er forskel p\aa{} hvor meget folk g\aa{}r op i forskellige emner, kan det v\ae{}re sv\ae{}rt at lave en general bed\o{}mmelse. Vi har dog snakket om hvordan det er g\aa{}et med de enkelte l\ae{}ringsm\aa{}l, og generalt i gruppen har de fleste opn\aa{}et de enkelte m\aa{}l.

Vi har arbejdet l\o{}st med vores l\ae{}ringsm\aa{}l, det har bevirket at alle har st\aa{}et for egen l\ae{}ring. Det har virket for os, selvom vi ikke har gjort s\aa{} meget for det.
Vi tror at det er fordi vi i gruppen er meget ansvarsbeviste og derfor har alle taget del i arbejds- og l\ae{}ringsprocessen.

Til P3 kunne man tage flere noter og l\o{}bende evalurere p\aa{} disse.

