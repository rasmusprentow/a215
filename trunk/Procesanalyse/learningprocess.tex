Hvordan har i arbejdet med læringsmål i gruppen?
Hvordan har i fulgt op på de læringsmål der blev defineret i gruppen? Har i nået målene?
Hvordan lærer du bedst, via individuelt arbejde – gruppediskussion – forelæsning etc.?
Hvordan har I brugt resultaterne af jeres individuelle læringsstilstest?
Hvilken læringsstrategi er efter jeres mening bedst i forbindelse med kurser? Hvorfor?
Hvordan hjælper I hinanden med at løse opgaver i kurserne?
Hvad gør I hvis I ikke forstår en forelæser eller det der står i bogen?
Hvilken læringsstrategi er efter jeres mening bedst i forbindelse med projektarbejdet? Hvorfor
Hvordan stimulerer og fremmer jeres vejledere jeres læreprocesser?


\section{Beskrivelse}
Da vi startede i vores P2 gruppe blev vi hurtige enige om nogen grundregler, og herefter fandt vi nogen l\ae{}ringsm\aa{}l som vi mente var realistiske at nå i løbet af projektforløbet.

Vi blev hurtigt enige om nogle forskellige læringsmål da vi begyndte med projektet, og de var som følger:

\begin{itemize}
\item L\ae{}r at l\o{}se Rubiks Cuben.
\item L\ae{}re at programmere i Java.
\item L\ae{}re  og forst\aa{} det bagved liggende matematik i Rubiks Terningen. 
\item L\ae{}e gruppe- og grafteori.
\end{itemize}

Vi har i gruppen arbejdet mod vores læringsmål, vi har ikke brugt det direkte som drivkræft for projektet, men vi har brugt dem til at definere retningen på vores projekt.
Flere af vores læringsmål har været essentielle punkter for indholdet af vores rapport, og derfor har vi kunnet arbejde med vores læringsmål samtidig med at vi har skrevet på vores projekt.

Vi har holdt et lille møde hvor vi har snakket lidt om hvordan det er gået med vores læringsmål. Da det er forskelligt hvordan folk lærer, og der er forskel på hvor meget folk går op i forskellige emner, så kan det være svært at lave en general bedømmelse. Vi har dog snakket lidt om hvordan det er gået med de enkelte læringsmål,

I skal beskrive jeres arbejdsprocesser i P1 så detaljeret som muligt indenfor de fire områder:
    •   Projektplanlægning
    •   Gruppesamarbejde
    •   Samarbejde med vejledere
    •   Læreprocesser
Beskrivelsen kan eksempelvis give et overblik over udviklingen i jeres arbejdsprocesser, ændringer
i processer undervejs, læringsmål, forventningsafklaringer i gruppen og med vejleder, aktiviteter til
opfølgning af målsætninger osv.


\section{Vurdering}

Når I er færdige med at beskrive hvad I gjorde, skal I vurdere hvordan det efter jeres mening gik.
Eksempelvis kan i komme ind på, hvordan i fulgte op på gruppens beslutninger og målsætninger og
i hvilket omfang jeres arbejdsprocesser indenfor de fire områder fungerede efter hensigten.


\section{Analyse}

Dernæst skal I analysere jeres arbejdsprocesser og få klarlagt hvorfor noget gik godt mens andet gik
dårligt. Med andre ord: Hvilke faktorer har indvirket på arbejdsprocesserne, og hvordan håndterede
i de udfordringer der opstod undervejs?


\section{Syntese}

Hvis jeres vurdering og analyse skal bidrage til at forbedre jeres evne til at håndtere det
problemorienterede og projektorganiserede gruppearbejde, skal I til slut konkretisere jeres
erfaringer i nogle ’Gode råd’ til jer selv og jeres medstuderende. En god måde at formulere sådanne
gode råd på er som en ’start-stop-fortsæt’-liste, dvs. en liste med følgende tre sektioner:
    •   Dette vil vi begynde at gøre i P2, som vi ikke gjorde i P1
    •   Dette vil vi ikke gøre i P2, som vi gjorde i P1
    •   Dette vil vi fortsætte med at gøre (gerne anderledes og bedre) i P2, som vi også gjorde i P1
Det er en god idé at tage ét af de fire områder ad gangen og gøre det færdigt. Husk hvad angår
strukturen at skelne klart mellem beskrivelse; vurdering og analyse og husk, at de ’Gode råd’ skal
være konkrete og operationelle, så de fører til reelle forbedringer i P2.
