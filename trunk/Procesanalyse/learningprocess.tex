\section{L\ae{}ringsm\aa{}l}

\begin{comment}
Hvordan har i arbejdet med l\ae{}ringsm\aa{}l i gruppen?
Hvordan har i fulgt op p\aa{} de l\ae{}ringsm\aa{}l der blev defineret i gruppen? Har i n\aa{}et m\aa{}lene?
Hvordan l\ae{}rer du bedst, via individuelt arbejde -- gruppediskussion -- forel\ae{}sning etc.?
Hvordan har I brugt resultaterne af jeres individuelle l\ae{}ringsstilstest?
Hvilken l\ae{}ringsstrategi er efter jeres mening bedst i forbindelse med kurser? Hvorfor?
Hvordan hj\ae{}lper I hinanden med at l\o{}se opgaver i kurserne?
Hvad g\o{}r I hvis I ikke forst\aa{}r en forel\ae{}ser eller det der st\aa{}r i bogen?
Hvilken l\ae{}ringsstrategi er efter jeres mening bedst i forbindelse med projektarbejdet? Hvorfor
Hvordan stimulerer og fremmer jeres vejledere jeres l\ae{}reprocesser?
\end{comment}

\subsection{Beskrivelse}
Da vi startede i vores P2 gruppe blev vi hurtige enige om nogen grundregler, og herefter fandt vi nogen l\ae{}ringsm\aa{}l som vi mente var realistiske at n\aa{} i l\o{}bet af projektforl\o{}bet.

Vi blev hurtigt enige om nogle forskellige l\ae{}ringsm\aa{}l da vi begyndte med projektet, og de var som f\o{}lger:

\begin{itemize}
\item L\ae{}r at l\o{}se Rubiks Cuben.
\item L\ae{}re at programmere i Java.
\item L\ae{}re  og forst\aa{} det bagved liggende matematik i Rubiks Terningen. 
\item L\ae{}e gruppe- og grafteori.
\end{itemize}

Vi har i gruppen arbejdet mod vores l\ae{}ringsm\aa{}l, vi har ikke brugt det direkte som drivkr\ae{}ft for projektet, men vi har brugt dem til at definere retningen p\aa{} vores projekt.
Flere af vores l\ae{}ringsm\aa{}l har v\ae{}ret essentielle punkter for indholdet af vores rapport, og derfor har vi kunnet arbejde med vores l\ae{}ringsm\aa{}l samtidig med at vi har skrevet p\aa{} vores projekt.

Vi har holdt et lille m\o{}de hvor vi har snakket lidt om hvordan det er g\aa{}et med vores l\ae{}ringsm\aa{}l. Da det er forskelligt hvordan folk l\ae{}rer, og der er forskel p\aa{} hvor meget folk g\aa{}r op i forskellige emner, s\aa{} kan det v\ae{}re sv\ae{}rt at lave en general bed\o{}mmelse. Vi har dog snakket om hvordan det er g\aa{}et med de enkelte l\ae{}ringsm\aa{}l, og generalt i gruppen så har de fleste opnået de enkelte mål.

\begin{comment}
I skal beskrive jeres arbejdsprocesser i P1 s\aa{} detaljeret som muligt indenfor de fire omr\aa{}der:
    •   Projektplanl\ae{}gning
    •   Gruppesamarbejde
    •   Samarbejde med vejledere
    •   L\ae{}reprocesser
Beskrivelsen kan eksempelvis give et overblik over udviklingen i jeres arbejdsprocesser, \ae{}ndringer
i processer undervejs, l\ae{}ringsm\aa{}l, forventningsafklaringer i gruppen og med vejleder, aktiviteter til
opf\o{}lgning af m\aa{}ls\ae{}tninger osv.


\subsection{Vurdering}

N\aa{}r I er f\ae{}rdige med at beskrive hvad I gjorde, skal I vurdere hvordan det efter jeres mening gik.
Eksempelvis kan i komme ind p\aa{}, hvordan i fulgte op p\aa{} gruppens beslutninger og m\aa{}ls\ae{}tninger og
i hvilket omfang jeres arbejdsprocesser indenfor de fire omr\aa{}der fungerede efter hensigten.


\subsection{Analyse}

Dern\ae{}st skal I analysere jeres arbejdsprocesser og f\aa{} klarlagt hvorfor noget gik godt mens andet gik
d\aa{}rligt. Med andre ord: Hvilke faktorer har indvirket p\aa{} arbejdsprocesserne, og hvordan h\aa{}ndterede
i de udfordringer der opstod undervejs?


\subsection{Syntese}

Hvis jeres vurdering og analyse skal bidrage til at forbedre jeres evne til at h\aa{}ndtere det
problemorienterede og projektorganiserede gruppearbejde, skal I til slut konkretisere jeres
erfaringer i nogle ’Gode r\aa{}d’ til jer selv og jeres medstuderende. En god m\aa{}de at formulere s\aa{}danne
gode r\aa{}d p\aa{} er som en ’start-stop-forts\ae{}t’-liste, dvs. en liste med f\o{}lgende tre sektioner:
    •   Dette vil vi begynde at g\o{}re i P2, som vi ikke gjorde i P1
    •   Dette vil vi ikke g\o{}re i P2, som vi gjorde i P1
    •   Dette vil vi forts\ae{}tte med at g\o{}re (gerne anderledes og bedre) i P2, som vi ogs\aa{} gjorde i P1
Det er en god idé at tage ét af de fire omr\aa{}der ad gangen og g\o{}re det f\ae{}rdigt. Husk hvad ang\aa{}r
strukturen at skelne klart mellem beskrivelse; vurdering og analyse og husk, at de ’Gode r\aa{}d’ skal
v\ae{}re konkrete og operationelle, s\aa{} de f\o{}rer til reelle forbedringer i P2.
\end{comment}