\documentclass{report}
\begin{document}
\chapter{Problem Analysis}
Since 1977, when the Rubiks cube was initially released for sale, the cube has frustrated, inspired and entertained many people. This 3x3x3 cube has so many possible settings that the solution can not just be guessed out of sheer luck and since 1982 people has been competing each other in solving the cube fastest or by the least number of twist. Because of these competitions, it has been interesting for the competitors to find algorithms for solving the cube in the least number of twists. The development of these algorithms is an ongoing process which has given the latest theory in 2008, that states that an algorithm which can solve the cube in 22 twists, no matter which setting the cube starts in, is possible to create. No such algorithm has been created so far \cite{}.

Now that computers are getting faster and faster, it could be interesting to study the implementation of the solving algorithms in a computer program. The efficiency of these algorithms with respect to the time of calculation and the number of twists is an interesting focus point.
\section{Computer vs. Cube}
Is it possible to create a program, that can solve the Rubiks Cube?
\begin{itemize}
	\item Is it viable to brute force a cube?
	\item How can the cube be described mathematically and how can this be used for solving the cube?
	\item Can the Rubiks Cube communities help finding a solving strategy for the cube?
\end{itemize}

\section{22 Steps}
How come that it has been proven that it is possible to solve the Rubiks Cube in 22 steps and no such algorithm has been made?
\begin{itemize}
	\item Which algorithms are there now and how efficient are they?
	\item If a 22 step algorithm was found, how would it affect the Rubiks Cube Community?
	\item Is it possible to create an application that can solve the Rubiks Cube?
	\begin{itemize}
		\item How efficient can we make this application?
	\end{itemize}
\end{itemize}

\section{The Best Algorithm}
Can we find the best algorithm for cube solving, with respect to the numbers of twists?
\begin{itemize}
	\item Has the best algorithm been found?
	\item Can it be implemented in an application?
\end{itemize}
\end{document}