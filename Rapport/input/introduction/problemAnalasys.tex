\section{Problem Analysis}
The group concerned  with the twist-wise efficiency of the \rubik{} attempts to prove the upper and lower bounds of the \cube{}. 
An important algorithm in this group is Kociemba's optimal solver which is the twist-wise  most efficient solver \cite[rokicki09]. 
In the speedsolving group, the time-wise efficiency is a primary concern and beginner's algorithm is a fundament for solving the \cube{} in this group. 

It would be interesting to test these two algorithms both in time-wise and twist-wise efficiency, so a comparison between the two algorithms can be made. 
This is interesting because we could not find any previous comparisons of the two algorithms. 
%it has never been done before.

Another interesting part of the twist-wise concerned group is finding god's algorithm.
Since god's algorithm is unique for every position the length of the solution is more interesting, or namely the longest length. 
This length is defined where the upper and lower bound meet. 

Hereby it is interesting to examine how the upper and lower bounds has been proven and how the process of proving the bounds has evolved. 
