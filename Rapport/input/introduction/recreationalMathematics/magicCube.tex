\subsection{Magic Cube}
\label{sub:mcube}

%\begin{figure}[h]	\centering		\includegraphics[scale=0.8]{input/pics/presentMagicCube2}	\caption{\myCaption{This is a magic cube split up into 3 magic squares.}}	\end{figure}

Both a \msquare{} and a \mcube{} have a magic constant, which is the sum of each row, column, and pillar.
However this is where the similarity ends. 

It has been shown how to calculate the magic constant in a \msquare{}.
In a \mcube{} there is not a big difference in the formula to calculate the magic constant.
\begin{equation}
	M(n)=\frac{n \cdot (n^3+1)}{2}
\end{equation}
As shown in the formula the only difference is the power of $n$ that is changed from 2 to 3.
%See appendix \ref{sec:proofOfMagicConstant} for an explanation.

To create a  \mcube{}, there are some parts that need to be explained.
These parts can be seen on figure \ref{fig:cubeparts}.

\begin{figure}[htb]
	\centering
		\includegraphics[scale=0.5]{input/pics/cubeparts.pdf}
	\caption{\myCaption{This is a Magic Cube where the colors show all of the parts.}}
	\label{fig:cubeparts}
\end{figure}

Because of all these different parts there are a lot of different ways to define the \mcube{}s.
The simplest of them is called a simple \mcube{}. The only requirements to make such a cube is the following:
\begin{itemize}
	\item All 9 rows, columns, and pillars must be equal to the magic constant.
	\item All 4 triagonals must also be equal to the magic constant.
\end{itemize}



The \rubik{} has quite a few similarities with the \mcube{}. If all the integers in the \mcube{} was replaced with the small cubes of the \rubik{} called \cpiece{}s, a \rubik{} would emerge.
There are three differences. 
The first is that the \mcube{} consists of numbers whereas the \rubik{} has colors, which are different on each \face{}.
The other difference is that the \mcube{} has a number in the center where the \rubik{} center is invisible and not of importance. 
The last difference is that when permuting the \rubik{}, it is necessary to move several \cpiece{}s as opposed to the \mcube{} where integers can be changed individually. See figure \ref{fig:rubiksCube}.

This \mcube{} should not be confused with the later presented \mcube{} made by \erno{}.

\begin{figure}[hb]
	\centering
		\includegraphics[scale=0.30]{input/pics/rubiksCube}
	\caption{\myCaption{A Rubik's Cube.}}
	\label{fig:rubiksCube}
\end{figure}