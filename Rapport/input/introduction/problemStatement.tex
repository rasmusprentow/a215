\chapter{Problem Definition}
%\myTop{The following problem statement and problem limitation is based on the short introduction from the previous chapter.Since the problem of the \rubik{} is not a society oriented no problem analysis is given.}
\emptyTop{}
\section{Problem Analysis}
The group concerned  with the twist-wise efficiency of the \rubik{} attempts to prove the upper and lower bounds of the \cube{}. 
An important algorithm in this group is Kociemba's optimal solver which is the twist-wise  most efficient solver \cite[rokicki09]. 
In the speedsolving group the time-wise efficiency is a primary concern and beginner's algorithm is a foundation for solving the \cube{} in this group. 

It would be interesting to test these two algorithms both in time-wise and twist-wise efficiency, so a comparison between the two algorithms can be made. 
This is interesting because we were unable to find any previous comparisons of the two algorithms. 
%it has never been done before.

Another interesting part of the twist-wise concerned group is finding \textit{God's algorithm}.
Since \textit{God's algorithm} is unique for every position the length of the solution also very interesting, or namely the longest length. 
This length is where the upper and lower bound meet. 

Hereby it is interesting to examine how the upper and lower bounds have been proven and how the process of proving them.

\section{Problem Statement}
The following problem statement has been defined:

%\vspace{2mm}
%\begin{centering}
%\hspace{2mm}
%\framebox[\textwidth - 6mm]{
%\parbox{\textwidth - 12mm}{
%\vspace{2mm}
%\textit{What are the current upper and lower bounds of the \rubik{} and how have they been proven? \newline\newline 
%Which solving algorithms exist and how efficient are they? \newline\newline
%How can we create an application which can solve the \rubik{}?
%\vspace{2mm}
%}
%}}

\vspace{2mm}
\begin{centering}
\hspace{2mm}
\framebox[\textwidth - 6mm]{
\parbox{\textwidth - 12mm}{
\vspace{2mm}
\textit{What are the current upper and lower bounds of the \rubik{} and how have they been proven? \newline\newline 
Which solving algorithms exist and how efficient are they? \newline\newline
How can we create an application which can solve the \rubik{}?
\vspace{2mm}
}
}}

\section{Problem Limitations}
\label{sec:problemLimitations}
We will only look at the improvements of the upper bound since 1981, when Thistlewaite published his proof of the upper bound of 52 \cite{knowledgerush2}.% Only the proof of the latest proved upper bound will be fully described. 

The efficiencies we wish to test are the \twist{}-wise efficiency -- the fewer \twist{}s, the more efficient -- and the time-wise efficiency, which is the amount of time it takes the algorithm to find a solution.
Both of these will be based on a computer solving the \rubik{} since this will give the most reliable results.
%The \rubik{} application will primarily be for technical use, meaning that usability will not be in focus.

%Our application must be able to solve a \rubik{} using only operations which are possible on an actual \rubik{}.
%It must also be able to generate a scrambled \rubik{}.
%The application has to be able to present the \rubik{} in a graphical way, but this will not be a primary concern and is considered irrelevant to our problem statement. 

% the details of this will not be covered and it is not considered as an important part of the project.
%Only the part of the theory subjects which relates to our project will be presented. Any full description of a theory subject will be found in our referenced sources.