\chapter{Discussion}
\myTop{Something something of something discussion}





\section{Another Language}

\section{Comparing the Implemented Algorithms}
The two algorithms implemented in our application differs a lot from each other when looking at efficiency. 
Our implementation of Kociemba's optimal solver is twist-wise efficient but time-wise inefficient while this is opposite, to some extend, for the beginner's algorithm. 

Our implementation of Kociemba's optimal solver is the twist-wise optimal solver; it uses the least amount of \twist{}s to solve any position.
The time-wise efficiency for Kociemba's optimal solver is not impressive. It is rather inefficient. On a 2,5 GHz Duo Core, it uses in a worst-case scenario approximately 28 million years to solve a \cube{} 18 \twist{}s\footnote{Most positions can be solved in 18 \twist{}s \cite{kociemba09}} away from the solved state. 22 \twist{}s away from the solved state it will use $9.4\cdot{}10^{11}$ years. See section \ref{app:kociembaTime} for detailed calculations.   

Our implementation of the beginner's algorithm is not designed for twist-wise or time-wise efficiency. It uses less than a second to solve 1000 cubes. 
Twist-wise it is way less efficient since it uses an average of 151 moves to solve any cube (see section \ref{sec:beginnersStat}); where a human speed cuber uses around 60 \twist{}s in average \cite{larsPetrus97} \cite{cubeFreak}.


\section{Improving the Algorithms}

\subsection{Kociemba's Algorithm}
Write about the following improvements:

When the cube is solved to H it is possible to look at it from other angels. Secondary as primary and tertiary as primary.
Both the move into H and in H could be saved in a lookup table to increase the speed.
Cloning the cube, to make the program multithreaded.
Running the program on a x64 bit processer has a lot of influence on the program. 

\subsection{Beginner's Algorithm}
%%%BLA BLA BLA read beginner's first etc.

When describing the improvement of the beginner's algorithm, only improvements are described step by step. 
A twist sequence which leads to the next step may be the most efficient for doing just that, but can potentially cause the entire twist sequence from the scrambled state to the solved state to be longer. 
i.e. a long partial twist sequence may lead to a shorter overall twist sequence. In order to utilize this a lot of analysis is necessary, which is nearly impossible for a person to perform and would take a very large amount of programming to implement.


The beginner's algorithm is as mentioned before a very twist-wise inefficient solving algorithm. 
There are however quite a few areas in which the algorithm can be improved.
First of all choosing the right face to be the face of the first layer will decrease the number of twists needed to complete the first step.




