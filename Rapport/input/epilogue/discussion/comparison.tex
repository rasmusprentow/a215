\section{Comparing the Implemented Algorithms}
The two algorithms implemented in our application differs a lot from each other when looking at efficiency. 
Our implementation of Kociemba's optimal solver is twist-wise efficient but time-wise inefficient while this is opposite, to some extend, for the beginner's algorithm. 

Our implementation of Kociemba's optimal solver is a twist-wise optimal solver, i.e. it uses the least amount of \twist{}s to solve any position (see section \ref{sec:kociemba}).
The time-wise efficiency for Kociemba's optimal solver is not impressive.
It is rather inefficient.
On a 2,5 GHz Quad Core, it would use approximately 17 million years to solve a \cube{} 18 \twist{}s\footnote{Most positions can be solved in 18 \twist{}s \cite{kociemba09}} away from the solved state.
%22 \twist{}s away from the solved state it will use $9.4\cdot{}10^{11}$ years.
See section \ref{app:kociembaTime} for detailed calculations.   

Our implementation of the beginner's algorithm is not designed for twist-wise or time-wise efficiency. It uses less than a second to solve 1000 cubes (see section \ref{sec:beginnersStat}). 
Twist-wise it is way less efficient since it uses an average of 151 moves to solve any cube (see section \ref{sec:beginnersStat}).
%; where a human speed cuber uses around 60 \twist{}s in average \cite{larsPetrus97} \cite{cubeFreak}.

Time-wise beginner's algorithm is clearly better than Kociemba's optimal solver.
Specifically it is $\frac{5.5383 \cdot 10^{17} \text{ ms}}{0.478923077 \text{ ms}} \approx 1.2 \cdot 10^{18}$ times faster on average.
However this is based on that the shortest solution found by Kociemba's optimal solver is by not going into \m{H} and using \m{A} \twist{}s.
On an actual test the \rubik{} will almost certainly have an optimal solution which goes into \m{H} and then starts to use \m{A} \twist{}s and thus lowering the search time substantially.
Considering that depth 18 in \m{H} is finished in approximately 256 years against the 17 million years outside \m{H}.

With respect to the number of \twist{}s Kociemba's optimal solver always gives the shortest \twist{} sequence.
Since it will simply brute force its way to a solution with a breadth first search.%Evt. nævn breadth før
Beginner's algorithm on the other hand seem never to be optimal since the tests shows a minimum of 56 moves, which is above the upper bound thus the solution is not optimal.

From this we can conclude that beginner's algorithm is a faster algorithm than Kociemba's optimal solver.
This was partially expected since beginner's algorithm will use a ``recipe'' to solve the \rubik{} where Kociemba's optimal solver tests every possible \twist{} sequence until it gets into \m{H}, which in most cases will require a lot of \twist{}s compared to the maximum of 241 which beginner's algorithm uses to solve a randomly scrambled \rubik{}.
Further we can conclude that the beginner's algorithm is unable to solve a \rubik{} with an optimal solution.
This is quite obvious since beginner's algorithm is divided into steps and it is very unlikely that these steps will go in a ``straight line'' towards the solved state.