\chapter{Conclusion}
Before we started the writing process we gathered information about what we found to be interesting aspects of the \rubik{}. 
These aspects included the different solving algorithms and some general mathematic theories that could be applied to the \rubik{}.
After gathering sufficient information we formulated a problem statement. Our problem statement is as follows:

\vspace{2mm}
\begin{centering}
\hspace{2mm}
\framebox[\textwidth - 6mm]{
\parbox{\textwidth - 12mm}{
\vspace{2mm}
\textit{What are the current upper and lower bounds of the \rubik{} and how have they been proven? \newline\newline 
Which solving algorithms exist and how efficient are they? \newline\newline
How can we create an application which can solve the \rubik{}?
\vspace{2mm}
}
}}
\end{centering}
\linebreak

The current upper and lower bounds are respectively 22 and 20. 
This means that it is possible to solve a \rubik{} in any position in no more than 22 moves, and that the \rubik{} position that requires the most twists to solve, needs 20 twists.
The upper bound of 22 has been found by dividing the positions of the \rubik{} into cosets. A set solver was able to test if an entire set of positions needs 22 twists or less to solve the \rubik{}. The result showed that no \rubik{} position needs more than 22 twists to solve.

A specific \rubik{} position, known as superflip, has a solution, which is a sequence of 20 twists.
The lower bound was proven by testing all sequences of twists with a length shorter than 20, and finding that none of these solved this position.

The efficiency of a solving algorithm has to be tested in a controlled environment with as little human interference as possible. 
That is why an application, where a user can scramble but only the application performs the solving algorithms, is a good choice when running a test.
After we implemented the algorithms, we tested their efficiencies separately.
Our implementation of the beginner's algorithm was run 100,000 times, and the average length of the twist sequence that solved the \rubik{} was found to be 151 as well as the time spent on solving them which were XXXXXX. See section \ref{sec:beginnersStat}. 
Our implementation of Kociemba's optimal solver was run over a period of 48 hours. By this we estimated the time it would take to find an optimal solution of a \rubik{}, which were XXXX years. See section \ref{app:kociembaTime}.


\chapter{Perspective}
In this chapter we will look at some ways of improving the two implemented algorithms if we had more time. 

\section{General Improvements}
If we had more time our GUI could be improved by making the cube in the application 3D. Furthermore could we have made our program multitreading, so that the solving time of our algorithms would be improved.


\section{Improving the Algorithms}
The two algorithms can be improved in several ways. How this can be done will be described in the following.
In order to understand this section it is necessary to have read the general description of the beginner's algorithm and Kociemba's optimal solver (see chapter \ref{cha:solvingAlgorithms}) and the descriptions of our implementations (see chapter \ref{chap:beginnerImplement} and \ref{chap:kociembaImplement}).



\subsection{Beginner's Algorithm}
%%%BLA BLA BLA read beginner's first etc.

When describing the improvement of the beginner's algorithm, only improvements are described step by step. 
A twist sequence which leads to the next step may be the most efficient for doing just that, but can potentially cause the entire twist sequence from the scrambled state to the solved state to be longer. 
i.e. a long partial twist sequence may lead to a shorter overall twist sequence. In order to utilize this a lot of analysis is necessary, which is nearly impossible for a person to perform and would take a very large amount of programming to implement.


The beginner's algorithm is as mentioned before a very twist-wise inefficient solving algorithm. 
There are however quite a few areas in which the algorithm can be improved.
First of all choosing the right face to be the face of the first layer will decrease the number of twists needed to complete the first step, since a different part of the cross is assembled on the different faces.

In the second part, where the corners needs to be positioned in the first layer, there are two ways to improve the algorithm.


\subsection{Kociemba's Algorithm}
Write about the following improvements:

When the cube is solved to H it is possible to look at it from other angels. Secondary as primary and tertiary as primary.
Both the move into H and in H could be saved in a lookup table to increase the speed.
Cloning the cube, to make the program multithreaded.
Running the program on a x64 bit processer has a lot of influence on the program. 

\begin{comment}
There are many algorithms, which can solve the \rubik{}. 
There are generally two types of algorithms; algorithms for human solving and algorithms for computer solving. 
A requirement for the human algorithms is that one should be able to remember it, which limits the effectiveness. 
The computer algorithms are generally based upon systematic trial and error, which is why they are not effective to use as a human solver.
The human algorithms are time-wise efficient both for a human and a computer solver, but are not twist-wise efficient.
The computer algorithms require a lot of time, but give a twist-wise efficient result.
\end{comment}