\chapter{Implementing Kociemba's Optimal Solver}
\label{chap:kociembaImplement}
\myTop{The actual implementation of Kociemba's optimal solver will be covered in this chapter. The reader will be presented with general description of how we have chosen to implement Kociemba's optimal solver along with a detailed explanation of key points of the source code.}
The basics of Kociemba's optimal solver have been covered in section \ref{sec:kociemba}. The algorithm which we are implementing is differing from the original at one point:
We are interested in the shortest move sequence to solve a scrambled cube, where as the original Kociemba optimal solver only tries to find the shortest length, not the actual sequence of moves \cite{rokicki09}.

Because of this we have chosen not to have a lookup table since it would have to contain 20 milliard move sequences, which is impossible on todays computers considering that the lookup table only containing the length would have a size of 4 GB \cite{cubeExplorer} and a table containing the move sequences would have to be much larger.

A move can be defined using 4 bits, since there are 10 $A$ moves and $2^4=16$ which is enough.
To calculate the size of a lookup table containing the move sequence it is just a mater of multiplying the size of one move with the number of moves a sequence, e.g. a move sequence of the length 14 would have the size: $4 \cdot 14 = 56$ bits. This size is again multiplied with the number of positions which needs this length for solving.
The size of a lookup table containing every move sequences to solve inside $H$ would be: $987$ GB, which is 

	\section{The Solving Method}
\label{sec:kSolve}
The method named \vr{solve} (see code snippet \ref{src:kociemba} for a key point) acts like a main routine for the solving mechanism of Kociemba's optimal solver. By calling this method kociemba's optimal solver will be applied to the \rubik{}. 

The solver's approach to solving the \rubik{} is trying every move sequence not tried so far. If the cube is inside \m{H} it will start using only \m{A} moves. Every time a shorter move sequence is found it is saved. 

\begin{lstlisting}[style=sourceCode, caption=\myCaption{Key point in the solve method of kociemba's optimal solver}, label=src:kociemba]
try {
	c = solveFromH(l - d);
	if (d + c.length < l) {
		curTime = System.currentTimeMillis();
		l = d + c.length;
		result = new MoveButtons[l];
		output.addTextln("The solutions of the length " + l + ". The solution is:");
		int j = 0;
		for ( ; j < d; j++) {
			result[j] = b[j];
			output.addText(b[j] + " ");
		}
		for (int k = 0 ; k < c.length; k++,j++) {
			result[j] = c[k];
			output.addText(c[k] + " ");
		}
		output.addTextln("");
		output.addTextln("Time spend: " + ((curTime - startTime)/1000) + " seconds");
	}
} catch (InvalidCubeException e) {}
\end{lstlisting}

It start with initializing five variables; 
\begin{itemize}
\item The array \vr{result} which will contain the final move sequence.
\item The integer \vr{d} which determines the search depth. 
\item The integer \vr{l} which determines the length of the current solution. 
\item The array \vr{b} which will contain the move sequence until the \rubik{} enters \m{H}.
\item The array \vr{c} which will be the move sequence after the \rubik{} has entered \m{H}.
\end{itemize}

The \vr{solve} method starts by calling the \vr{solveFromH}, which solves the cube, when the cube is in \m{H}. If the cube is not in \m{H}  the exception \vr{InvalidCubeException} will be thrown. When this exception is caught it forces the \vr{solve} method to skip to the next part of the algorithm.  
If the new move sequence (\vr{b+c}) is shorter than the currently shortest move sequence (\vr{result}) the new one will be saved in \vr{result} and \vr{l} is set to be the length of the new move sequnce.
When the cube has been solved from \m{H} it test if the length of the new solution is better than the old solution this will be saved as the new solution. 
\begin{comment}
 method to test if the cube is in \m{H}. The method \vr{solveFromH} is called with a parameter, which determines how deep it will search.

If the \rubik{} is in \m{H} then \vr{solveFromH} will start solving it with \m{A} moves. If the \rubik{} is not inside \m{H} the \vr{solveFromH} will throw an exception called \vr{InvalidCubeException}. 

When the InvalidCubeException is caught by it cause 

\end{comment}
After the code snippet \ref{src:kociemba} the method will permute the \rubik{} back with the inverse of the last move. When it has been permuted the method \vr{increaseWithSNotEndingWithA} will be called. See section \ref{sec:increaseWithSNotEndingWithA}.

When the method \vr{increaseWithSNotEndingWithA} has been through every move sequence of the length \vr{d} it will throw an exception of the type  
\vr{UnableToIncreaseMoveSequenceException}. The \vr{solve} method will catch this exception and increments \vr{d}. When \vr{d} is incremented to \vr{l} the method will return the array \vr{result} which contains the shortest move sequence to the given \rubik{}.  
\begin{comment}
Thereafter it tests if \vr{d + c.length} > \vr{l}. 
If that is true, \vr{l} will be set equal to the sum of \vr{d + c.length} and the \vr{result} array will be initialized with the size \vr{l}. 
In the \vr{result} array the move sequence \vr{b} and the move sequence \vr{c} is added and the console will print the current \vr{result} and the time it took in seconds.

After the method has tested if \vr{d + c.length < l} the method will permute the \rubik{} back with the inverse of the last move. When it has been permuted the method \vr{increaseWithSNotEndingWithA} will be called. See section \ref{sec:increaseWithSNotEndingWithA}.

When the method \vr{increaseWithSNotEndingWithA} throws an exception of the type \vr{UnableToIncreaseMoveSequenceException} the method increments \vr{d}. 
When \vr{d} is incremented to \vr{l} the method will return the \vr{result} array.
\end{comment}
	\section{The Move Incrementing Methods}
\label{sec:increaseWithSNotEndingWithA}
The method named \vr{increaseWithSNotEndingWithA} (see code snippet \ref{src:kociemba2}) increments a move sequence.
When the method is called it takes two parameters, a move sequence of the type MoveButtons and an integer.
As the method initializes it creates two variables of the type integer, \vr{length} and \vr{i}.
The variable \vr{length} is set equal to the length of the move sequence, so every time the length of the move sequence is needed the variable \vr{length} will be used.
The variable \vr{i} is used to keep track of where in the move sequence a move is changed and it is set equal to the input integer parameter. 

\begin{lstlisting}[style=sourceCode, caption=\myCaption{Key point in the incrementing method of kociemba's optimal solver}, label=src:kociemba2]
if (i == length - 1) {
	try {
		do {
			moveSequence[i] = (MoveButtons)S.toArray()[moveSequence[i].ordinal() + 1];
		} while(A.contains(moveSequence[i]));
		try {
			if(isSameFace(moveSequence[i], moveSequence[i-1])) {
				increaseWithSNotEndingWithA(moveSequence, i);
			} else {
				Cube.permute(cube, moveSequence[i]);
			}
		} catch (ArrayIndexOutOfBoundsException e2) {
			Cube.permute(cube, moveSequence[i]);
		}
		return;
	} catch (ArrayIndexOutOfBoundsException e) {
		moveSequence[i] = F;
		i--;
		try {
			Cube.permute(cube, moveSequence[i].invert());
			moveSequence[i].invert();
		} catch (ArrayIndexOutOfBoundsException e4) {
			throw new UnableToIncreaseMoveSequenceException();
		}
			increaseWithSNotEndingWithA(moveSequence, i);
		return;
	}
}
\end{lstlisting}

The last move performed on the cube can never be an \m{A} move since \m{A} is a closed group, see section \ref{sec:subgroup}.
At first the method tests if it is working with the last move, if this is the case then a \textbf{do-while} loop will be executed.
Inside this loop the parameter \vr{moveSequence} will be updated with a move from \m{S}, but the condition in the loop is that it must run while the enumset \m{A} contains the new updated move.
When a move not in \m{A} is found, then the loop is ended.

After the loop is over a new method is called, \vr{isSameFace} this method tests if the new move is the same as the move before this.
If they are the same (see section \ref{sec:groupDefinition}), then the method will call itself with the last move as a parameter else it will permute the cube with the current move sequence.
When the \rubik{} has been permuted it will \vr{return} and the method will end.

The method increments the move, and at some point it will reach the end of the enumset.
When this occurs an \vr{ArrayIndexOutOfBoundsException} will be thrown, this exception is caught in a \textbf{try-catch}.
If the exception is caught then the last place in the move sequence array will be set equal to the move \m{F}.
After this the counter variable \vr{i} will be decremented.
Then the method will call itself, and after it has increased the second last move it will \textit{return} and end the method.

\begin{lstlisting}[style=sourceCode, caption=\myCaption{The definition of the enumsets S, A, and notA.}, label=src:enumset]
private EnumSet<MoveButtons> S = EnumSet.of(U, UP ,U2, D, DP, D2, F, FP, F2, B, BP, B2, L, LP, L2, R, RP, R2);
private EnumSet<MoveButtons> A = EnumSet.of(U, UP ,U2, D, DP, D2, F2, B2, L2, R2);
private EnumSet<MoveButtons> notA = EnumSet.of(F, FP, B, BP, L, LP, R, RP);
\end{lstlisting}

All this code are executed if and only if the first condition is true.
If it is false, then the method will start increasing the current move with a textit{S} move.
This is done the same way as above, except there is no do-while loop around it since we are interested in all the moves in textit{S}.

When the method tests if two moves are the same, just as it did before if the first condition was true.
It will now test if all the moves from the current position \textit{i} to the \textit{length}.
This is done with a for-loop that count every step, and the same code snippet that tested if they were on the same face.

Again we have been able to reuse the same code snippet.
When the method gets to the end of the enumset S, it will set the move to the first move in S.
This is done with the same code as before, except this time it sets the move to \textit{U}.

\myTail{}