\chapter{Digitalizing the Cube}
\myTop{In order to implement solving algorithms on a \rubik{} the cube itself must first be digitalized and a visual interface must be created in order to see the result of an algorithm. In this chapter that process is described.}
A \rubik{} is a rather simple three-dimensional structure, but implementing this structure into a computer system and getting it depicted on a two-dimensional screen is not a simple task.
The \rubik{} is build up by 26 moving \cpiece{}s held together by each other.  This type of structure is not straight forward to implement into a computer, so a way to handle these \cpiece{}s is needed.

%If the \rubik{} is depicted in the computer in a two-dimensional space, its depiction will be far from the original \rubik{} structure.
A simple way to handle the \rubik{} in the program is a two-dimensional depiction and  just simply move the \facelet{}s around.
The analogue to this on a real \rubik{} would be to take the colored stickers off and move them to their new position rather than actually move the \cubie{}s when twisting the \rubik{}.
This approach would be far from reality and would make the implementation of solving algorithms, such as Kociemba's optimal solver and the beginner's algorithm more complicated, since they need to know the position of the \cpiece{}s, and not just the \facelet{}s in order to determine the next step.

A more object-oriented approach than to move the sticker would give a more useful structure for solving the problem.
If we divided the \rubik{} into its sub structures it would consist of six faces each with nine shared \cubie{}s.
%Dividing the \rubik{} into its sub structures. 
%The cube consist of the 6 faces each with 9 shared \cpiece{}. 
A face also consists of nine \cubicle{}s which act as placeholders for the \cpiece{}s. 
There are three types of \cpiece{}s and \cubicle{}s: corners, edges, and centers. 
The center \cpiece{}s will never move and therefore there is no reason to define a \cubicle{} and a \cpiece{} for those. Instead the center piece of a face is granted a colored \facelet{}, which defines the color of the face in the completed state of the \rubik{}.
\begin{figure}[h]
	\centering
		\includegraphics[scale=0.5]{input/pics/twistOfUpFace}
	\caption{\myCaption{A twist of a face will not only inflict the twisted face but also its adjacent faces. }}
	\label{fig:twistOfUpFace}
\end{figure}
This structure allows for the same \cubicle{} to be in several faces -- two faces for edges and three for corners. 
In this structure moving a \face{} will move the \cpiece{}s from one \cubicle{} to another. See figure \ref{fig:twistOfUpFace}. By performing a \twist{} the \face{}s adjacent to the twisted \face{} are also affected. 
For example when \twist{}ing the up face of a \rubik{} the \cubie{}s adjacent to the up face in R, L, B, and F faces will be permuted.

However the problem of this structure is that it requires variables to determine how the \cubie{}s are oriented. 
How this is done is described in the section \ref{sec:orientation}.

	\section{Orientation}
In order to draw a \rubik{} on the screen there is to things which are needed; the position of each \cubie{} and the orientation of each \cubie{}.

A corner \cubie{} in a given \cubicle{} can ``sit'' in this \cubicle{} in one of three ways.
These different orientations can thereby be defined with an integer between 0 and 2.
0 if the primary \facelet{} is on the primary \face{}, 1 if the primary \facelet{} is on the secondary \face{}, and 2 if the primary \facelet{} is on the tertiary \face{}.
See the figure \ref{fig:orientation} for an illustration.

\begin{figure}[htb]
	\centering
		\subfloat[\myCaption{The white/blue/red corner is correctly oriented, which gives it the orientation value 0.}]{\label{fig:orientation:orientation0}\includegraphics[width=0.28\textwidth]{input/pics/orientation0}}
		\hspace{0.02\textwidth}
		\subfloat[\myCaption{The white/blue/red corner is incorrectly oriented, the orientation value of this is 1 since the primary color is on a secondary face.}]{\label{fig:orientation:orientation1}\includegraphics[width=0.28\textwidth]{input/pics/orientation1}}
		\hspace{0.02\textwidth}
		\subfloat[\myCaption{The white/blue/red corner is incorrectly oriented, the orientation value of this is 1 since the primary color is on a secondary face.}]{\label{fig:orientation:orientation2}\includegraphics[width=0.28\textwidth]{input/pics/orientation2}}
		\caption{\myCaption{The three different orientation which the white/blue/red cubie can have.}}
		\label{fig:orientation}
\end{figure}

The orientation of a edge \cubie{} can be defined with a boolean value; either it is correctly oriented or it is not.
This is easiest to see when the \cubie{} is in the right position e.g.
If the white/blue edge \cubie{} is in its correct \cubicle{} then it is correctly oriented if the white \facelet{} is on the white \face{} and the blue \facelet{} is on the blue \face{}.
The orientation is wrongly oriented if the blue \facelet{} is on the white \face{} and the white \facelet{} is on the blue \face{}.

This raises the question: What is the orientation of a \cubie{} which is not in its correct \cubicle{}?

	\section{Class Interaction}
In this section the different classes of the \rubik{} and their interaction with each other will be determined.
\subsection{The Cube}
The cube class creates the 20 \cubicle{}s in which the \cpiece{}s can be positioned. Furthermore the cube class creates the \face{}s and \cubie{}s. The class also gives each \cpiece{} its \facelet{}s. The \cubicle{}s are then put into a \face{} by its constructor. The cube class creates six \face{}s -- two primary, two secondary and two tertiary faces. Each type of \face{}s are placed opposite to each other. e.g. the up and down faces are both primary. The cube has a static method, \vr{permute}, it has two arguments, a cube and a sequence of \twist{}. The \twist{}s are applied to the cube.

\subsection{The Face}
The \face{} class constructor takes eight cubicles (four corners and four edges) and position them in a clockwise order. Furthermore the \face{} class constructor requires a \facelet{} which it will use as center \facelet{}. In addition the \face{} class implements two methods which twist the face clockwise or counterclockwise -- namely \textit{cwTwist} and \textit{ccwTwist} respectively.

There are three different types of \face{}s; primary \face{}s, secondary \face{}s, and tertiary \face{}s.
Each type has its own subclass, which inherits from the \face{} superclass.
The reason for the different classes is that there is a difference in how the orientation of the \cubie{}s changes depending on what type of \face{} being \twist{}ed.
%Orientation of \cubie{}s will be covered in ref\{orientation\}.

\subsection{The Cubicle}
%Måske noget om cubicle class, hvis der bliver lavet noget.
The corner and edge \cubicle{} classes each has one instance variable, namely the \cubie{} they hold. The \cpiece{} is set by the constructor of the \cubicle{} class.

\subsection{The Cubie}
The \cpiece{} class sets the orientation correctly by default, and it implements methods that return the orientation and \facelet{}s of the \cpiece{}. The corner and edge \cpiece{} classes implements two different methods which return the orientation of the \cpiece{}.

\subsection{The Facelet}
%%The \facelet{} class is an enumeration that contains the colors of the \facelet{}s, it implements a \textit{toColor} method which define the color of each \facelet{}.
The \facelet{}s are defined in an enumeration which contains the colors of the \facelet{}s, it implements a \textit{toColor} method which define the actual color of each \facelet{}.