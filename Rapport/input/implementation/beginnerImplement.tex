\chapter{Beginner's Algorithm}
\myTop{In this chapter the process of implementing the beginner's algorithm, which was described in the theory part. This implementation will not be as effective as the beginner's algorithm is capable of being. The reason for this is that we only want our implementation to solve the \cube{}, and the number of twists in this algorithm is of low priority.}

This chapter is divided into the different steps used in the beginner's algorithm.
Each step is using different algorithms which all are structured the same way. The algorithms are all using a switch/case statement, which does something different depending on the \cpiece{} to be moved. The algorithm performed is actually the same if the cube were rotated, but this is not implemented so the switch case solution will be used.

For the sake of simplicity the first layer(down face) in our implementation will always be the same face. We have chosen that face to be the yellow face, the green is the front face, and the red is the right face. This will make the implementation process easier, but the solution will require more twists since it is unlikely that the yellow face is the optimal choice as the first layer face for every solve.


\section{Step 1 -- the First layer Cross}
The first question our program needs to ask is whether the edge pieces already are positioned correctly. If that is the case and the edge piece is also oriented correctly, the program proceeds to the next edge piece. If not an algorithm is performed, which changes the edge piece's orientation without ruining other possibly correctly positioned edge pieces(This algorithm will be known as "`algorithm 1"'). When the edge piece is in the the correct position and has the correct orientation, the program moves on to another edge piece. 

If the edge \cpiece{} in question is not in it's correct position, the program needs to know where the edge is positioned. 
The way this is done is by first checking if the edge \cpiece{} is positioned in the first layer. 
If this is the case the edge \cpiece{} in question is moved to the opposite layer. 
If the edge \cpiece{} is not in the first layer it can either be in the opposite layer or in the second layer.
Is the edge piece positioned in the second layer an algorithm is performed which moves the edge piece to the white layer, which is the opposite of the layer in which we make the cross.

Now that the edge piece is in white layer it needs to be moved to the position directly above where it is correctly positioned. 
The program does this by twisting the up face once (known as an U move) and checks if the edge piece is above it's correct position.
The edge piece is now positioned in two faces -- the white face and the face which has the same color as the edge piece's second \facelet{} i.e. the face that is not yellow.
In order to position the edge piece correctly that face is twisted twice.
The edge piece is now in it's correct position, and if it is oriented correctly the program moves on to another edge piece.
If the edge piece is oriented incorrectly algorithm 1 is performed. If more edge pieces needs to be positioned or oriented correctly the program will continue with the same method until the cross is finished in which case the program moves on to the next step.

%INSERT FLOWCHART
\section{Step 2 -- completing the first layer}
This step fits the first layer corners into place. The method used for this is very similar to the methods of step 1. 
As with step one the algorithm starts with a specific cube and then try to fit this into place. 
The step is generally described in subsection \ref{sub:step2}.

First it checks whether the cube is already in the right place, if so and it is not correctly orientated an algorithm denoted \textit{algorithm 3} will ``rotate'' this \cpiece{} in its spot until the orientation is correct. 

If the cube is somewhere else in the down layer then an algorithm will move it up to the top layer. 
By now we know that the corner \cpiece{} is either placed correct or is in the white layer. 
The up face will be twisted until the \cpiece{} is placed directly above its correct place. 
Then the same algorithm used to move the cube up will now move it down. 
\textit{Algorithm 3} is then run until the cube is oriented correctly.

\section{Step 3 -- solving the second layer}
In this step the second layer will be solved. 
This means that the edges in the second or middle layer will get placed and orientated correctly. 
In subsection \ref{sub:step3} this is step is generally described. 

The program solve this layer by going through the 4 \cubicle{}s of the second layer and finding and placing the \cpiece{} in its \cubicle{}

If the \cpiece is oriented incorrectly in the correct \cubicle{} the program will use an algorithm to move the edge \cpiece{} to the top layer and move it back to the same position -- but now the orientation will be correct. 

If the edge \cpiece{} is not in it's correctly position the program needs to know where the correct edge piece is positioned. First the program checks for it in the top layer. 
If it is in the top layer up moves are applied to the \rubik{} until the edge piece is positioned in such a way that an algorithm can be applied to move it to it's correct position with the correct orientation. 
If the \cpiece{} is in the second layer but in a wrong \cubicle{} the same algorithm used to put it into place will get it out of place. 



\section{Step 4 --  getting the last layer cross}
In the previous steps the algorithms has solved one \cpiece{} at a time. 
This is not an option for this step. 
Here several \cpiece{}s will be placed at a time. 

This step will make the orientation of the top face cross correct. 
Placing the \cpiece{}s correctly will first be done in the next. 
This is done by checking each case of possible orientation for the edge \cpiece{}s in the top layer. 
There are 3 primary positions where different actions should be taken. See figure \ref{fig:cross}. 
\begin{enumerate}
\item L - shape
\item None is up.
\item A row
\end{enumerate}
It is \textit{algorithm 8} that needs to be performed no matter the case. In case one the algorithm is applied twice, case two the algorithm is applied three times, case 3 the algorithm is applied 1 time. 

\ref{fig:cross}



\section{Step 5 -- Completing the Last Layer Cross}
In the theoretical description of the beginner's algorithm step five is the implementation's step five, six and seven. 
In this step the edges of the last layer cross will be positioned correctly. Their orientation will not be changed, since they are already oriented correctly. See subsection \ref{sub:step5}.

The four \cpiece{}s in the white layer cross can only be in the four \cubicle{}. This leaves us with essentially two cases. Either all \cpiece{}s are placed correctly or two are placed correctly. This requires twisting the up face until one of these cases will appear.

The program only needs to respond to the case where two \cpiece{}s are placed correctly. 
If the the two correctly positioned edges are positioned directly across each other \textit{algorithm 9} is performed.
The correctly positioned edge \cpiece{}s are now on two faces next to each other. This is either reached by \textit{algorithm 9} or they were already positioned this way.
The program checks which two edge \cpiece{}s are positioned correctly and performs \textit{algorithm 9} accordingly. 
The edges are now positioned and oriented correctly and the program moves on to the next step.
<<<<<<< .mine
\section{Step 6 -- placing the last layer corners}
In this step the corners in the last layer will be positioned correctly. 
The basic theory of the cases in this step is similar to the ones in step five.

The initial action performed is a check of the number of correctly positioned corner \cpiece{}s. 
If all four corner \cpiece{}s are positioned correctly the program moves on to the next step.
If none of the corners are positioned correctly \textit{algorithm 10} is performed. This will position one of the corner \cpiece{}s correctly.
When one corner is positioned correctly, then the program will perform \textit{algorithm 10} based on what corner is positioned correctly.
Now the program will perform a check again to see if all the corners are positioned correctly. If all the four corners are not positioned correctly \textit{algorithm 10} will be performed again.
When that is done all the four corner \cpiece{}s are positioned correctly and we can move on to the last step.
=======
\section{Step 6 -- placing the last layer corners}
In this step the corners in the last layer will be positioned correctly. 
The basic theory of the cases in this step is similar to the ones in step five.

The initial action performed is a check of the number of correctly positioned corner \cpiece{}s. 
If all four corner \cpiece{}s are positioned correctly the program moves on to the next step.
If none of the corners are positioned correctly \textit{algorithm 10} is performed. This will position one of the corner \cpiece{}s correctly.
When one corner is positioned correctly, then the program will perform \textit{algorithm 10} based on what corner is positioned correctly.
Now the program will perform a check again to see if all the corners are positioned correctly. If all the four corners are not positioned correctly \textit{algorithm 10} will be performed again.
When that is done all the four corner \cpiece{}s are positioned correctly and we can move on to the last step.
>>>>>>> .r884

\section{Step -- Solving last layer orientation}
This is the last step and when the algorithm gets to this step everything is placed correctly, the corners of the last layer only needs to be orientated correctly.

The algorithm takes one corner at a time and checks this orientation. This is alway the up right front corner. If it orientated wrongly algorithm 10 is applied, this algorithm rotates a corner without ruining the rest of the \cube{}. When the corner is correctly orientated the U face is \twist{}ed so a new corner is placed in this position and the same procedure is applied. After four U turns and appliance of algorithm 10 the cube is finally solved. 



\begin{comment}
In the last layer the corners in the last layer were poisitend correctly but not oreiented. In this step will the coreners be oreinted correctly and as result it will lead to that the \rubiks{} will be solved.

This step is very simple because it there is only four corners to control and either the corner is oriented correctly or is isn't.  

The program vil first control that the front-right-up corner is oriented correctly if not the will use the an algorithm twice and after the program vil control the corner again if the corner is not oriented correctly his time the use teh algorithme and will continue with this until the corner is oriented correctly. 
Then the corner is orented correctly the program will make at up move ("U") and will control the new corner and the program will do this with every corner in the last layer until they are oriented corectly.  

\end{comment}
%husk flow diagram 

\section{Twist shortening}
The implementation of the beginners algorithm is not the most twist wise efficient so in order to reduce the amount of twist used can easily be reduced. Based on the group theory (ref) it is known that some moves can be reduced. This is inverse moves and  three or two equal moves  in a row. e.g.  the twists U U U is equivalent to U'. Also U' and U is each others inverses and are equivalent to doing nothing. 