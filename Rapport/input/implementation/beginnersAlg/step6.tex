\section{Step 6 -- placing the last layer corners}
In this step the corners in the last layer will be positioned correctly. 
The basic theory of the cases in this step is similar to the ones in step five.

The initial action performed is a check of the number of correctly positioned corner \cpiece{}s. 
If all four corner \cpiece{}s are positioned correctly the program moves on to the next step.
If none of the corners are positioned correctly \textit{algorithm 10} is performed. This will position one of the corner \cpiece{}s correctly.
When one corner is positioned correctly, then the program will perform \textit{algorithm 10} based on what corner is positioned correctly.
Now the program will perform a check again to see if all the corners are positioned correctly. If all the four corners are not positioned correctly \textit{algorithm 10} will be performed again.
When that is done all the four corner \cpiece{}s are positioned correctly and we can move on to the last step.