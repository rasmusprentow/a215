\section{Step 5 -- solving the white cross edge}
In the previous section the white layer cross was solved, but the edge piece were not positioned in the correct position. In this step the edge pieces will be solved all at once as opposed to previous sections where the program has solve the \cpiece{} one at the time. 

In this step the edge pieces in the white cross will be positioned correctly. First every case of possible position of the edge pieces without any interest in orientation of the edge piece, as result of the previous step there only were looking at orientation instead of position. 

The program will look at how the edge piece can be positionen in wery case. There is only three main possible position for the edge piece and it will use the same algorithm in all three steps. 
The first of the three positions is that two of the edge piece are positend side by side if this is the case the algorithme vil positend the last two edge piece in the lase layer. 
The second position is if the to edge piece are positend opposite of each other in the last layer and again the same algorithm vil be used so the edge piece will be posited side by side as in the previous example and the procedural will be the same as in the previous example. 
The last position is the edge piece in the white cross are positend correctly already in this case the program will continue to next step in solving the \rubik{}.

