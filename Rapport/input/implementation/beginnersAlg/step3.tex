In this step the second layer will be solved. 
This means that the edges in the second layer (middle layer) will be positioned and orientated correctly. 
In subsection \ref{sub:step3} this step is generally described. 

The method solves this layer by going through the four \cubicle{}s of the second layer and finding and placing the \cpiece{} in its \cubicle{}

If the \cpiece{} is orientated incorrectly in the correct \cubicle{} the program will use \vr{algorithm 3A}  to move the edge \cpiece{} to the top layer.

If the edge \cpiece{} is not in its correct position the program needs to know where the correct edge \cpiece{} is positioned. First the program checks for it in the top layer. 
If it is in the top layer \m{U} moves are applied to the \rubik{} until the edge \cpiece{} is positioned in such a way that \vr{algorithm 3B} or \vr{algorithm 3C} can be applied to move it to the correct position with the correct orientation. 
If the \cpiece{} is in the second layer but in a wrong \cubicle{}, \vr{algorithm 3A} will move it to the top layer. 
Then it can be positioned using \m{U} moves so that \vr{algorithm 3B} or \vr{algorithm 3C} can move it to its correct \cubicle{}.