\section{Step 2 -- completing the first layer}
This step fits the first layer corners into the correct position and orientation. The method used for this is very similar to the methods of step 1. 
As with step one the algorithm starts with a specific \cpiece{} and then try to fit this into place. 
The step is generally described in subsection \ref{sub:step2}.

First it checks whether the cube is already in the right position, if so and it is not correctly orientated an algorithm named \textit{algorithm 3} will ``rotate'' this \cpiece{} in its \cubicle{} until the orientation is correct. 

If the \cpiece{} is somewhere else in the down layer it will be moved to the top layer with \textit{algorithm 4}. 
By now we know that the corner \cpiece{} is either placed correct or is in the white layer. 
The up face will be twisted until the \cpiece{} is placed directly above its correct place. 
Then \textit{algorithm 4} that was used to move the \cpiece{} up will now move it down. 
\textit{Algorithm 3} is then run until the \cpiece{} is orientated correctly.
