\section{Orientation}
In order to draw a \rubik{} on the screen there is to things which are needed; the position of each \cubie{} and the orientation of each \cubie{}.

A corner \cubie{} in a given \cubicle{} can ``sit'' in this \cubicle{} in one of three ways.
This is illustrated with an example:


The orientation of a edge \cubie{} can be defined with a boolean value; either it is correctly oriented or it is not.
This is easiest to see when the \cubie{} is in the right position e.g.
If the white/blue edge \cubie{} is in its correct \cubicle{} then it is correctly oriented if the white \facelet{} is on the white \face{} and the blue \facelet{} is on the blue \face{}.
The orientation is wrongly oriented if the blue \facelet{} is on the white \face{} and the white \facelet{} is on the blue \face{}.

This raises the question: What is the orientation of a \cubie{} which is not in its correct \cubicle{}?
