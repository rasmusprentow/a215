\section{Method Solve}

The method Solve is the method that connects all the submethods into one big method that apply the Kociemba's algorithm onto the \rubik{}. It start with initializing five variables; a result array which will contain the final move sequence, a d integer which will determine the search depth, a l integer which also is used to determine the search depth, a b array which will contain the move sequence until the \rubik{} enters H and a c array which will be the move sequence after the \rubik{} has entered H. Thereafter it enters a while loop which initialize an array of moves everytime the loop runs. Thereafter the method enters a while loop which runs in a endless loop where it permutes with the move sequence b. The permute method applies the moves to the cube. After that it will try solve the \rubik{} from H with the method of the same name, solveFromH, with the parameter l - d.  This will return a result that will be putted in the c array. Thereafter it test if the length d + c's length is lower than the length of l. If that is true the case l will be set as the product of d + c and the result array will be initialized with l. In The result array the move sequence b and the movesequence c is added. after it has tested the length the \rubik{}  permutes the b move sequence back one depth and perform the increaseWithSNotEndingWithA method which will be described in section BLABLA( REF HER!!!!!!!!). After the increaseWithSNotEndingWithA method the program counts the d integer up by one. When the program is unable


\begin{verbatim}
try {
				while (true) {
					Cube.permute(cube, b);
					try {
						c = solveFromH(l - d);
						if (d + c.length < l) {
							l = d + c.length;
							result = new MoveButtons[l];
							output.addTextln("The solutions of the length " + l + ". The solution is:");
							int j = 0;
							for ( ; j < d; j++) {
								result[j] = b[j];
								output.addText(b[j] + " ");
							} 
							for (int k = 0 ; k < c.length; k++,j++) {
								result[j] = c[k];
								output.addText(c[k] + " "););
							}
							output.addTextln("");
						}

					} catch (InvalidCube e) {
					}
					
					Cube.permute(cube, MoveButtons.inverseOf(b));
					MoveButtons.inverseOf(b);					
					
					increaseWithSNotEndingWithA(b, b.length-1);
				}
			} catch (UnableToIncreaseMoveSequenceException e) {

			}
			d++;
		}
		return result;
	}
\end{verbatim}