\section{Orientation}
\label{sec:orientation}

\begin{figure}[h]
	\centering
		\includegraphics[scale=0.5]{input/pics/twistOfUpFace}
	\caption{\myCaption{A twist of a face will not only affect the twisted face but also its adjacent faces. }}
	\label{fig:twistOfUpFace}
\end{figure}

A \face{} will move the \cpiece{}s from one \cubicle{} to another. See figure \ref{fig:twistOfUpFace}. By performing a \twist{} the \face{}s adjacent to the twisted \face{} are also affected. 
E.g. when \twist{}ing the up face of a \rubik{} the \cubie{}s adjacent to the up face in R, L, B, and F faces will be permuted.

However the problem is that it requires variables to determine how the \cubie{}s are orientated. 
In order to draw a \rubik{} on the screen there are two things which are needed; the position of each \cubie{} and the orientation of each \cubie{}. This section will deal with the orientation of a \cubie{}.

The way to define an edge \cubie{} differs from the way to define a corner \cubie{}, which is why these to are dealt with separately.

\subsection{Corner Cubie}
A corner \cubie{} in a given \cubicle{} can be orientated in this \cubicle{} in three different ways. See figure \ref{fig:orientation} for the three different orientations of the white/blue/red corner \cubie{}.
Since each corner \cubie{} has one of each \facelet{} type -- one primary, one secondary, and one tertiary -- the orientation can be defined from where one of these \facelet{}s are positioned.
These different orientations can thereby be defined with an integer between 0 and 2.
0 if the primary \facelet{} of the \cubie{} is on the primary \face{}, 1 if the primary \facelet{} is on the secondary \face{}, and 2 if the primary \facelet{} is on the tertiary \face{}.

\begin{figure}[htb]
	\centering
		\subfloat[\myCaption{The white/blue/red corner is correctly orientated, which gives it the primary orientation value 0.}]{\label{fig:orientation:orientation0}\includegraphics[width=0.28\textwidth]{input/pics/orientation0}}
		\hspace{0.02\textwidth}
		\subfloat[\myCaption{The white/blue/red corner is incorrectly orientated, the primary orientation value is 1 since the primary color is on a secondary face.}]{\label{fig:orientation:orientation1}\includegraphics[width=0.28\textwidth]{input/pics/orientation1}}
		\hspace{0.02\textwidth}
		\subfloat[\myCaption{The white/blue/red corner is incorrectly orientated, the orientation value is 2 since the primary color is on a tertiary face.}]{\label{fig:orientation:orientation2}\includegraphics[width=0.28\textwidth]{input/pics/orientation2}}
		\caption{\myCaption{The three different orientations, which the white/blue/red cubie can have. In these figures the white and yellow faces are primary, the blue and green are secondary, and red and orange are tertiary.}}
		\label{fig:orientation}
\end{figure}

In order to make it easier to draw the \rubik{} a secondary orientation is added. This orientation is defined from which \face{} the secondary \facelet{} is on.
This orientation is also defined from an integer between 0 and 2.
0 if the secondary \facelet{} of the \cubie{} is on the primary \face{}, 1 if the secondary \facelet{} is on the secondary \face{}, and 2 if the secondary \facelet{} is on the tertiary \face{}.
Note that a correctly orientated \cubie{} will have primary orientation 0 and secondary orientation 1.
It will also be easy to define a tertiary orientation of a \cubie{} by where the tertiary \facelet{} is positioned, but this will be redundant since this integer will always have the remaining value, e.g. the primary orientation is 0, the secondary orientation is 1, then the tertiary orientation must be 2.
See figure \ref{fig:orientationFlow} for a flowchart on how to find the primary and the secondary orientations of a corner \cubie{}.

\subsection{Edge Cubie}
The orientation of an edge \cubie{} can be defined with a boolean value; either it is correctly orientated or it is not.
This is easiest to see when the \cubie{} is in the right position.
If the white/blue edge \cubie{} is in its correct \cubicle{} then it is correctly orientated if the white \facelet{} is on the white \face{} and the blue \facelet{} is on the blue \face{}.
The \cubie{} is wrongly orientated if the blue \facelet{} is on the white \face{} and the white \facelet{} is on the blue \face{}.

This raises the question: What is the orientation of a \cubie{} which is not in its correct \cubicle{}?
This was not too difficult with corner \cubie{}s because they always had a \facelet{} of each type, which an edge \cubie{} does not. An edge \cubie{} can be one of the following combinations of \facelet{} types: Primary/secondary, primary/tertiary, or secondary/tertiary. With this knowledge it is possible to look at the ``highest'' \facelet{}, meaning the \facelet{} with the highest rank.
The rank refers to the type of \facelet{}, where primary is the highest, secondary the second highest rank, and tertiary the lowest.
The orientation is then defined as the following: If the highest \facelet{} of a \cubie{} is on the highest \face{}, then the \cubie{} is correctly orientated.
In order to stay in connection with corner \cubie{} the correct orientation of a \cubie{} is given the number 0 and the incorrect orientation the number 1.
Note that this is opposite to what is usually done in programming where 1 is true and 0 is false \cite{boolean2}.
See the flowchart in figure \ref{fig:orientationFlow}.
\begin{figure}[hbp]
	\centering
		\includegraphics[width = \textwidth, trim = 10mm 150mm 0mm 10mm, clip]{input/pics/orientationFlow}
	\caption{\myCaption{This flowchart illustrates how to obtain the orientation of a cubie.}}
	\label{fig:orientationFlow}
\end{figure}