\section{Definition classes (forslag accepteres)}
In order to get a good interpretation of the aspects of the \cube{}, then the different positions, faces and moves must be defined in an easy accessible way for both humans and computers. We solved this by defining three enumeration classes. 

One containing the possible positions and one defining the possible moves and one defining the faces.

\subsection{Moves}
\label{sub:moves}
The enumeration class that contains all possible moves actually contains a property for each button in the GUI. Confusing as this might seem it actually saved a lot of time and worked quite well. Of course the permute method of the cube can be called with a button which is not a move and cause an error. This is not a robust program but a proof-of-concept. 
The moves are named from a human perspective where the prime (\m{'}) is replaces by a ``P'', e.g. \m{U'} is \vr{UP}.

\subsection{Positions}
The different positions are defined where the faces are based. There are two types of positions; corners and edges.
An edge position could be named \vr{P1S0}; which means \textit{primary\_1} and \textit{secondary\_0}.
In human terms this will be the down front edge.
Corners have three faces and the corner immediately to the right of the edge would be \vr{P1S0T1}, which refers to the corner on the \textit{primary\_1}, \textit{secondary\_0}, and \textit{tertiary\_1} faces.
