\section{Proof of Magic Constant}
\label{sec:proofOfMagicConstant}
\begin{theorem}[\textsc{\textbf{Magic Constant}}]
	A hyper cube is a term that covers both the \msquare{} and the \mcube{}. In theory the numbers of dimensions of a hyper cube can be any positive integer, the illustration of a hyper cube of any dimension higher than 3 has to be an abstraction. It is still possible to compute the magic constant of a hyper cube of any dimension $d$ of the order $n$ using the function in equation \ref{eq:magicConstantD}:

	\begin{align}
	\label{eq:magicConstantD}
		M \left( n,d \right) = \frac{n^d \cdot \left( n+1 \right)}{2}
	\end{align}
\end{theorem}
\begin{proof}
	The proof of this function resembles that of the function for 2 dimensions -- which is the magic Square(See section \ref{sec:magicSquare}).
	
	First of all a hyper cube of $d$ dimensions and the order $n$, contains the integers from 1 through $n^d$.
	
	The magic constant of the given hyper cube can be obtain by calculating the sum of any line of numbers (be that a row, column, pilar or any other line that is appropriate for the given dimension). This sum can than be multiplied by $n^{d-1}$ which is the same as adding all the numbers in the hyper cube together, since you add one dimension's magic constant's together every time $n$ is multiplied. Therefore we can write:
	
	\begin{align*}
		n^{d-1} \cdot M \left( n \right) & = \sum ^{n^d}_{i = 1} i = 1 + \cdots + n^d \\
		n^{d-1} \cdot M \left( n \right) & = \frac{n^d \cdot \left( n^d + 1 \right)}{2} \\
		M \left( n \right) & = \frac{n \cdot \left( n^d + 1 \right)}{2} & \qedhere
	\end{align*}
\end{proof}