%\documentclass{report}\begin{document}
\chapter{Group Theory}

\myTop{In this chapter we will explain group theory and how it can be used to solve the \rubik{}. When we have a good understand of the group theory we will be able to use it later for our own program.
}

\section{Permutations}
In this section we will use the same move notations as in \ref{moveNotation}.
To calculate how many positions the cube can be placed in, we have to look at the general cube terminology \ref{generalNotation}.
As stated in general cube terminology there is 8 corners \cpiece{} and the first corner you place can be placed in 8 positions, and after that is placed the next corner \cpiece{} can be placed in one of the 7 positions left, since we already use 1, and so on. So that means that the corner pieces can be placed in $8*7*6*5*4*3*2*1=8!$ Now the 8 corner \cpiece{} is placed at the right position they might not have the right orientation since a corner \cpiece{} have three different colors and therefor 3 different orientations, so there is $3^8$ orientations of the 8 corner \cpiece{}. That means there is $8!*3^8$ ways the corner \cpiece{} can be placed. As stated in the terminology there is 12 edge \cpiece{}. These edge \cpiece{}s can be positioned in 12 different positions and every edge \cpiece{} can be oriented in two different ways. So there is $2^12$ different orientations and $12!$ different positions of the edge \cpiece{}. This gives us $2^{12}*12!$ different edge permutations and a total of $3^8*2^{12}*12!*8!$

\section{Definition of the Rubik's Cube group}
To fully understand if the \rubik{} can be described as group theory, we will have to understand what a group is.
The group we have choosen to look at is the (G, *) group. This group consists of a set G and an operation *. (ref til gruppen (G, *) her, for bedre at forstaa den)

To describe the \rubik{} with group theory we will take a set of moves and make them into a group, which we will call (G, *). G is the possible moves of the \rubik{}.

G defines all the moves of the \rubik{} that is possible. Group operations can be defined the following way: M1 is a move and M2 is a move, so therefore M1*M2 is a move where you have to do the M1 move first, and then the M2 move. To prove that the \rubik{} is a group there is four points that must be correct.

\begin {itemize}
\item The element G is underneath * because M1 and M2 are moves  and M1*M2 is also a move.

\item e is a empty move (which does not change the configuration of the \rubik{}), So if you have to do the move M*e that basicly means that you have the move M and then do nothing, so that means that $M*e=M$

\item If M is a move then it is possible to reverse this move, this moved is called M'. Therefore $M*M' = e$, so every elements in G has a reverse move.

\item To prove * is associative it is important to remember that the moves made on the \rubik{} can be defined on the changes it makes to the configuration of the \rubik{}. 
If c is an oriented rubik's cubie, M(c) will be the orientation c for the oriented cubicle c ends in after the move is applied.
Example the move R wll move the ur cubie to the br cubicle, so therefore $R(ur)=br$. If there is more than one move sequence then the operation will look like this $B'(R(ur))=db$. If there is another move the cubie will be oriented in the M2(M1(c)), therefore $(M1*M2)(c)=M2(M1(c))$. 

\end {itemize}

The multiplications operator is used because the \rubik{} movements is not commutative and the addition operator is used with commutative elements which is the reason that can not be used. (evt matrix eksempel) KILDE HER.

* is associative (samme som not commutative?) because $(M1*M2)*M3 = M1*(M2*M3)$ for any moves M1, M2 and M3. $(M1*M2)*M3$ and $M1*(M2*M3)$ does the same operation to every cubie. This is the same as saying $[(M1*M2)*M3](C)=[M1*(M2*M3)](C)=M3(M2(M1(C)))$ for any cubie C. Therefore * is associative.

\section{Subgroup}
As said in the Permutation section, there is $3^8*2^{12}*12!*8!$ possible configurations.




\section{The Symmetric Group}

instead of than just looking at configurations of 8 cubies, the configurations of the cube can be seen as $n$ objects. 
these objects be named $1, 2, . . . , n,$ these names are arbitrary. the arranging of these objects can be seen as
putting them into n slots. If the slots is numberet $1, 2, . . . , n,$ can it be define as a function $\sigma : {1, 2, . . . , n} \rightarrow
{1, 2, . . . , n}$ by letting $\sigma(i)$ be the number put into slot i.

\subsection{Example 5.1.} 
Tag the objects $1, 2, 3$ in the order $3 1 2$. So, it corresponds to the function $\sigma: {1, 2, 3} \rightarrow {1, 2, 3}$
defined by $\sigma(1) = 3$,$\sigma(2) = 1$, and $\sigma(3) = 2$.

\subsection{Lemma 5,2}
Imagine that x\neq y. Since a number cannot be in more than one slot, if $x \neq y$, slots x and y must contain
different numbers. That is, $\sigma(x) \neq \sigma(y)$. Therefore, $\sigma$ is one-to-one.

Any number $y \in {1, 2, . . . , n}$ must lie in some slot, say slot x. Then, $\sigma(x) = y$.

On the other hand, if $\sigma: {1, . . . , n} \rightarrow {1, . . . , n}$ is a bijection, then $\sigma$ defines an arrangement of the n
objects: just put object $\sigma(i)$ in slot i. So, the set of possible arrangements is really the same as the set of
bijections ${1, . . . , n} \rightarrow {1, . . . , n}$. Therefore, instead of studying possible arrangements, we can study these
bijections.


\myTail{
}
%\end{document}