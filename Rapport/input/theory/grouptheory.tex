%\documentclass{report}\begin{document}
\chapter{Group Theory (\textit{Not done})}

\myTop{In this chapter we will explain group theory and how it can be used to solve the \rubik{}. When we have a good understand of the group theory we will be able to use it later for our own program.
}

\section{Permutations}
In this section we will use the same move notations as in \ref{sec:moveNotation}.
To calculate how many positions the cube can be placed in, we have to look at the general cube terminology \ref{sec:generalNotation}.
As stated in general cube terminology there is 8 corners \cpiece{} and the first corner you place can be placed in 8 positions, and after that is placed the next corner \cpiece{} can be placed in one of the 7 positions left, since we already use 1, and so on. So that means that the corner pieces can be placed in $8*7*6*5*4*3*2*1=8!$ Now the 8 corner \cpiece{} is placed at the right position they might not have the right orientation since a corner \cpiece{} have three different colors and therefor 3 different orientations, so there is $3^8$ orientations of the 8 corner \cpiece{}. That means there is $8!*3^8$ ways the corner \cpiece{} can be placed. As stated in the terminology there is 12 edge \cpiece{}. These edge \cpiece{}s can be positioned in 12 different positions and every edge \cpiece{} can be oriented in two different ways. So there is $2^12$ different orientations and $12!$ different positions of the edge \cpiece{}. This gives us $2^{12}*12!$ different edge permutations and a total of $3^8*2^{12}*12!*8!$

\section{Definition of the Rubik's Cube group}
To fully understand if the \rubik{} can be described as group theory, we will have to understand what a group is.
The group we have choosen to look at is the $(G, *)$ group. This group consists of a set $G$ and an operation $*$. (ref til gruppen (G, *) her, for bedre at forstaa den)

To describe the \rubik{} with group theory we will take a set of moves and make them into a group, which we will call $(G, *)$.

$G$ defines all the moves of the \rubik{} that is possible. Group operations can be defined the following way: $M\_1$ is a move and $M\_2$ is a move, so therefore $M\_1 * M\_2$ is a move where you have to do the $M\_1$ move first, and then the $M\_2$ move. To prove that the \rubik{} is a group there is four points that must be correct.

\begin {itemize}
\item The element $G$ is underneath $*$ because $M\_1$ and $M\_2$ are moves  and $M\_1 * M\_2$ is also a move.

\item $e$ is an empty move (which does not change the configuration of the \rubik{}), So if you have to do the move $M * e$ that basicly means that you have the move $M$ and then do nothing, so that means that $M*e=M$

\item If $M$ is a move then it is possible to reverse this move, this moved is called $M'$. Therefore $M*M' = e$, so every elements in $G$ has a reverse move.

\item To prove * is associative it is important to remember that the moves made on the \rubik{} can be defined on the changes it makes to the configuration of the \rubik{}. 
If c is an oriented rubik's cubie, M(c) will be the orientation c for the oriented cubicle c ends in after the move is applied.
Example the move R wll move the ur cubie to the br cubicle, so therefore $R(ur)=br$. If there is more than one move sequence then the operation will look like this $B'(R(ur))=db$. If there is another move the cubie will be oriented in the M2(M1(c)), therefore $(M1*M2)(c)=M2(M1(c))$. 

\end {itemize}

The multiplications operator is used because the \rubik{} movements is not commutative and the addition operator is used with commutative elements which is the reason that can not be used. (evt matrix eksempel) KILDE HER.

* is associative (samme som not commutative?) because $(M1*M2)*M3 = M1*(M2*M3)$ for any moves M1, M2 and M3. $(M1*M2)*M3$ and $M1*(M2*M3)$ does the same operation to every cubie. This is the same as saying $[(M1*M2)*M3](C)=[M1*(M2*M3)](C)=M3(M2(M1(C)))$ for any cubie C. Therefore * is associative.

\section{Subgroup}
As said in the Permutation section, there is $3^8*2^{12}*12!*8!$ possible configurations.




\section{The Symmetric Group}

instead of than just looking at configurations of 8 cubies, the configurations of the cube can be seen as $n$ objects. 
these objects be named $1, 2, . . . , n,$ these names are arbitrary. the arranging of these objects can be seen as
putting them into n slots. If the slots is numberet $1, 2, . . . , n,$ can it be define as a function $\sigma : {1, 2, . . . , n} \rightarrow
{1, 2, . . . , n}$ by letting $\sigma(i)$ be the number put into slot i.

\subsection{Example 5.1.} 
Tag the objects $1, 2, 3$ in the order $3 1 2$. So, it corresponds to the function $\sigma: {1, 2, 3} \rightarrow {1, 2, 3}$
defined by $\sigma(1) = 3$,$\sigma(2) = 1$, and $\sigma(3) = 2$.

\subsection{Lemma 5,2}
<<<<<<< .mine
Imagine that $x\neq y$. Since a number cannot be in more than one slot, if $x \neq y$, slots x and y must contain
=======
Imagine that x $\neq$ y. Since a number cannot be in more than one slot, if $x \neq y$, slots x and y must contain
>>>>>>> .r274
different numbers. That is, $\sigma(x) \neq \sigma(y)$. Therefore, $\sigma$ is one-to-one.

Any number $y \in {1, 2, . . . , n}$ must lie in some slot, say slot x. Then, $\sigma(x) = y$.

But if $\sigma: {1, . . . , n} \rightarrow {1, . . . , n}$ is a bijection, So $\sigma$ can be used to defines the arrangement of the n
objects: just put object $\sigma(i)$ in slot i. So, the arrangements is the same as the set of
bijections ${1, . . . , n} \rightarrow {1, . . . , n}$. Therefore, instead of studying possible arrangements, the bijections can be studied instead.

\subsection{Definition 5,3}

The Symmetric Group of $n$ letters is the set of bijection from ${1,2,......n}$ to ${1,2,......n}$ with with the operation of composition.
this group of $S_n$

\subsection{Exemple 5,4}

Let $\sigma,\tau \neq S_3$ be defined as $\sigma(1)=3, \sigma(2)=1, \sigma(3)=2, \tau(1)=1,\tau(2)=3,\tau(3)=2$ so $(\sigma\tau)(1)=\tau(3)=2,(\sigma\tau)(2)=\tau(1)=1, (\sigma\tau)(3)=\tau(2)=3.$

\section{5,1 Disjoint Cycle Decomposition}

There is a more compact way of writing elements of the symmetric group; this is best explained by an
example.

\subsection{Example 5.5.} Consider $\sigma \in S\_12$ defined by

	$\sigma(1) = 12 \sigma(2) = 4 \sigma(3) = 5 \sigma(4) = 2 \sigma(5) = 6 \sigma(6) = 9
	\sigma(7) = 7 \sigma(8) = 3 \sigma(9) = 10 \sigma(10) = 1 \sigma(11) = 11 \sigma(12) = 8$

We will write $i \rightarrow j$ (i maps to j) to mean $\sigma(i) = j$. Then,

	$1 \rightarrow 12, 12 \rightarrow 8, 8 \rightarrow 3, 3 \rightarrow 5, 5 \rightarrow 6, 6 \rightarrow 9, 9 \rightarrow 10, 10 7\rightarrow 1
	2 \rightarrow 4, 4 \rightarrow 2
	7 \rightarrow 7
	11 \rightarrow 11$


This data tells us what $\sigma$ does to each number, so it defines $\sigma$. As shorthand, we write
$\sigma = (1 12 8 3 5 6 9 10) (2 4) (7) (11).$

Here, $(1 12 8 3 5 6 9 10), (2 4), (7), and (10)$ are called cycles. When writing the disjoint cycle decomposition,
we leave out the cycles with just one number, so the disjoint cycle decomposition of $\sigma$ is
$\sigma = (1 12 8 3 5 6 9 10) (2 4).$ 

Now, let's actually define what a cycle is.

\subsection{Definition 5.6}. The cycle $(i\_1 i\_2 � � � i\_k)$ is the element $\tau \in S\_n$ defined by $\tau(i\_1) = i\_2, \tau(i\_2) = i\_3, . . . , \tau(i\_k -1) =
i\_k, \tau(i\_k) = i\_1$ and $\tau(j) = j$ if $j \neq i\_r$ for any $r$. The length of this cycle is $k$, and the support of the cycle
is the set ${i\_1, . . . , i\_k}$ of numbers which appear in the cycle. The support is denoted by supp $\tau$ . A cycle of
length $k$ is also called a $k-cycle$.

\subsection{Definition 5.7}. Two cycles $\sigma$ and $\tau$ are disjoint if they have no numbers in common; that is, supp$\sigma \cap$ 
supp $\tau = \oslash$

\subsection{Lemma 5.8} Let $\sigma,\tau\in S\_n$ be cycles. If $\sigma$ and $\tau$ are disjoint, then $\sigma\tau = \tau\sigma$.

Proof. Let $i\in{1, . . . , n}.$ Since $supp \sigma\cap supp \tau = \oslash,$ there are only two possibilities:
\begin{itemize}
\item $i \notin supp \sigma and i \notin supp \tau. In this case, \sigma(i) = i and \tau(i) = i, so (\sigma \circ \tau)(i) =\tau(i) = i and
(\tau \circ \sigma)(i) = \sigma(i) = i.$

\item Otherwise, i is in the support of exactly one of $\sigma$ and $\tau$. We may suppose without loss of generality that
$i \notin supp \sigma$ and $i \in supp \tau.$ Then, $\sigma(i) = i,$ so $(\sigma \circ \tau)(i) = \tau(i).$ On the other hand, $(\tau \circ \sigma)(i) = \sigma(\tau(i)).$
Now, since $\tau(i) \in supp \tau and supp \tau \cap supp \sigma = \oslash, \tau(i) \notin supp \sigma.$ Therefore, $\sigma(\tau(i)) = \tau(i).$ So, we
again have $(\sigma\tau)(i) = (\tau \sigma)(i).$
\end{itemize}

Therefore, $(\sigma \tau)(i) = (\tau \sigma)(i) = i$ for all $i$, which shows that $\sigma \tau= \tau \sigma.$
Any $\sigma \in S\_n$ can be written as a product (under the group operation, which is composition) of disjoint cycles.
This product is called the disjoint cycle decomposition of $\sigma.$ In our example, we gave a method for finding
the disjoint cycle decomposition of a permutation.
We write the identity permutation as $1$.

\subsection{Example 5.9} $S\_2$ consists of two permutations,$ 1 and (1 2).$

\subsection{Example 5.10} Let $\sigma, \tau \in S\_6$ be defined by

$\sigma(1) = 3 \sigma(2) = 5 \sigma(3) = 4 \sigma(4) = 1 \sigma(5) = 2 \sigma(6) = 6
\tau(1) = 5 \tau(2) = 4 \tau(3) = 3 \tau(4) = 2 \tau(5) = 1 \tau(6) = 6$

In cycle notation, $\sigma = (1 3 4)(2 5) and \tau = (1 5)(2 4). Then, \sigma \tau= (1 3 2) (4 5) and \sigma \tau= (1 2)(3 4 5). We
can also easily compute \sigma^2 = (1 4 3) and \tau^2 = 1.$

\subsection{Definition 5.11} If $\sigma \in S\_n$ is the product of disjoint cycles of lengths$ n\_1, . . . , n\_r$ (including its 1-cycles),
then the integers $n\_1, . . . , n\_r$ are called the cycle type of $\sigma.$

\section{\rubik{}}

We can write each move of the Rubik's cube using a slightly modified cycle notation. We want to describe
what happens to each oriented cubie; that is, we want to describe where each cubie moves and where each
face of the cubie moves. For example, if we unfold the cube and draw the down face, it looks like

If we rotate this face clockwise by $90^\circ$ (that is, we apply the move D), then the down face looks like

So, D(dlf) = dfr because the dlf cubie now lives in the dfr cubicle (with the d face of the cubie lying in the d
face of the cubicle, the l face of the cubie lying in the f face of the cubicle, and the f face of the cubie lying
in the r face of the cubicle). Similarly, D(dfr) = drb, D(drb) = dbl, and D(dbl) = dlf. If we do the same thing
for the edge cubies, we find D = (dlf dfr drb dbl)(df dr db dl).

\myTail{
}
%\end{document}