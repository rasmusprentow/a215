\section{Subgroup}
\label{sec:subgroup}
%From section \ref{sec:permutations} it is given that there is $3^8\cdot2^{12}\cdot12!\cdot8!$ permutations, but it is not all of these permutations that are possible to do. To make it more easy there will only be looked at some moves of the \rubik{} but only the moves of the down and right faces. To better understand $G$, it can be split up in small pieces.
In this section it will be proven that a subset of a group is also a subgroup.
A nonempty subset \m{H} of the group $(\m{G},*)$ is called a subgroup of \m{G} if $(\m{H},*)$ is a group.

Let $(\m{G},*)$ be a group. A nonempty subset \m{H} of \m{G} is a subgroup of $(\m{G},*)$ if, for every $a, b \in $\m{ H} then $a * b^{-1} \in  \m{H}$.

\begin{proof}
First, suppose \m{H} is a subgroup. If $b \in \m{H}$, then $b^{-1} \in \m{H}$ since $ (\m{H},*)$ is a group. So, if $a \in \m{ H}$ as well, then $a * b^{-1} \in \m{H}$.

Conversely, suppose that, for every $a, b \in \m{H}$ then $a * b^{-1} \in \m{H}$.

\begin {itemize}
\item First, notice that $*$ is associative since $(\m{G},*)$ is a group.
\item Let $a \in \m{H}$. Then, $e = a * a^{-1}$, so $e \in \m{H}$.
\item Let $b \in \m{H}$. Then $b^{-1} = e * b^{-1} \in \m{H}$, so inverse exist in \m{H}.
\item Let $a, b \in \m{H}$. By the previous step, $b^{-1} \in \m{H}$, so $a* (b^{-1})^{-1} = a* b \in \m{H}$. Thus, \m{H} is closed under $*$.
\end {itemize}

Therefore, $($\m{H}$,*)$ is a group, which means that \m{H} is a subgroup of \m{G}.
\end{proof}

