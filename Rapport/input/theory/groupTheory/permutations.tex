\section{Permutations}
\label{sec:permutations}
In this section we will use the same move notations as in \ref{sec:moveNotation}.
To calculate how many positions the cube can be placed in, we have to look at the general cube terminology \ref{sec:generalNotation}.
As stated in the general cube terminology there are 8 corner \cpiece{}s. The first corner \cpiece{} can be placed in 8 positions, the next corner \cpiece{} can be placed in one of the 7 remaining positions. %, since we already use 1, and so on.% 
This means that the corner \cpiece{}s can be placed in $8\cdot7\cdot6\cdot5\cdot4\cdot3\cdot2\cdot1=8!$ positions. 
%Now the 8 corner \cpiece{}s are placed at the right position. 
This does not mean they have the right orientation, since a corner \cpiece{} has 3 different colors and therefore 3 different orientations. This means there are $3^8$ orientations of the 8 corner \cpiece{}s, which yields $8!\cdot3^8$ corner permutations. 

As stated in the terminology there are 12 edge \cpiece{}s. These edge \cpiece{}s can be positioned in 12 different positions and every edge \cpiece{} can be oriented in 2 different ways. This yields $2^12$ different orientations and $12!$ different positions of the edge \cpiece{}s. This gives us $2^{12}\cdot12!$ different edge permutations. All of these corner and edge permutations adds up to a total of  $3^8\cdot2^{12}\cdot12!\cdot8!=519,024,039,293,878,272,000$ permutations.
