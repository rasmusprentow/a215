\chapter{Terminology}
\myTop{In order to fully understand the theoretical part of this report it is necessary to know the used terminology.}
% When adding a new entry, make sure to use the same syntax as the previous entries.
% Make sure to insert new entries at the right place, alphabetically
\section{General cube terminology}
\label{sec:generalNotation}
\begin{itemize}
\item \myTermHigh{Cuber}The self reference for people who are devoted to the community of the \rubik{}. 
\item \myTermHigh{Face}A face is an entire side of the cube. A \rubik{} has 6 faces.
\item \myTermHigh{\facet{c}}The small stickers on the cube. Each face has 9 \facet{}s.
\item \myTermHigh{\cubicle{}} A static frame that the \cpiece{} can be placed in.
\item \myTermHigh{Corner}Corner pieces have 3 \facet{}s and are placed at the corners. The \rubik{} has 8 corners. 
\item \myTermHigh{Edge}Edge pieces have 2 \facet{}s and are placed at the edges of each face. The \rubik{} has 12 edges.
\item \myTermHigh{Center}Center pieces have 1 \facet{} and are placed at the center of each face and are immovable unless the cube is turned. 
\item \myTermHigh{Turn}A turn of the cube is equal to rotating the whole cube 90 degrees -- changing view angle.
\item \myTermHigh{Twist/move}A rotation of a face.%Twisting means the rotation of a side of the cube. Often denoted as R' for exampel. This yields twist the right side counter-clockwise.
\item \myTermHigh{Move sequence} The same as a sequence of \twist{}s.
\end{itemize}

\section{Movement notation terminology}
\label{sec:moveNotation}
A cube consists of 6 faces and the notations of these are the following.
\begin{itemize}
\item \myTermHigh{Front face -- F}This face faces the cuber.
\item \myTermHigh{Left face -- L}This face faces the left hand side of the cuber.
\item \myTermHigh{Right face -- R}This face faces the right hand side of the cuber.
\item \myTermHigh{Up face -- U}This face faces up.
\item \myTermHigh{Down face -- D}This face faces down.
\item \myTermHigh{Back face -- B}This face faces away from the cuber.
\item \myTermHigh{General move -- M}Can be any of the above faces.
\end{itemize} 

A face can be twisted in two directions -- clockwise and counterclockwise. When twisting a face the direction is determined as if you were facing the face.
A twist in the clockwise direction has the same name as the face. i.e. a clockwise turn of the right face is notated R and pronounced ``right''.
A counterclockwise twist of the right face is notated R' and pronounced ``right prime''. This goes for all the faces.
A 180 degree \twist{} of a face is denoted M2. 
Beside the normal face \twist{} there are middle \twist{}s. These are denoted Mm Mm' and Mm2 e.g. Rm2 is a \twist{} of the middle section looking from the R face. This \twist{} is equal to the two \twist{} R2 and L2.  

A turn of the cube can be done in six directions. Clockwise and counterclockwise around each of the three axes.

\section{Algorithm terminology}
\begin{itemize}
\item \myTermHigh{$S$}The 18 standard \twist{}s.
\item \myTermHigh{$s$}A specific position of the \rubik{}.
\item \myTermHigh{$|s|$}The number of \twist{}s of a move sequence that transform $e$ into $s$.
\item \myTermHigh{$H$}The set of positions which obey the following:
\begin{itemize}
	\item Every \cpiece{} is correctly orientated (see section \ref{sec:orientation}.
	\item Every edge \cpiece{} not containing either an up \facelet{} or a down \facelet{} is positioned in the middle layer.
\end{itemize}
\item \myTermHigh{$A$}Set containing the following \twist{}s: \m{U U' U2 F2 R2 B2 L2 D D' D2}.
\item \myTermHigh{$G$}All positions which can be obtained without disassembling the \rubik{}. 
\item \myTermHigh{$r(s)$}The relabeling of the position $s$.
\item \myTermHigh{$u(r(s))$}The unlabeling of the relabeled position $s$.$u(r(s))=s$.
\item \myTermHigh{$S^{n}$}Set of move sequences consisting every possible sequence of max $n$ moves in $S$.
\item \myTermHigh{$S^*$}Every combination of \twist{}s in $S$.
\item \myTermHigh{$d(s)$}Distance, the shortest $|s|$.
\item \myTermHigh{$R$}set of $r(G)$.
\item \myTermHigh{$M$}48 color permutation turn and mirror.
\item \myTermHigh{$e$}The solved state of the \rubik{}(unit cube).
\end{itemize}

\subsection{Kociemba}
Here are the terms which are specifically used section \ref{sec:kociemba}. Note that positions can be considered as the shortest move sequence that transform it into $e$.
\begin{itemize}
\item \myTermHigh{$a$}A position in $G$.
\item \myTermHigh{$b$}Path from $a$ to a position in $H$ using move sequences in $S*$.
\item \myTermHigh{$c$}Path from $b$ to $e$ using move sequences in $A*$.
\item \myTermHigh{$ab$}Combination of the to move sequences $a$ and $b$, in that order.
\item \myTermHigh{$d$}Distance for phase 1.
\item \myTermHigh{$d2$}Table lookup for the distance from $ab$ to $e$.
\end{itemize}
\myTail{By reading this chapter the Foundation of understanding this report is laid.}
