\chapter{Terminology}
\myTop{In order to fully understand the theoretical part of this report it is necessary to know the used terminology.}
% When adding a new entry, make sure to use the same syntax as the previous entries.
% Make sure to insert new entries at the right place, alphabetically
\section{General Cube Terminology}
\label{sec:generalNotation}
\begin{itemize}
\item \myTermHigh{Center}Center \cpiece{}s have one \facet{} and are placed at the center of each face and are immovable unless the cube is turned. 
\item \myTermHigh{Cuber}The self reference for people who are devoted to the community of the \rubik{}. 
\item \myTermHigh{\cubicle{c}}An imaginary static frame that the \cpiece{}s can be placed in.
\item \myTermHigh{\cpiece{c}}The \cpiece{} is a small piece of the \cube{}, which has one, two, or three \facelet{}s. The \rubik{} consists of 26 \cpiece{}s.
\item \myTermHigh{Corner}Corner \cpiece{}s have three \facet{}s and are placed at the corners. The \rubik{} has eight corners. 
\item \myTermHigh{Edge}Edge \cpiece{}s have two \facet{}s and are placed at the edges of each face. The \rubik{} has 12 edges.
\item \myTermHigh{Face}A face is an entire side of the cube. The \rubik{} has six faces.
\item \myTermHigh{\facet{c}}The small stickers on the cube. Each face has nine \facet{}s.

\item \myTermHigh{Move sequence}The same as a sequence of \twist{}s.

\item \myTermHigh{Turn}A turn of the cube is equal to rotating the whole cube 90 degrees i..e. changing the view angle.
\item \myTermHigh{Twist/move}A rotation of a face.%Twisting means the rotation of a side of the cube. Often denoted as R' for exampel. This yields twist the right side counter-clockwise.

\end{itemize}

\section{Movement Notation Terminology}
\label{sec:moveNotation}
A \cube{} consists of six faces and the notations of these are the following.
\begin{itemize}
\item \myTermHigh{Front face -- F}This face faces the cuber.
\item \myTermHigh{Left face -- L}This face faces the left hand side of the cuber.
\item \myTermHigh{Right face -- R}This face faces the right hand side of the cuber.
\item \myTermHigh{Up face -- U}This face faces up.
\item \myTermHigh{Down face -- D}This face faces down.
\item \myTermHigh{Back face -- B}This face faces away from the cuber.
\item \myTermHigh{General move -- M}Can be any of the above faces.
\item \myTermHigh{Half-twist metric}In this metric of notation half-twists (\m{M2}) are allowed. This metric will be used in this report.
\item \myTermHigh{Quarter-twist metric}In this metric half-twists are not allowed.
\end{itemize} 

A face can be twisted in two directions -- clockwise and counterclockwise. When twisting a face the direction is determined as if you were facing the face.
A twist in the clockwise direction has the same name as the face. i.e. a clockwise turn of the right face is notated \m{R} and pronounced ``right''.
A counterclockwise twist of the right face is notated \m{R'} and pronounced ``right prime''. This goes for all the faces.
A 180 degree \twist{} of a face is denoted \m{M2} e.g. \m{R2} will \twist{} the right face 180 degrees, known as a half-twist.  
Beside the normal face \twist{}s there are middle \twist{}s. These are denoted \m{Mm Mm' Mm2} e.g. \m{Rm2} is a \twist{} of the middle slice looking from the \m{R} face. This \twist{} is equal to the two \twist{}s \m{R2 L2}. Middle \twist{}s will not be used unless stated.   

A turn of the cube can be done in six directions. Clockwise and counterclockwise around each of the three axes.

\section{Algorithm Terminology}
\begin{itemize}
\item \myTermHigh{$e$}The solved state of the \rubik{} (unit cube).
\item \myTermHigh{\m{S}}The 18 standard \twist{}s of the half-twist metric: \m{U U' U2 D D' D2 F F' F2 B B' B2 R R' R2 L L' L2}.
\item \myTermHigh{$s$}A specific position of the \rubik{}.
%\item \myTermHigh{$|s|$}The number of \twist{}s of a move sequence that transforms $e$ into $s$.
\item \myTermHigh{\m{H}}The set of positions which follows these rules:
\begin{itemize}
	\item Every \cpiece{} is correctly orientated (see section \ref{sec:orientation}).
	\item Every edge \cpiece{} not containing either an up \facelet{} or a down \facelet{} is positioned in the middle layer.
\end{itemize}
\item \myTermHigh{\m{A}}A set of \twist{}s containing the following \twist{}s: \m{U U' U2 F2 R2 B2 L2 D D' D2}.
\item \myTermHigh{\m{G}}All positions which can be obtained without disassembling the \rubik{}. 
\item \myTermHigh{$r(s)$}The relabeling of the position $s$.
%\item \myTermHigh{$u(r(s))$}The unlabeling of the relabeled position $s$. $u(r(s))=s$.
\item \myTermHigh{\m{S$^{n}$}}Set of move sequences consisting of every possible sequence of maximum $n$ moves in \m{S}.
\item \myTermHigh{\m{S$^*$}}Every combination of \twist{}s in \m{S}.
\end{itemize}
Note that positions can be considered as the shortest move sequence that transforms the \cube{} from $e$ to the given position $s$.

\myTail{This chapter gives some basic terms used in this report.}
