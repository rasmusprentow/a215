\section{Kociemba's Optimal Solver}
\label{sec:kociemba}
Kociemba's optimal solver is an algorithm created with the purpose of finding the twist-wise optimal solution to any scrambled \rubik{} \cite{kociemba09} \cite{cubelovers92}. It has been developed by Herbert Kociemba who has studied physics and mathematics at Technische Universit\"{a}t Darmstadt \cite{TUD} (1974-1979) and is currently teaching at a gymnasium. His interest in the \rubik{} started in the 1980 and he implemented his first algorithm -- Kociemba's Two Phase algorithm -- in 1990. See appendix  \ref{chap:emailCorrespondence} for the e-mail correspondence with Herbert Kociemba.

This section uses the following terminology:
\begin{itemize}
\item \myTermHigh{$a$}A position in \m{G}.
\item \myTermHigh{$b$}Path from $a$ to a position in \m{H} using move sequences in \m{S$^{*}$}.
\item \myTermHigh{$c$}Path from $b$ to $e$ using move sequences in \m{A$^{*}$}.
\item \myTermHigh{$ab$}Combination of the two move sequences $a$ and $b$, in that order.
\item \myTermHigh{$d$}Distance of phase 1 .
\item \myTermHigh{$d2$}Table lookup for the distance from $ab$ to $e$.
\end{itemize}
Kociemba's optimal solver consists of two phases and is based on Kociemba's Two Phase algorithm. The first phase is based on a principle called relabeling, which will be defined at first. 

%In the following the Kociemba optimal solver will be described in detiail. Kociemba's Algorithm finds a solution to a scrambled \rubik{} using two phases and is based on another algorithm by Kociemba called Two Phase Algorithm. In order to fully understand kociemba's algorithm a process called relabeling must be defined at first.

\subsection{Relabeling}
The relabeling process  starts with choosing an up face with a corresponding down face. Then choosing a front face with a corresponding back face. Each \facelet{} with the color of the up or down face is marked with the letters ``UD''.  %Each edge \cpiece{} on the front and back where the \cpiece{} does not contain a ``UD'' sticker is marked with the letters ``FB''. 
Every edge \cpiece{} that is not labeled with ``UD'' is labeled with ``FB'' on the front and back face.
Figure \ref{fig:relabelClean} shows an example.
\vspace{-1mm}
\begin{figure}[!hbt]
	\centering
		\includegraphics[scale = 0.44]{input/pics/relabelClean}
	\caption{\myCaption{A relabeled Rubik's Cube with the up and down faces as blue and green and red and orange as front and back.}}
	\label{fig:relabelClean}
\end{figure}
\vspace{-2mm}
When relabeling a \rubik{} in the position $s$ it is written as $r(s)$. A \rubik{} in any position, $s$, is said to be in the set of positions called \m{H} (see subsection \ref{sub:theSubgroupH}) if and only if $r(s)=r(e)$.
\vspace{-2mm}
\subsection{The Subgroup H}
\label{sub:theSubgroupH}
%%% Beaware if the subgroup is introduced somewhere else. it should actually be. 
The moves \m{U}, \m{U'}, \m{U2}, \m{D}, \m{D'}, \m{D2}, \m{R2}, \m{L2}, \m{F2} and \m{B2} is the set of moves \m{A}. Using moves from \m{A} on a position in \m{H}, will always result in a \rubik{} in \m{H}. The reason for this can easily be tested with a \rubik{}. If using one of the three up face moves or one of the three down face moves, the ``FB'' labels are not moved and the ``UD'' labels are simply rotated on the face that is being twisted, see figure \ref{fig:relabel2:D}. The last four moves are all very similar to each other. If any side face -- not up or down -- is \twist{}ed 180 degrees, the three \facelet{}s on the up face is moved to the down face and vice versa and thereby keeping all the ``UD'' labels on the up and down face and keeping the orientation correct. The two remaining relabeled \facelet{}s are swapped which keeps the ``FB'' labels placed and orientated correctly, see figure \ref{fig:relabel2:R2}.
\vspace{-1mm}
\begin{figure}[htb]
	\centering
	\subfloat[\myCaption{A relabeled cube that has been permutated with the move \m{D}.}]{\label{fig:relabel2:D}\includegraphics[width=0.4\textwidth]{input/pics/relabelD.PNG}}
	\hspace{0.05\textwidth}
	\subfloat[\myCaption{A relabeled cube that has been permutated with the move \m{R2}.}]{\label{fig:relabel2:R2}\includegraphics[width=0.4\textwidth]{input/pics/relabelR2.PNG}}
	\caption{\myCaption{Two positions which have been permuted with a move in \m{A}.}}
	\label{fig:relabel2}
\end{figure}
\vspace{-1mm}
%\pagebreak
\subsection{Overall Description}
\label{sub:overallDescription}
The first phase takes a scrambled cube $a$ and relabels it $r(a)$ then it finds a move sequence $b$ which will transform the relabeled cube into the subgroup \m{H}. This move will be denoted: $r(a)*b \in \m{H}$. The second phase will then determine the length from the position $ab$ to the unit position $e$ by a table lookup. 


\begin{algorithm}[!htb]                     
\caption{Kociemba's Optimal Solver \cite{rokicki09}}          
\label{alg:kociemba}        
\begin{algorithmic}[1]
\STATE $d=0$
\STATE {$l=\infty$}
\WHILE {$d<l$} 
	\FOR {$b \in S^{d}$}
		\IF {$r(ab) \in H$}
			\IF {$d + d_{2}(ab)<l$}
				\STATE {$l=d+d_{2}(ab)$}
			\ENDIF
		\ENDIF
	\ENDFOR
	\STATE {$d=d+1$}
\ENDWHILE
\STATE $l$ is now the length of the optimal solution to $a$
\end{algorithmic}
\end{algorithm}

The search distance in the algorithm $d$ (see figure \ref{fig:searchExpansion}) is initially set to zero and the total length $l$ is set to infinite. The total length is the amount of moves used to get from $a$ to $e$, while $d$ is the limited search length to find a move sequence that transforms the \rubik{} into a position in \m{H}. 
The \textbf{while} loop will run as long as $d$ is less than $l$.
\begin{figure}[!hbt]
	\centering
		\includegraphics[scale=0.75]{input/pics/searchExpansion.pdf}
	\caption{\myCaption{As the distance of the search $d$ increases, the possible move sequences expands exponentially. For each vertex on the graph there is 18 child vertices. The amount of leaves for each search depth would be $18^{d}$.}}
	\label{fig:searchExpansion}
\end{figure}

The \textbf{for} loop runs through all move sequences in the range $d$ -- recall that $\m{S}^d$ is the set containing every move sequence that uses $d$ twists. This search of moves in $\m{S}^d$ is the first phase and is further described in subsection \ref{sub:firstPhase}.
 If the move sequence transforms the \cube{} into the subgroup \m{H}, the algorithm checks whether $d + d_2(ab)$ is less than the length of the last solution. 
 If so, $d + d_2(ab)$ is the new shortest move sequence, $l$, to $e$. 
The lookup table denoted $d_2(ab)$ returns the numbers of twist required to transform a position in \m{H} to the position $e$. This is done only by using moves in \m{A}. %$d_2(ab)$ is a lookup table which takes a position in $H$ and returns the number of twists it takes to transform the position to $e$ using moves in \m{A}
The lookup table is the second phase and is further described in subsection \ref{sub:secondPhase}. 
The first time a solution is found the length of this will be set to $l$ and the solution is saved. The \textbf{while} loop will end when $d$ is incremented to $l$, then the optimal solution has the length $l$. The actual solution is $b$ plus some move sequence in $\m{A}^{d_{2}(ab)}$. Figure \ref{fig:kociemba2} illustrates the algorithm.
\begin{figure}[hb]
	\centering
	\includegraphics[scale=0.7]{input/pics/kocieambe2.pdf}
	\caption{\myCaption{The lines going from $a$ to a point in \m{H} is the move sequence denoted $b$. The lines from points in \m{H} to the point $e$ is the move sequence denoted $c$. The numbers on the lines are the number of moves. Note that the moves in $c$ is decreasing as the numbers of moves in $b$ is increasing. The line that goes directly from $a$ to $e$ is the shortest move sequence possible. }}
	\label{fig:kociemba2}
\end{figure}

\subsection{First Phase}
\label{sub:firstPhase}
The first phase finds a move sequence, $b$, from a position $a$ that transforms the \rubik{} into the subgroup \m{H}. 
This is done by going through all possible moves with a sequence of the length $d$. 
This is a breadth-first search algorithm \cite[pp. 729-731]{Rosen07}.
In figure \ref{fig:searchExpansion} the amount of elements in $\m{S}^d$ to a specific $d$ is illustrated.
An example of a search from a position $a$ is given. The algorithm starts by searching in $\m{S}^{d}$, where $d =  0$. This will give the position $a$ itself. The distance $d$ is then incremented to one. 
This will result in 18 new possible move sequences all one twist away from $a$. 
Thereafter $d$ is incremented and the algorithm now searches 18 moves from the previous positions obtained in the search where $d = 1$. 
This allows a possibility for optimization of the algorithm since some of the moves will eventually be the same. e.g. the two move sequences \m{R2} and \m{R' R'} (see section \ref{sec:groupDefinition}) is of length one and two respectively, but the result is the same transformed cube. The algorithm checks whether the transformed cube is in \m{H} by relabeling it and checking if $r(ab) = r(e)$. 

$d$ continues to be incremented until it reaches the length $l$. Since $l$ starts with the value infinite, it has to be changed in order to stop the loop. This happens in the second phase.




\subsection{Second Phase}
\label{sub:secondPhase}
The goal of the second phase is to find the length of the shortest move sequence to transform the \rubik{} from a position in \m{H} to $e$. A way to solve this problem is by having a lookup table. This table has to be very large, considering the amount of positions in \m{H}. 

The amount of positions can be calculated by imagining that one were to assemble the \rubik{}, starting with a disassembled \rubik{}. The first corner \cpiece{} can be placed in eight different \cubicle{}s. The next corner has seven possible \cubicle{}s etc. Note that all \cpiece{}s in a cube in \m{H} has the correct orientation. 
%until the last two corners which has a specific place since two corners can not be swapped. 
This gives $8! = 40320$ possibilities for the corners. 
The eight edges of the top and down layer can be placed similarly and also yields $8!$.  
The four edges of the middle layer can be placed in four different \cubicle{}s this yield $4! = 24$. Since it is impossible to swap two corners without swapping any edges the amount of possibilities is halved. 
The final result is 
\begin{equation*}
\frac{4!\cdot(8!)^{2}}{2} \approx 19.508\cdot10^{9}
\end{equation*}
elements in \m{H}.
The actual implementation of such a table will be discussed in chapter \ref{chap:kociembaImplement}.