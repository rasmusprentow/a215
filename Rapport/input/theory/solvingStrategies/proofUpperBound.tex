\section{Proof of the Upper Bound}

%INTRODUCTION

\subsection{Set Solver}
The algorithm used to prove the current lowest upper bound is known as a set solver and uses Kociemba's algorithm. The set solver is a viable method for proving the upper bound because it does not solve every single cube but a whole set of cubes at the time as the name suggests. This means that it solves approximately 19.5 billion cubes at a time.
The set solver does this by finding all the move sequences of a relabeled cube of the distance d that transforms the cube into $H$. %All the move sequences to transform the unlabeled cube to $e$ is in a database so it is fast to find the shortest move sequence, once the move sequence to $H$ has been found.

\begin{algorithm}[!h]                     
\caption{Set Solver \cite{rokicki09}}          
\label{alg:setSolver}        
\begin{algorithmic}[1]
\STATE {$f=null$}  %TJEK OM NULL VIRKER
\STATE {$d=0$}
\WHILE {true} 
	\STATE {$f = f \cup fA$ \{prepass\} }
		\IF {$f = H$}
			\STATE {return $d$} %% SKAL DETTE S�TTES F�R WHILE LOOP?
		\ENDIF
		\IF {$d \leq m$}
			\FOR {$b \in S^d, r(ab) = e $}
				\STATE {$f = f \cup ab$}
			\ENDFOR
		\ENDIF
		\IF {$f = H$}
			\STATE {return $d$}
		\ENDIF
	\STATE {$d = d + 1$}
\ENDWHILE
\end{algorithmic}
\end{algorithm}

\subsubsection{Prepass}

When the algorithm starts the prepass is set, this is done by applying all the different move sequences from $A$ to f. After this an if-statement tests if subgroup f equals subgroup H. If that is the case the distance is returned and the while loop ends. If not