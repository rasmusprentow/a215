\chapter {Graph Theory}
\label{chap:graphTheory}
\myTop{This chapter regards the graph theory in relation to the Rubik's Cube. 
There will be a description of the shortest path and calculation of the diameter of a graph, which could be a way to prove the upper bound. 
The chapter will  conclude with an example using a much smaller Rubik's Cube graph; the Middle movement graph.}

A graph consists of a set of vertices, $V$ \cite[p. 592]{Rosen07}, and a set of edges, $E$. There are several types of graphs, but in this chapter only simple graphs are described. 
This means the graph is undirected, has no loops and there can only be one edge between two vertices. 
Since all weights in the \rubik{} graph is $1$ the graph can be considered non-weighted. 

The \rubik{} graph can have different sizes depending on the allowed \twist{}. The full \rubik{} graph is the graph where all moves are allowed. 



\section{Shortest Path}
The shortest path from one vertex to another in a graph can be found by checking the length of each possible path. 
This is easy for small graph such as the one described in subsection \ref{sub:middleMoveGroup} but for bigger graph such as the full \rubik{} graph this is practically impossible. 

The shortest path between two vertices can be found with Dijkstra's Algorithm \cite[p. 651]{Rosen07}. The description of the algorithm is omitted for brevity. Dijkstra's Algorithm takes an weighted graph and two vertices as input. 
Because of this the \rubik{} graph has to be weighted. Since each \twist{} contributes to the total number of \twist{}s by the same number, one, each edge is given the weight 1. The weight will be omitted in every illustration in this chapter because they are all 1.  

\section{Solving the Diameter}
To find the diameter of a graph is to find the longest shortest path between any two vertices in the given graph i.e. the shortest path with the highest value. 
This can be done by using Dijkstra's Algorithm on every set of vertices in the graph. 
This can be applied to the \rubik{} graph as well in order to find the maximum number of \twist{}s to solve the \rubik{} in the worst case scenario. %Unfortunately this approach is not viable because of the tremendous amount of positions the \rubik{} has even when considering symmetries.

%The diameter of the \rubik{} graph can be found be finding where the upper and lower bound meet with some other method.

\section{Describing the Cube as a Graph}
In order to describe the \rubik{} as a graph it is necessary to determine the edges and the vertices. For the \rubik{} graph edges are defined to represent \twist{}s and vertices to represent the positions. 
The full \rubik{} graph is indeed a simple graph. A move can be reversed, hence no directions. In order to get from a position $a$ to an adjacent position $b$ one can only get there by a single move. 
Unless a detour is taken trough one or several other positions. 
The full \rubik{} graph consist of approximately $4.33\cdot10^{19}$ vertices and all having $18$ edges.
It is practically impossible to draw this graph. Therefore the graph will be explained with a much simpler graph; the Middle Movement graph.


\subsection{The Middle Movement Graph}
\label{sub:middleMoveGroup}
This graph is a \rubik{} graph consisting of only the moves that \twist{} the middle sections. 
This is Rm2 Fm2 Um2. Note: Rm2 = R2 L2.  This graph is fairly small, since it only consist of eight vertices and 12 edges. See figure \ref{fig:graphMiddleSlice2}. \cite[pp. 158-167]{Rubik87}

\begin{figure}
	\centering
		\includegraphics[width = \textwidth]{input/pics/graphMiddleSlice2.PNG}
	\caption{\myCaption{The graph of the middle movement positions.}}
	\label{fig:graphMiddleSlice2}
\end{figure}

Because of this graphs relatively small size the computation of the diameter is a some what simple task. It is possible to calculate the distance from all vertices to all other vertices, but since the graph is clearly symmetrical many calculations can be omitted. It is easy to see that the diameter must go from one corner to an opposite corner e.g. from the solved state to the pons asinorum\footnote{Pons asinorum is obtained from the solved state of a Rubik's Cube by the move sequence Rm2 Fm2 Um2.}. The diameter here is 3. 

In the full \rubik{} group it is believed that the analogous position to the pons asinorum, the superflip position, is the position which has the longest shortest path \cite{speedsolving.wiki}. This has never been proven. This shortest path from the superflip to the solved position is 20\cite{rokicki09}.

\myTail{A short general description of graph theory has been presented in this chapter along with a way to calculate the diameter of a graph. For better understanding of the theory, an example on applying the theory on the Rubik's Cube has been shown.}