\chapter {Graph theory}
\myTop{This chapter regards the graph theory in relation to the Rubik's Cube. Their would be a description of the shortest path and calculation of the diameter. The chapter will end in an example using a much smaller Rubik's Cube graph; the Middle movement graph.}

Graph theory  in general..


\section{Shortest path}
The shortest path from one point to another in a graph can be found by checking the length of each possible path. This is easy for small graph such as the one described in subsection \ref{sub:middleMoveGroup} but for bigger graph such as the \rubik{} graph this is practical impossible. 

The shortest path between two vertices can be found with Dijkstra's Algorithm \cite[p. 651]{Rosen07}. The description of the algorithm is omitted for brevity. Dijkstra's Algorithm takes an weighted graph and two vertices as input. Because of this the \rubik{} graph has to be weighted. Since each move contributes to the total number of moves by the same number, one, each edge is given the weight 1. The weight will be omitted in every illustration in this chapter because they are all 1.  

\section{Solving the diameter}



\section{Describing the Cube as a graph}
In order to describe the \rubik{} as a graph it is necessary to determine the edges and the vertices. In this example each edge of the graph is a move and the vertices represent the positions.  



\subsection{The middle movement graph}
\label{sub:middleMoveGroup}
This graph is a \rubik{} graph consisting of only the moves that \twist{} the middle sections. This is R2m F2m U2m. Note: R2m = R2 L2.  The graph of this group is fairly small, since it only consist of eight vertices and 12 edges. 