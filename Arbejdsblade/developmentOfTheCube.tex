\chapter{Development of the Cube}
\erno{c} is the inventor behind the world famous \rubik{}. He was born in Budapest, Hungary in 1944.  In college he studied sculpture. After his graduation he started studying architecture. Once he graduated in architecture he stayed at the college to teach interior design.

\erno{c} got the idea for the \cube{} when he wanted to make a three-dimensional design with blocks that could be manipulated individually, though this require the movement of several pieces at a time. Rubik's first attempt to make the cube used rubber band to hold the pieces in place. This attempt failed. Later he created the cube where each piece hold the other pieces in place. This resulted in a 2x2x2 cube that could \twist{} each \face{} individually. Rubik got the inspiration for the cube from the \mpuzzle{}. See chapter \ref{chap:recreationalMathematics}. At first the cube was named the \mcube{}, but when The company Ideal Toy bought exclusive rights for the \mcube{} in 1980 they had to rename it within a year in order to get trademark protection and therefore they named it the \rubik{}.

Rubik said some of the most important features behind the cube were that the parts of the cube stay together, which many other puzzle do not. He also pointed out that you can move several pieces at once. Also that it is three dimensional. 

\erno mentioned that the most important feature of the cube was that the parts of the cube was able to stay together as opposed to many other puzzles. \erno{c} also expressed his fondness for the cube's ability to move several pieces at once and the fact that the cube is three dimensional.

In \myDate{}{1}{1975} he applied for patent for his invention in Hungary. Two years later in 1977 he got the patent on the \rubik{}.

At that time there were two others applying for patent for products similar to the \rubik{}.  One of them was an American named Doctor Larry D. Nichols, and his cube was a 2x2x2 cube held together by magnets. The other patent applicant was a Japanese man named Terutoshi Ishige. He applied for patent a year after Rubik. Terutoshi Ishige's cube was almost identically to the \rubik{}. 
 
\section{The Nichols Cube Puzzle}
Dr. Larry D. Nichols has studied chemistry at DePauw University in Greencastle, Indiana, USA, before moving to Massachusetts to attend Harvard Graduate School. 
He is a lifelong puzzle enthusiast and inventor who began developing a twist cube puzzle with six colored faces in 1957. It was made of eight smaller cubes assembled to a 2x2x2 cube. The eight cubes were held together by magnets.

\begin{figure}[H]
\begin{center}
\includegraphics[scale=0.8]{\picturepath{}Nicholspatent2.png}
\caption{\myCaption{Figure of Nichols Patent.}}
\label{fig:Nicholspatent}
\end{center}
\end{figure}

On \myDate{11}{4}{1972}, he was granted U.S. Patent 3,655,201 on behalf of Moleculon Research Corp. U.S. Patent 3,655,201 covered Nichols Cube and the possibility for making larger versions later. This was two years before \erno{} took out the patent for his \rubik{}. 

In 1982 Moleculon Research Corp. sued Ideal Toy Company that had the U.S. Patent 4,378,116 for \rubik{} because they believed that Ideal Toy Company violated their patent, but the U.S. District Court ruled in Ideal Toy Company's favor. In 1986 the Court of Appeals ruled that the Pocket \rubik{} 2x2x2 was guilty of infringement, but not the 3x3x3 \rubik{}.


