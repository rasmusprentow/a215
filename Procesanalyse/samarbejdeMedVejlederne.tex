\section{Samarbejde med vejlederne}
\label{H8R}

Vores samarbejde med vores hovedvejleder har v\ae{}ret rigtig godt. 
N\aa{}r vi har sendt noget materiale til ham, har vi f\aa{}et god og hurtig respons. 
Vi har n\ae{}sten haft m\o{}der hver uge med vores hovedvejleder, og der har altid v\ae{}ret god og
konstruktiv kritik. 
Vi har f\o{}lt at vores hovedvejleder har fulgt os igennem dette projekt og vist stor interesse for emnet, samt kommet med masser af forslag til forbedringer.

Vores samarbejde med vores bivejleder har ogs\aa{} fungeret godt, og hun har ogs\aa{} givet os god feedback. 
I starten havde vi mange m\o{}der med hende, men den sidste halvanden m\aa{}ned havde vi kun et m\o{}de, da vi kun skrev teori og implementation i denne periode.

N\aa{}r vi skulle sende arbejdsblade, som vi ville have respons p\aa{}, sendte vi dem til vejlederne 48 timer f\o{}r m\o{}det. Dette gav dem tid til at l\ae{}se dem igennem.

Et eksempel af et referat fra et af vores vejlederm\o{}der se sektion \ref{ref}

\subsection{Refleksion}
Vores samarbejde med vejlederne har generelt fungeret rigtig godt, og de har bidraget meget til projektets udformning. 
Nogle gange har vi v\ae{}ret frustrerede over nogle af de ting, de har sagt. 
Fx hvis det har v\ae{}ret st\o{}rre \ae{}ndringer i strukturen af rapporten. Andre gange har vi ikke forst\aa{}et pr\ae{}cist, hvad de
mente. 
Efter langt de fleste vejlederm\o{}der har vi dog siddet med f\o{}lelse af at have et godt overblik over rapporten og hvad der skulle ske efterf\o{}lgende. 
I slutfasen burde vi have holdt nogle flere m\o{}der med vores bivejleder, da hun havde en hel del \ae{}ndringer til det sidste m\o{}de. Vi burde have holdt det sidste m\o{} med vores bivejleder noget f\o{}r, og ikke tage for god vare, at hun sagde vores kontekstuelle del var fint halvanden m\aa{}ned inden afleverings datoen.

