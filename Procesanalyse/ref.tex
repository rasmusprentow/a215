\section{Referat}
I denne sektion er et referat fra vores vejlederm\o{}de d. 17/5 med hoved- og bivejleder.
\label{ref}
\begin{itemize}
	\item Advisor
	\item 
\end{itemize}

\subsection{Problem}
\begin{itemize}
	\item Mangler context, hvor er det henne?
	\item Kunne eventuelt skrive noget mere om Kociemba
	\item Noget konkret context
	\item Noget context med til det f\o{}rste punkt af problem statement, men man kan ikke se det med det samme
	\item Mere tydeligt i tops and tails om der er context, skriv det helt konkret
	\item Flyt Upper bound grafen op til ??? (ikke problem analysis, men m\aa{}ske chapter 4) og \ae{}ndre lidt i det f\o{}rste punkt af problem statement
	\item Tag evt. en del af chap 9 med op
	\item Lav en problem analyse til at sn\ae{}vre indledeningen ind til problem statement.
	\item Hvad er form\aa{}let med projektet
	\item 
\end{itemize}


Nu snakker Heather
\begin{itemize}
	\item Parts of community, use groups instead
	\item God's algorithm, evt. italic
	\item Moving -$>$ developping
	\item Noget galt til sidst i indledning
	\item Parts in limitations, what is it?
	\item Given -$>$ chosen.
	\item page 7: both top and introduction, why? Maybe kill
	\item remove intro in chap 3
	\item page 8: hails - kill!! change to originate
	\item page 9: elaborate later -$>$ below
	\item Similarities between magic cube and rubiks cube. State them!
	\item Permutations state similarities between and magic puzzle and rubiks cube -- The picture.
	\item page 11: Explain how Erno was inspired by the magic puzzle
	\item Cubie needs to be defined.
	\item page 14: using caps, watch it!
	\item p 19: 
\end{itemize}

\subsection{Theory}
\begin{itemize}
	\item p. 39: tail, use namely and ref to implementation part
	\item Perhaps use graph and group as context
	\item How has it been used earlier
	\item Add it to tops and tails
\end{itemize}

\subsection{Implementation}
\begin{itemize}
	\item Choices prior: change ref til prob. limit til at have det inline.
	\item Relate the tail to twist-wise community
	\item How does our choices relate to this community
	\item We position us in the technical community
	\item We make a contribution to the community in the form of analysis of the two algorithms
	\item Only refer back, don't tell the reader to read it.
\end{itemize}

Anders:
\begin{itemize}
	\item Statistics of Beginners: Elaborate, graphs, time
	\item Stats Kociemba: Eloborate, look at twist, change the header, describe $R^2$
\end{itemize}

\subsection{Epilogue}
\begin{itemize}
	\item What does the improvements do, Move to perspective/future work which is after conc
	\item Conclusion: Second question, is it important/interesting?
	\item Has anyone ever compared Kociamba to beginner's?
	\item In problem analysis descibe that we are interested in the two algos and maybe remove second question in problem statement.
	\item Upper and lower bound -- Define! -- should be done already.
	\item Refer back to the report, but no citations!
	\item Present the results from our tests, perhaps add to problem statement as well
	\item Maybe combine question 2 and 3
	\item Elaborate on the different algorithms there exists
\end{itemize}

\subsection{What to do}
See Heather's paper.


Remember to print the presentation and give it at the exam